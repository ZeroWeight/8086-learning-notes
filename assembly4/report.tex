\documentclass[UTF8,a4paper]{paper}
\usepackage{ctex}
\usepackage[utf8]{inputenc}
\usepackage{amsmath}
\usepackage{pdfpages}
\usepackage{graphicx}
\usepackage{wrapfig}
\usepackage{listings}
\usepackage{color}
\usepackage{alltt}
\usepackage{marvosym}
\usepackage{xcolor}
\input {highlight.sty}
\title{计算机原理第四次实验报告}
\author{张蔚桐\ 2015011493\ 自55}
\begin {document}
\maketitle
\section{实验目的}
理解中断控制器8259 及计数/定时器8253 的工作原理,掌握其使用方法。
\section{中断控制器8259的应用}
程序首先完成关中断,完成中断矢量表的修改,之后对8259编程开启IRQ3中断,之后开中断进行计数和显示工作
\noindent
\ttfamily
\hlstd{}\hllin{01\ }\hlslc{;IRQ3\ }\\
\hllin{02\ }\hlstd{DATA\ }\hlkwa{SEGMENT}\\
\hllin{03\ }\hlstd{NUM\ }\hlkwa{DB\ }\hlstd{}\hlnum{0}\\
\hllin{04\ }\hlstd{MESS1\ }\hlkwa{DB\ }\hlstd{}\hlstr{'PRESSED'}\hlstd{}\hlopt{,}\hlstd{}\hlnum{0}\hlstd{}\hlkwb{DH}\hlstd{}\hlopt{,}\hlstd{}\hlnum{0}\hlstd{}\hlkwb{AH}\hlstd{}\hlopt{,}\hlstd{}\hlstr{'\$'}\hlstd{\\
\hllin{05\ }MESS2\ }\hlkwa{DB\ }\hlstd{}\hlstr{'DONE'}\hlstd{}\hlopt{,}\hlstd{}\hlnum{0}\hlstd{}\hlkwb{DH}\hlstd{}\hlopt{,}\hlstd{}\hlnum{0}\hlstd{}\hlkwb{AH}\hlstd{}\hlopt{,}\hlstd{}\hlstr{'\$'}\hlstd{\\
\hllin{06\ }KEEP\textunderscore CS\ }\hlkwa{DW\ }\hlstd{}\hlnum{0}\\
\hllin{07\ }\hlstd{KEEP\textunderscore IP\ }\hlkwa{DW\ }\hlstd{}\hlnum{0}\\
\hllin{08\ }\hlstd{DATA\ }\hlkwa{ENDS}\\
\hllin{09\ }\hlstd{\\
\hllin{10\ }STACK\ }\hlkwa{SEGMENT\ }\hlstd{PARA\ STACK}\\
\hllin{11\ }\hlkwa{DW\ }\hlstd{}\hlnum{10\ }\hlstd{DUP}\hlopt{(}\hlstd{?}\hlopt{)}\\
\hllin{12\ }\hlstd{STACK\ }\hlkwa{ENDS}\\
\hllin{13\ }\hlstd{\\
\hllin{14\ }CODE\ }\hlkwa{SEGMENT}\\
\hllin{15\ }\hlstd{}\hlkwa{ASSUME\ }\hlstd{}\hlkwb{CS}\hlstd{}\hlopt{:}\hlstd{CODE}\hlopt{,}\hlstd{}\hlkwb{DS}\hlstd{}\hlopt{:}\hlstd{DATA}\hlopt{,}\hlstd{}\hlkwb{ES}\hlstd{}\hlopt{:}\hlstd{DATA}\hlopt{,}\hlstd{}\hlkwb{SS}\hlstd{}\hlopt{:}\hlstd{STACK\\
\hllin{16\ }\\
\hllin{17\ }INTER\ }\hlkwa{PROC}\\
\hllin{18\ }\hlstd{}\hlkwa{INC\ }\hlstd{NUM}\\
\hllin{19\ }\hlkwa{MOV\ }\hlstd{}\hlkwb{DX}\hlstd{}\hlopt{,}\hlstd{}\hlkwb{OFFSET\ }\hlstd{MESS1}\\
\hllin{20\ }\hlkwa{MOV\ }\hlstd{}\hlkwb{AH}\hlstd{}\hlopt{,}\hlstd{}\hlnum{0920H}\\
\hllin{21\ }\hlstd{}\hlkwa{INT\ }\hlstd{}\hlnum{21H}\\
\hllin{22\ }\hlstd{}\hlkwa{MOV\ }\hlstd{}\hlkwb{AL}\hlstd{}\hlopt{,}\hlstd{}\hlnum{20H}\\
\hllin{23\ }\hlstd{}\hlkwa{OUT\ }\hlstd{}\hlnum{20H}\hlstd{}\hlopt{,}\hlstd{}\hlkwb{AL}\\
\hllin{24\ }\hlstd{}\hlkwa{IRET}\\
\hllin{25\ }\hlstd{INTER\ }\hlkwa{ENDP}\\
\hllin{26\ }\hlstd{}\\
\hllin{27\ }\hlkwc{START:}\\
\hllin{28\ }\hlstd{}\hlkwa{CLI}\\
\hllin{29\ }\hlstd{}\hlkwa{MOV\ }\hlstd{}\hlkwb{AX}\hlstd{}\hlopt{,}\hlstd{DATA}\\
\hllin{30\ }\hlkwa{MOV\ }\hlstd{}\hlkwb{ES}\hlstd{}\hlopt{,}\hlstd{}\hlkwb{AX}\\
\hllin{31\ }\hlstd{}\hlkwa{MOV\ }\hlstd{}\hlkwb{DS}\hlstd{}\hlopt{,}\hlstd{}\hlkwb{AX}\\
\hllin{32\ }\hlstd{}\hlkwa{MOV\ }\hlstd{}\hlkwb{AX}\hlstd{}\hlopt{,}\hlstd{STACK}\\
\hllin{33\ }\hlkwa{MOV\ }\hlstd{}\hlkwb{SS}\hlstd{}\hlopt{,}\hlstd{}\hlkwb{AX}\\
\hllin{34\ }\hlstd{}\hlslc{;INTER\ No.\ GET\ AND\ PROTECT}\\
\hllin{35\ }\hlstd{}\hlkwa{MOV\ }\hlstd{}\hlkwb{AH}\hlstd{}\hlopt{,}\hlstd{}\hlnum{0350}\hlstd{}\hlkwb{BH}\\
\hllin{36\ }\hlstd{}\hlkwa{INT\ }\hlstd{}\hlnum{21H}\\
\hllin{37\ }\hlstd{}\hlkwa{MOV\ }\hlstd{KEEP\textunderscore IP}\hlopt{,}\hlstd{}\hlkwb{BX}\\
\hllin{38\ }\hlstd{}\hlkwa{MOV\ }\hlstd{KEEP\textunderscore CS}\hlopt{,}\hlstd{}\hlkwb{ES}\\
\hllin{39\ }\hlstd{}\hlkwa{PUSH\ }\hlstd{}\hlkwb{DS}\\
\hllin{40\ }\hlstd{}\hlslc{;SET\ THE\ INTER\ TABLE\ }\\
\hllin{41\ }\hlstd{}\hlkwa{MOV\ }\hlstd{}\hlkwb{DX}\hlstd{}\hlopt{,}\hlstd{}\hlkwb{OFFSET\ }\hlstd{INTER}\\
\hllin{42\ }\hlkwa{MOV\ }\hlstd{}\hlkwb{AX}\hlstd{}\hlopt{,}\hlstd{}\hlkwa{SEG\ }\hlstd{INTER}\\
\hllin{43\ }\hlkwa{MOV\ }\hlstd{}\hlkwb{DS}\hlstd{}\hlopt{,}\hlstd{}\hlkwb{AX}\\
\hllin{44\ }\hlstd{}\hlkwa{MOV\ }\hlstd{}\hlkwb{AX}\hlstd{}\hlopt{,}\hlstd{}\hlnum{250}\hlstd{}\hlkwb{BH}\\
\hllin{45\ }\hlstd{}\hlkwa{INT\ }\hlstd{}\hlnum{21H}\\
\hllin{46\ }\hlstd{}\hlkwa{POP\ }\hlstd{}\hlkwb{DS}\\
\hllin{47\ }\hlstd{}\hlslc{;SET\ THE\ INTER\ MASK}\\
\hllin{48\ }\hlstd{}\hlkwa{IN\ }\hlstd{}\hlkwb{AL}\hlstd{}\hlopt{,}\hlstd{}\hlnum{21H}\\
\hllin{49\ }\hlstd{}\hlkwa{AND\ }\hlstd{}\hlkwb{AL}\hlstd{}\hlopt{,}\hlstd{}\hlnum{0}\hlstd{F7H}\hlslc{;CANCEL\ MASK\ IRQ3}\\
\hllin{50\ }\hlstd{}\hlkwa{OUT\ }\hlstd{}\hlnum{21H}\hlstd{}\hlopt{,}\hlstd{}\hlkwb{AL}\\
\hllin{51\ }\hlstd{}\hlkwa{STI}\\
\hllin{52\ }\hlstd{}\hlkwc{L1:}\\
\hllin{53\ }\hlstd{}\hlkwa{CMP\ }\hlstd{NUM}\hlopt{,}\hlstd{}\hlnum{10}\\
\hllin{54\ }\hlstd{}\hlkwa{JNZ\ }\hlstd{L1}\\
\hllin{55\ }\\
\hllin{56\ }\hlkwa{MOV\ }\hlstd{}\hlkwb{DX}\hlstd{}\hlopt{,}\hlstd{}\hlkwb{OFFSET\ }\hlstd{MESS2}\\
\hllin{57\ }\hlkwa{MOV\ }\hlstd{}\hlkwb{AH}\hlstd{}\hlopt{,}\hlstd{}\hlnum{09H}\\
\hllin{58\ }\hlstd{}\hlkwa{INT\ }\hlstd{}\hlnum{21H}\\
\hllin{59\ }\hlstd{}\\
\hllin{60\ }\hlslc{;END\ AND\ RECOVER}\\
\hllin{61\ }\hlstd{}\hlslc{;RECOVER\ THE\ INTER\ MASK}\\
\hllin{62\ }\hlstd{}\hlkwa{IN\ }\hlstd{}\hlkwb{AL}\hlstd{}\hlopt{,}\hlstd{}\hlnum{21H}\\
\hllin{63\ }\hlstd{}\hlkwa{OR\ }\hlstd{}\hlkwb{AL}\hlstd{}\hlopt{,}\hlstd{}\hlnum{008H}\\
\hllin{64\ }\hlstd{}\hlkwa{OUT\ }\hlstd{}\hlnum{21H}\hlstd{}\hlopt{,}\hlstd{}\hlkwb{AL}\\
\hllin{65\ }\hlstd{}\hlslc{;RECOVER\ THE\ INTER\ TABLE}\\
\hllin{66\ }\hlstd{}\hlkwa{PUSH\ }\hlstd{}\hlkwb{DS}\\
\hllin{67\ }\hlstd{}\hlkwa{MOV\ }\hlstd{}\hlkwb{DX}\hlstd{}\hlopt{,}\hlstd{KEEP\textunderscore IP}\\
\hllin{68\ }\hlkwa{MOV\ }\hlstd{}\hlkwb{AX}\hlstd{}\hlopt{,}\hlstd{KEEP\textunderscore CS}\\
\hllin{69\ }\hlkwa{MOV\ }\hlstd{}\hlkwb{DS}\hlstd{}\hlopt{,}\hlstd{}\hlkwb{AX}\\
\hllin{70\ }\hlstd{}\hlkwa{MOV\ }\hlstd{}\hlkwb{AH}\hlstd{}\hlopt{,}\hlstd{}\hlnum{250}\hlstd{}\hlkwb{BH}\\
\hllin{71\ }\hlstd{}\hlkwa{INT\ }\hlstd{}\hlnum{21H}\\
\hllin{72\ }\hlstd{}\hlkwa{POP\ }\hlstd{}\hlkwb{DS}\\
\hllin{73\ }\hlstd{}\\
\hllin{74\ }\hlkwa{MOV\ }\hlstd{}\hlkwb{AX}\hlstd{}\hlopt{,}\hlstd{}\hlnum{04}\hlstd{C00H}\\
\hllin{75\ }\hlkwa{INT\ }\hlstd{}\hlnum{21H}\\
\hllin{76\ }\hlstd{CODE\ }\hlkwa{ENDS}\\
\hllin{77\ }\hlstd{}\hlkwa{END\ }\hlstd{START}\\
\mbox{}
\normalfont
\normalsize

\section{计数/定时器8253的应用}
程序完成对8253的编程,硬件上将CLK0和1MHz时钟连接,OUT0接CLK1,OUT1接LED,GATE0和GATE1拉高,经双计数器分频将1MHz时钟降到1Hz控制LED闪烁。
\noindent
\ttfamily
\hlstd{}\hllin{01\ }\hlslc{;1MHz\ TO\ CLK0,\ OUT0\ TO\ CLK1,\ OUT1\ TO\ LED}\\
\hllin{02\ }\hlstd{STACK\ }\hlkwa{SEGMENT\ }\hlstd{PARA\ STACK\ }\\
\hllin{03\ }\hlkwa{DW\ }\hlstd{}\hlnum{10\ }\hlstd{DUP\ }\hlopt{(}\hlstd{?}\hlopt{)}\\
\hllin{04\ }\hlstd{}\hlkwa{END\ }\hlstd{STACK\ \\
\hllin{05\ }\\
\hllin{06\ }CODE\ }\hlkwa{SEGMENT}\\
\hllin{07\ }\hlstd{}\hlkwa{ASSUME\ }\hlstd{}\hlkwb{SS}\hlstd{}\hlopt{:}\hlstd{STACK}\hlopt{,}\hlstd{}\hlkwb{CS}\hlstd{}\hlopt{,}\hlstd{CODE}\\
\hllin{08\ }\hlkwc{START:}\\
\hllin{09\ }\hlstd{}\hlkwa{MOV\ }\hlstd{}\hlkwb{AX}\hlstd{}\hlopt{,}\hlstd{STACK}\\
\hllin{10\ }\hlkwa{MOV\ }\hlstd{}\hlkwb{SS}\hlstd{}\hlopt{,}\hlstd{}\hlkwb{AX}\\
\hllin{11\ }\hlstd{}\hlkwa{MOV\ }\hlstd{}\hlkwb{DX}\hlstd{}\hlopt{,}\hlstd{}\hlnum{0283H}\\
\hllin{12\ }\hlstd{}\hlkwa{MOV\ }\hlstd{}\hlkwb{AL}\hlstd{}\hlopt{,}\hlstd{}\hlnum{037H}\\
\hllin{13\ }\hlstd{}\hlkwa{OUT\ }\hlstd{}\hlkwb{DX}\hlstd{}\hlopt{,}\hlstd{}\hlkwb{AL}\\
\hllin{14\ }\hlstd{}\hlkwa{MOV\ }\hlstd{}\hlkwb{DX}\hlstd{}\hlopt{,}\hlstd{}\hlnum{280H}\\
\hllin{15\ }\hlstd{}\hlkwa{XOR\ }\hlstd{}\hlkwb{AL}\hlstd{}\hlopt{,}\hlstd{}\hlkwb{AL}\\
\hllin{16\ }\hlstd{}\hlkwa{OUT\ }\hlstd{}\hlkwb{DX}\hlstd{}\hlopt{,}\hlstd{}\hlkwb{AL}\\
\hllin{17\ }\hlstd{}\hlkwa{MOV\ }\hlstd{}\hlkwb{AL}\hlstd{}\hlopt{,}\hlstd{}\hlnum{010H}\\
\hllin{18\ }\hlstd{}\hlkwa{OUT\ }\hlstd{}\hlkwb{DX}\hlstd{}\hlopt{,}\hlstd{}\hlkwb{AL}\\
\hllin{19\ }\hlstd{}\hlkwa{MOV\ }\hlstd{}\hlkwb{DX}\hlstd{}\hlopt{,}\hlstd{}\hlnum{0283H}\\
\hllin{20\ }\hlstd{}\hlkwa{MOV\ }\hlstd{}\hlkwb{AL}\hlstd{}\hlopt{,}\hlstd{}\hlnum{077H}\\
\hllin{21\ }\hlstd{}\hlkwa{OUT\ }\hlstd{}\hlkwb{DX}\hlstd{}\hlopt{,}\hlstd{}\hlkwb{AL}\\
\hllin{22\ }\hlstd{}\hlkwa{MOV\ }\hlstd{}\hlkwb{DX}\hlstd{}\hlopt{,}\hlstd{}\hlnum{0281H}\\
\hllin{23\ }\hlstd{}\hlkwa{XOR\ }\hlstd{}\hlkwb{AL}\hlstd{}\hlopt{,}\hlstd{}\hlkwb{AL}\\
\hllin{24\ }\hlstd{}\hlkwa{OUT\ }\hlstd{}\hlkwb{DX}\hlstd{}\hlopt{,}\hlstd{}\hlkwb{AL}\\
\hllin{25\ }\hlstd{}\hlkwa{MOV\ }\hlstd{}\hlkwb{AL}\hlstd{}\hlopt{,}\hlstd{}\hlnum{10H}\\
\hllin{26\ }\hlstd{}\hlkwa{OUT\ }\hlstd{}\hlkwb{DX}\hlstd{}\hlopt{,}\hlstd{}\hlkwb{AL}\\
\hllin{27\ }\hlstd{}\hlkwa{MOV\ }\hlstd{}\hlkwb{AX}\hlstd{}\hlopt{,}\hlstd{}\hlnum{4}\hlstd{C00H}\\
\hllin{28\ }\hlkwa{INT\ }\hlstd{}\hlnum{21H}\\
\hllin{29\ }\hlstd{CODE\ }\hlkwa{ENDS}\\
\hllin{30\ }\hlstd{}\hlkwa{END\ }\hlstd{START}\\
\mbox{}
\normalfont
\normalsize

\section{选做部分}
将上述两端代码合并,接线仅将OUT1改接到IRQ3中断,显示自动计时输出
\noindent
\ttfamily
\hlstd{\hllin{01\ }DATA\ }\hlkwa{SEGMENT}\\
\hllin{02\ }\hlstd{NUM\ }\hlkwa{DB\ }\hlstd{}\hlnum{0}\\
\hllin{03\ }\hlstd{MESS1\ }\hlkwa{DB\ }\hlstd{}\hlstr{'PRESSED'}\hlstd{}\hlopt{,}\hlstd{}\hlnum{0}\hlstd{}\hlkwb{DH}\hlstd{}\hlopt{,}\hlstd{}\hlnum{0}\hlstd{}\hlkwb{AH}\hlstd{}\hlopt{,}\hlstd{}\hlstr{'\$'}\hlstd{\\
\hllin{04\ }MESS2\ }\hlkwa{DB\ }\hlstd{}\hlstr{'DONE'}\hlstd{}\hlopt{,}\hlstd{}\hlnum{0}\hlstd{}\hlkwb{DH}\hlstd{}\hlopt{,}\hlstd{}\hlnum{0}\hlstd{}\hlkwb{AH}\hlstd{}\hlopt{,}\hlstd{}\hlstr{'\$'}\hlstd{\\
\hllin{05\ }KEEP\textunderscore CS\ }\hlkwa{DW\ }\hlstd{}\hlnum{0}\\
\hllin{06\ }\hlstd{KEEP\textunderscore IP\ }\hlkwa{DW\ }\hlstd{}\hlnum{0}\\
\hllin{07\ }\hlstd{DATA\ }\hlkwa{ENDS}\\
\hllin{08\ }\hlstd{\\
\hllin{09\ }STACK\ }\hlkwa{SEGMENT\ }\hlstd{PARA\ STACK}\\
\hllin{10\ }\hlkwa{DW\ }\hlstd{}\hlnum{10\ }\hlstd{DUP}\hlopt{(}\hlstd{?}\hlopt{)}\\
\hllin{11\ }\hlstd{STACK\ }\hlkwa{ENDS}\\
\hllin{12\ }\hlstd{\\
\hllin{13\ }CODE\ }\hlkwa{SEGMENT}\\
\hllin{14\ }\hlstd{}\hlkwa{ASSUME\ }\hlstd{}\hlkwb{CS}\hlstd{}\hlopt{:}\hlstd{CODE}\hlopt{,}\hlstd{}\hlkwb{DS}\hlstd{}\hlopt{:}\hlstd{DATA}\hlopt{,}\hlstd{}\hlkwb{ES}\hlstd{}\hlopt{:}\hlstd{DATA}\hlopt{,}\hlstd{}\hlkwb{SS}\hlstd{}\hlopt{:}\hlstd{STACK\\
\hllin{15\ }\\
\hllin{16\ }INTER\ }\hlkwa{PROC}\\
\hllin{17\ }\hlstd{}\hlkwa{INC\ }\hlstd{NUM}\\
\hllin{18\ }\hlkwa{MOV\ }\hlstd{}\hlkwb{DX}\hlstd{}\hlopt{,}\hlstd{}\hlkwb{OFFSET\ }\hlstd{MESS1}\\
\hllin{19\ }\hlkwa{MOV\ }\hlstd{}\hlkwb{AX}\hlstd{}\hlopt{,}\hlstd{}\hlnum{0920H}\\
\hllin{20\ }\hlstd{}\hlkwa{INT\ }\hlstd{}\hlnum{21H}\\
\hllin{21\ }\hlstd{}\hlkwa{MOV\ }\hlstd{}\hlkwb{AL}\hlstd{}\hlopt{,}\hlstd{}\hlnum{20H}\\
\hllin{22\ }\hlstd{}\hlkwa{OUT\ }\hlstd{}\hlnum{20H}\hlstd{}\hlopt{,}\hlstd{}\hlkwb{AL}\\
\hllin{23\ }\hlstd{}\hlkwa{IRET\ }\\
\hllin{24\ }\hlstd{INTER\ }\hlkwa{ENDP}\\
\hllin{25\ }\hlstd{}\\
\hllin{26\ }\hlkwc{START:}\\
\hllin{27\ }\hlstd{}\hlkwa{MOV\ }\hlstd{}\hlkwb{AX}\hlstd{}\hlopt{,}\hlstd{DATA}\\
\hllin{28\ }\hlkwa{MOV\ }\hlstd{}\hlkwb{DS}\hlstd{}\hlopt{,}\hlstd{}\hlkwb{AX}\\
\hllin{29\ }\hlstd{}\hlkwa{MOV\ }\hlstd{}\hlkwb{ES}\hlstd{}\hlopt{,}\hlstd{}\hlkwb{AX}\\
\hllin{30\ }\hlstd{}\hlkwa{CLI}\\
\hllin{31\ }\hlstd{}\hlkwa{MOV\ }\hlstd{}\hlkwb{AX}\hlstd{}\hlopt{,}\hlstd{}\hlnum{0350}\hlstd{}\hlkwb{BH}\\
\hllin{32\ }\hlstd{}\hlkwa{INT\ }\hlstd{}\hlnum{21H}\\
\hllin{33\ }\hlstd{}\hlkwa{MOV\ }\hlstd{KEEP\textunderscore IP}\hlopt{,}\hlstd{}\hlkwb{BX}\\
\hllin{34\ }\hlstd{}\hlkwa{MOV\ }\hlstd{KEEP\textunderscore CS}\hlopt{,}\hlstd{}\hlkwb{ES}\\
\hllin{35\ }\hlstd{}\hlkwa{PUSH\ }\hlstd{}\hlkwb{DS}\\
\hllin{36\ }\hlstd{}\hlkwa{MOV\ }\hlstd{}\hlkwb{DX}\hlstd{}\hlopt{,}\hlstd{}\hlkwb{OFFSET\ }\hlstd{INTER}\\
\hllin{37\ }\hlkwa{MOV\ }\hlstd{}\hlkwb{AX}\hlstd{}\hlopt{,}\hlstd{}\hlkwa{SEG\ }\hlstd{INTER}\\
\hllin{38\ }\hlkwa{MOV\ }\hlstd{}\hlkwb{DS}\hlstd{}\hlopt{,}\hlstd{}\hlkwb{AX}\\
\hllin{39\ }\hlstd{}\hlkwa{MOV\ }\hlstd{}\hlkwb{AX}\hlstd{}\hlopt{,}\hlstd{}\hlnum{0250}\hlstd{}\hlkwb{BH}\\
\hllin{40\ }\hlstd{}\hlkwa{INT\ }\hlstd{}\hlnum{21H}\\
\hllin{41\ }\hlstd{}\hlkwa{POP\ }\hlstd{}\hlkwb{DS}\\
\hllin{42\ }\hlstd{}\\
\hllin{43\ }\hlkwa{IN\ }\hlstd{}\hlkwb{AL}\hlstd{}\hlopt{,}\hlstd{}\hlnum{21H}\\
\hllin{44\ }\hlstd{}\hlkwa{AND\ }\hlstd{}\hlkwb{AL}\hlstd{}\hlopt{,}\hlstd{}\hlnum{0}\hlstd{F7H}\\
\hllin{45\ }\hlkwa{OUT\ }\hlstd{}\hlnum{21H}\hlstd{}\hlopt{,}\hlstd{}\hlkwb{AL}\\
\hllin{46\ }\hlstd{}\\
\hllin{47\ }\hlkwa{MOV\ }\hlstd{}\hlkwb{DX}\hlstd{}\hlopt{,}\hlstd{}\hlnum{0283H}\\
\hllin{48\ }\hlstd{}\hlkwa{MOV\ }\hlstd{}\hlkwb{AL}\hlstd{}\hlopt{,}\hlstd{}\hlnum{037H}\\
\hllin{49\ }\hlstd{}\hlkwa{OUT\ }\hlstd{}\hlkwb{DX}\hlstd{}\hlopt{,}\hlstd{}\hlkwb{AL}\\
\hllin{50\ }\hlstd{}\hlkwa{MOV\ }\hlstd{}\hlkwb{DX}\hlstd{}\hlopt{,}\hlstd{}\hlnum{280H}\\
\hllin{51\ }\hlstd{}\hlkwa{XOR\ }\hlstd{}\hlkwb{AL}\hlstd{}\hlopt{,}\hlstd{}\hlkwb{AL}\\
\hllin{52\ }\hlstd{}\hlkwa{OUT\ }\hlstd{}\hlkwb{DX}\hlstd{}\hlopt{,}\hlstd{}\hlkwb{AL}\\
\hllin{53\ }\hlstd{}\hlkwa{MOV\ }\hlstd{}\hlkwb{AL}\hlstd{}\hlopt{,}\hlstd{}\hlnum{010H}\\
\hllin{54\ }\hlstd{}\hlkwa{OUT\ }\hlstd{}\hlkwb{DX}\hlstd{}\hlopt{,}\hlstd{}\hlkwb{AL}\\
\hllin{55\ }\hlstd{}\hlkwa{MOV\ }\hlstd{}\hlkwb{DX}\hlstd{}\hlopt{,}\hlstd{}\hlnum{0283H}\\
\hllin{56\ }\hlstd{}\hlkwa{MOV\ }\hlstd{}\hlkwb{AL}\hlstd{}\hlopt{,}\hlstd{}\hlnum{077H}\\
\hllin{57\ }\hlstd{}\hlkwa{OUT\ }\hlstd{}\hlkwb{DX}\hlstd{}\hlopt{,}\hlstd{}\hlkwb{AL}\\
\hllin{58\ }\hlstd{}\hlkwa{MOV\ }\hlstd{}\hlkwb{DX}\hlstd{}\hlopt{,}\hlstd{}\hlnum{0281H}\\
\hllin{59\ }\hlstd{}\hlkwa{XOR\ }\hlstd{}\hlkwb{AL}\hlstd{}\hlopt{,}\hlstd{}\hlkwb{AL}\\
\hllin{60\ }\hlstd{}\hlkwa{OUT\ }\hlstd{}\hlkwb{DX}\hlstd{}\hlopt{,}\hlstd{}\hlkwb{AL}\\
\hllin{61\ }\hlstd{}\hlkwa{MOV\ }\hlstd{}\hlkwb{AL}\hlstd{}\hlopt{,}\hlstd{}\hlnum{10H}\\
\hllin{62\ }\hlstd{}\hlkwa{OUT\ }\hlstd{}\hlkwb{DX}\hlstd{}\hlopt{,}\hlstd{}\hlkwb{AL}\\
\hllin{63\ }\hlstd{}\\
\hllin{64\ }\hlkwa{STI}\\
\hllin{65\ }\hlstd{}\hlkwc{L1:}\\
\hllin{66\ }\hlstd{}\hlkwa{CMP\ }\hlstd{NUM}\hlopt{,}\hlstd{}\hlnum{10}\\
\hllin{67\ }\hlstd{}\hlkwa{JNZ\ }\hlstd{L1}\\
\hllin{68\ }\\
\hllin{69\ }\hlkwa{MOV\ }\hlstd{}\hlkwb{DX}\hlstd{}\hlopt{,}\hlstd{}\hlkwb{OFFSET\ }\hlstd{MESS2}\\
\hllin{70\ }\hlkwa{MOV\ }\hlstd{}\hlkwb{AH}\hlstd{}\hlopt{,}\hlstd{}\hlnum{09H}\\
\hllin{71\ }\hlstd{}\hlkwa{INT\ }\hlstd{}\hlnum{21H}\\
\hllin{72\ }\hlstd{}\\
\hllin{73\ }\\
\hllin{74\ }\hlkwa{IN\ }\hlstd{}\hlkwb{AL}\hlstd{}\hlopt{,}\hlstd{}\hlnum{21H}\\
\hllin{75\ }\hlstd{}\hlkwa{OR\ }\hlstd{}\hlkwb{AL}\hlstd{}\hlopt{,}\hlstd{}\hlnum{008H}\\
\hllin{76\ }\hlstd{}\hlkwa{OUT\ }\hlstd{}\hlnum{21H}\hlstd{}\hlopt{,}\hlstd{}\hlkwb{AL}\\
\hllin{77\ }\hlstd{}\\
\hllin{78\ }\hlkwa{PUSH\ }\hlstd{}\hlkwb{DS}\\
\hllin{79\ }\hlstd{}\hlkwa{MOV\ }\hlstd{}\hlkwb{DX}\hlstd{}\hlopt{,}\hlstd{KEEP\textunderscore IP}\\
\hllin{80\ }\hlkwa{MOV\ }\hlstd{}\hlkwb{AX}\hlstd{}\hlopt{,}\hlstd{KEEP\textunderscore CS}\\
\hllin{81\ }\hlkwa{MOV\ }\hlstd{}\hlkwb{DS}\hlstd{}\hlopt{,}\hlstd{}\hlkwb{AX}\\
\hllin{82\ }\hlstd{}\hlkwa{MOV\ }\hlstd{}\hlkwb{AX}\hlstd{}\hlopt{,}\hlstd{}\hlnum{0250}\hlstd{}\hlkwb{BH}\\
\hllin{83\ }\hlstd{}\hlkwa{INT\ }\hlstd{}\hlnum{21H}\\
\hllin{84\ }\hlstd{}\hlkwa{POP\ }\hlstd{}\hlkwb{DS}\\
\hllin{85\ }\hlstd{}\\
\hllin{86\ }\hlkwa{MOV\ }\hlstd{}\hlkwb{AX}\hlstd{}\hlopt{,}\hlstd{}\hlnum{4}\hlstd{C00H}\\
\hllin{87\ }\hlkwa{INT\ }\hlstd{}\hlnum{21H}\\
\hllin{88\ }\hlstd{\\
\hllin{89\ }CODE\ }\hlkwa{ENDS}\\
\hllin{90\ }\hlstd{}\hlkwa{END\ }\hlstd{START}\\
\mbox{}
\normalfont
\normalsize

\section{思考题}
为什么要将8253 的两个计数器串起来(这种连接方式又称级连) 使用,只用一个计数器可以实现
106 分频吗?

单个计数器只有16 位,在十进制工作模式下,最多计数到9999(压缩BCD码),不可能用一个计数器实现$10^6$ 分频。
\section{完成情况及心得体会}
这次实验的主要目的是了解掌握中断控制器8259 和计数器8253 的使用方式。这次实验用到了实验箱,
是第一次软硬结合的实验,除了编程外还有硬件上的接线。编程上的难点主要在于弄清8259 及8253 的使
用方法,要对功能号、类型号以及各控制器、计数器的控制端口号熟悉才能完成。接线虽然不复杂,但也需
要仔细检查才能达到预期的结果。
\end{document}