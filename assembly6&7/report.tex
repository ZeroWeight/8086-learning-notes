\documentclass[UTF8,a4paper]{paper}
\usepackage{ctex}
\usepackage[utf8]{inputenc}
\usepackage{amsmath}
\usepackage{pdfpages}
\usepackage{graphicx}
\usepackage{wrapfig}
\usepackage{listings}
\usepackage{color}
\usepackage{alltt}
\usepackage{marvosym}
\usepackage{xcolor}
% Style definition file generated by highlight 3.6, http://www.andre-simon.de/ 

% Highlighting theme: Vim Editor 

\newcommand{\hlstd}[1]{\textcolor[rgb]{0,0,0}{#1}}
\newcommand{\hlnum}[1]{\textcolor[rgb]{1,0,0}{#1}}
\newcommand{\hlesc}[1]{\textcolor[rgb]{1,0.13,1}{#1}}
\newcommand{\hlstr}[1]{\textcolor[rgb]{1,0,0}{#1}}
\newcommand{\hlpps}[1]{\textcolor[rgb]{1,0,0}{#1}}
\newcommand{\hlslc}[1]{\textcolor[rgb]{0,0,1}{#1}}
\newcommand{\hlcom}[1]{\textcolor[rgb]{0,0,1}{#1}}
\newcommand{\hlppc}[1]{\textcolor[rgb]{1,0.13,1}{#1}}
\newcommand{\hlopt}[1]{\textcolor[rgb]{0,0,0}{#1}}
\newcommand{\hllin}[1]{\textcolor[rgb]{0,0,1}{#1}}
\newcommand{\hlkwa}[1]{\textcolor[rgb]{0.7,0.41,0.09}{#1}}
\newcommand{\hlkwb}[1]{\textcolor[rgb]{0,1,0}{#1}}
\newcommand{\hlkwc}[1]{\textcolor[rgb]{0.7,0.41,0.09}{#1}}
\newcommand{\hlkwd}[1]{\textcolor[rgb]{0,0,0}{#1}}
\definecolor{bgcolor}{rgb}{1,1,1}


\title{计算机原理夏季学期实验报告}
\author{张蔚桐\ 2015011493\ 自55}
\begin {document}
\maketitle
\section{第六次实验}
\subsection{实验目的}
\begin{enumerate}
    \item 学习D/A及A/D转换的基本原理
    \item 掌握转换器DAC0832及ADC0809的使用方法
\end{enumerate}
\subsection{任务一}
用DAC0832实现D/A转换,使产生的模拟电压波形分别为锯
齿波,三角波和正弦波并符合题目的要求完成相关功能。汇
编代码如下所示:
\noindent
\ttfamily
\hlstd{\hllin{01\ }\\
\hllin{02\ }\\
\hllin{03\ }DATA\ }\hlkwa{SEGMENT}\\
\hllin{04\ }\hlstd{}\hlstd{\ \ \ \ }\hlstd{SAW\textunderscore WAVE\ }\hlkwa{DB\ }\hlstd{}\hlnum{0}\hlstd{}\hlopt{,\ }\hlstd{}\hlnum{1}\hlstd{}\hlopt{,\ }\hlstd{}\hlnum{2}\hlstd{}\hlopt{,\ }\hlstd{}\hlnum{3}\hlstd{}\hlopt{,\ }\hlstd{}\hlnum{4}\hlstd{}\hlopt{,\ }\hlstd{}\hlnum{5}\hlstd{}\hlopt{,\ }\hlstd{}\hlnum{6}\hlstd{}\hlopt{,\ }\hlstd{}\hlnum{7}\hlstd{}\hlopt{,\ }\hlstd{}\hlnum{8}\hlstd{}\hlopt{,\ }\hlstd{}\hlnum{9}\hlstd{}\hlopt{,\ }\hlstd{}\hlnum{10}\hlstd{}\hlopt{,\ }\hlstd{}\hlnum{11}\hlstd{}\hlopt{,\ }\hlstd{}\hlnum{12}\hlstd{}\hlopt{,\ }\hlstd{}\hlnum{13}\hlstd{}\hlopt{,}\Righttorque\\
\hllin{05\ }\hlstd{}\hlstd{\ \ \ \ }\hlstd{}\hlnum{14}\hlstd{}\hlopt{,\ }\hlstd{}\hlnum{15}\\
\hllin{06\ }\hlstd{}\hlstd{\ \ \ \ \ \ \ \ }\hlstd{}\hlkwa{DB\ }\hlstd{}\hlnum{16}\hlstd{}\hlopt{,\ }\hlstd{}\hlnum{17}\hlstd{}\hlopt{,\ }\hlstd{}\hlnum{18}\hlstd{}\hlopt{,\ }\hlstd{}\hlnum{19}\hlstd{}\hlopt{,\ }\hlstd{}\hlnum{20}\hlstd{}\hlopt{,\ }\hlstd{}\hlnum{21}\hlstd{}\hlopt{,\ }\hlstd{}\hlnum{22}\hlstd{}\hlopt{,\ }\hlstd{}\hlnum{23}\hlstd{}\hlopt{,\ }\hlstd{}\hlnum{24}\hlstd{}\hlopt{,\ }\hlstd{}\hlnum{25}\hlstd{}\hlopt{,\ }\hlstd{}\hlnum{26}\hlstd{}\hlopt{,\ }\hlstd{}\hlnum{27}\hlstd{}\hlopt{,\ }\Righttorque\\
\hllin{07\ }\hlstd{}\hlstd{\ \ \ \ \ \ \ \ }\hlstd{}\hlnum{28}\hlstd{}\hlopt{,\ }\hlstd{}\hlnum{29}\hlstd{}\hlopt{,\ }\hlstd{}\hlnum{30}\hlstd{}\hlopt{,\ }\hlstd{}\hlnum{31}\\
\hllin{08\ }\hlstd{}\hlstd{\ \ \ \ \ \ \ \ }\hlstd{}\hlkwa{DB\ }\hlstd{}\hlnum{32}\hlstd{}\hlopt{,\ }\hlstd{}\hlnum{33}\hlstd{}\hlopt{,\ }\hlstd{}\hlnum{34}\hlstd{}\hlopt{,\ }\hlstd{}\hlnum{35}\hlstd{}\hlopt{,\ }\hlstd{}\hlnum{36}\hlstd{}\hlopt{,\ }\hlstd{}\hlnum{37}\hlstd{}\hlopt{,\ }\hlstd{}\hlnum{38}\hlstd{}\hlopt{,\ }\hlstd{}\hlnum{39}\hlstd{}\hlopt{,\ }\hlstd{}\hlnum{40}\hlstd{}\hlopt{,\ }\hlstd{}\hlnum{41}\hlstd{}\hlopt{,\ }\hlstd{}\hlnum{42}\hlstd{}\hlopt{,\ }\hlstd{}\hlnum{43}\hlstd{}\hlopt{,\ }\Righttorque\\
\hllin{09\ }\hlstd{}\hlstd{\ \ \ \ \ \ \ \ }\hlstd{}\hlnum{44}\hlstd{}\hlopt{,\ }\hlstd{}\hlnum{45}\hlstd{}\hlopt{,\ }\hlstd{}\hlnum{46}\hlstd{}\hlopt{,\ }\hlstd{}\hlnum{47}\\
\hllin{10\ }\hlstd{}\hlstd{\ \ \ \ \ \ \ \ }\hlstd{}\hlkwa{DB\ }\hlstd{}\hlnum{48}\hlstd{}\hlopt{,\ }\hlstd{}\hlnum{49}\hlstd{}\hlopt{,\ }\hlstd{}\hlnum{50}\hlstd{}\hlopt{,\ }\hlstd{}\hlnum{51}\hlstd{}\hlopt{,\ }\hlstd{}\hlnum{52}\hlstd{}\hlopt{,\ }\hlstd{}\hlnum{53}\hlstd{}\hlopt{,\ }\hlstd{}\hlnum{54}\hlstd{}\hlopt{,\ }\hlstd{}\hlnum{55}\hlstd{}\hlopt{,\ }\hlstd{}\hlnum{56}\hlstd{}\hlopt{,\ }\hlstd{}\hlnum{57}\hlstd{}\hlopt{,\ }\hlstd{}\hlnum{58}\hlstd{}\hlopt{,\ }\hlstd{}\hlnum{59}\hlstd{}\hlopt{,\ }\Righttorque\\
\hllin{11\ }\hlstd{}\hlstd{\ \ \ \ \ \ \ \ }\hlstd{}\hlnum{60}\hlstd{}\hlopt{,\ }\hlstd{}\hlnum{61}\hlstd{}\hlopt{,\ }\hlstd{}\hlnum{62}\hlstd{}\hlopt{,\ }\hlstd{}\hlnum{63}\\
\hllin{12\ }\hlstd{}\hlstd{\ \ \ \ \ \ \ \ }\hlstd{}\hlkwa{DB\ }\hlstd{}\hlnum{64}\hlstd{}\hlopt{,\ }\hlstd{}\hlnum{65}\hlstd{}\hlopt{,\ }\hlstd{}\hlnum{66}\hlstd{}\hlopt{,\ }\hlstd{}\hlnum{67}\hlstd{}\hlopt{,\ }\hlstd{}\hlnum{68}\hlstd{}\hlopt{,\ }\hlstd{}\hlnum{69}\hlstd{}\hlopt{,\ }\hlstd{}\hlnum{70}\hlstd{}\hlopt{,\ }\hlstd{}\hlnum{71}\hlstd{}\hlopt{,\ }\hlstd{}\hlnum{72}\hlstd{}\hlopt{,\ }\hlstd{}\hlnum{73}\hlstd{}\hlopt{,\ }\hlstd{}\hlnum{74}\hlstd{}\hlopt{,\ }\hlstd{}\hlnum{75}\hlstd{}\hlopt{,\ }\Righttorque\\
\hllin{13\ }\hlstd{}\hlstd{\ \ \ \ \ \ \ \ }\hlstd{}\hlnum{76}\hlstd{}\hlopt{,\ }\hlstd{}\hlnum{77}\hlstd{}\hlopt{,\ }\hlstd{}\hlnum{78}\hlstd{}\hlopt{,\ }\hlstd{}\hlnum{79}\\
\hllin{14\ }\hlstd{}\hlstd{\ \ \ \ \ \ \ \ }\hlstd{}\hlkwa{DB\ }\hlstd{}\hlnum{80}\hlstd{}\hlopt{,\ }\hlstd{}\hlnum{81}\hlstd{}\hlopt{,\ }\hlstd{}\hlnum{82}\hlstd{}\hlopt{,\ }\hlstd{}\hlnum{83}\hlstd{}\hlopt{,\ }\hlstd{}\hlnum{84}\hlstd{}\hlopt{,\ }\hlstd{}\hlnum{85}\hlstd{}\hlopt{,\ }\hlstd{}\hlnum{86}\hlstd{}\hlopt{,\ }\hlstd{}\hlnum{87}\hlstd{}\hlopt{,\ }\hlstd{}\hlnum{88}\hlstd{}\hlopt{,\ }\hlstd{}\hlnum{89}\hlstd{}\hlopt{,\ }\hlstd{}\hlnum{90}\hlstd{}\hlopt{,\ }\hlstd{}\hlnum{91}\hlstd{}\hlopt{,\ }\Righttorque\\
\hllin{15\ }\hlstd{}\hlstd{\ \ \ \ \ \ \ \ }\hlstd{}\hlnum{92}\hlstd{}\hlopt{,\ }\hlstd{}\hlnum{93}\hlstd{}\hlopt{,\ }\hlstd{}\hlnum{94}\hlstd{}\hlopt{,\ }\hlstd{}\hlnum{95}\\
\hllin{16\ }\hlstd{}\hlstd{\ \ \ \ \ \ \ \ }\hlstd{}\hlkwa{DB\ }\hlstd{}\hlnum{96}\hlstd{}\hlopt{,\ }\hlstd{}\hlnum{97}\hlstd{}\hlopt{,\ }\hlstd{}\hlnum{98}\hlstd{}\hlopt{,\ }\hlstd{}\hlnum{99}\hlstd{}\hlopt{,\ }\hlstd{}\hlnum{100}\hlstd{}\hlopt{,\ }\hlstd{}\hlnum{101}\hlstd{}\hlopt{,\ }\hlstd{}\hlnum{102}\hlstd{}\hlopt{,\ }\hlstd{}\hlnum{103}\hlstd{}\hlopt{,\ }\hlstd{}\hlnum{104}\hlstd{}\hlopt{,\ }\hlstd{}\hlnum{105}\hlstd{}\hlopt{,\ }\hlstd{}\hlnum{106}\hlstd{}\hlopt{,}\Righttorque\\
\hllin{17\ }\hlstd{}\hlstd{\ \ \ \ \ \ \ \ }\hlstd{}\hlnum{107}\hlstd{}\hlopt{,\ }\hlstd{}\hlnum{108}\hlstd{}\hlopt{,\ }\hlstd{}\hlnum{109}\hlstd{}\hlopt{,\ }\hlstd{}\hlnum{110}\hlstd{}\hlopt{,\ }\hlstd{}\hlnum{111}\\
\hllin{18\ }\hlstd{}\hlstd{\ \ \ \ \ \ \ \ }\hlstd{}\hlkwa{DB\ }\hlstd{}\hlnum{112}\hlstd{}\hlopt{,\ }\hlstd{}\hlnum{113}\hlstd{}\hlopt{,\ }\hlstd{}\hlnum{114}\hlstd{}\hlopt{,\ }\hlstd{}\hlnum{115}\hlstd{}\hlopt{,\ }\hlstd{}\hlnum{116}\hlstd{}\hlopt{,\ }\hlstd{}\hlnum{117}\hlstd{}\hlopt{,\ }\hlstd{}\hlnum{118}\hlstd{}\hlopt{,\ }\hlstd{}\hlnum{119}\hlstd{}\hlopt{,\ }\hlstd{}\hlnum{120}\hlstd{}\hlopt{,\ }\hlstd{}\hlnum{121}\hlstd{}\hlopt{,\ }\Righttorque\\
\hllin{19\ }\hlstd{}\hlstd{\ \ \ \ \ \ \ \ }\hlstd{}\hlnum{122}\hlstd{}\hlopt{,\ }\hlstd{}\hlnum{123}\hlstd{}\hlopt{,\ }\hlstd{}\hlnum{124}\hlstd{}\hlopt{,\ }\hlstd{}\hlnum{125}\hlstd{}\hlopt{,\ }\hlstd{}\hlnum{126}\hlstd{}\hlopt{,\ }\hlstd{}\hlnum{127}\\
\hllin{20\ }\hlstd{}\hlstd{\ \ \ \ \ \ \ \ }\hlstd{}\hlkwa{DB\ }\hlstd{}\hlnum{128}\hlstd{}\hlopt{,\ }\hlstd{}\hlnum{129}\hlstd{}\hlopt{,\ }\hlstd{}\hlnum{130}\hlstd{}\hlopt{,\ }\hlstd{}\hlnum{131}\hlstd{}\hlopt{,\ }\hlstd{}\hlnum{132}\hlstd{}\hlopt{,\ }\hlstd{}\hlnum{133}\hlstd{}\hlopt{,\ }\hlstd{}\hlnum{134}\hlstd{}\hlopt{,\ }\hlstd{}\hlnum{135}\hlstd{}\hlopt{,\ }\hlstd{}\hlnum{136}\hlstd{}\hlopt{,\ }\hlstd{}\hlnum{137}\hlstd{}\hlopt{,\ }\Righttorque\\
\hllin{21\ }\hlstd{}\hlstd{\ \ \ \ \ \ \ \ }\hlstd{}\hlnum{138}\hlstd{}\hlopt{,\ }\hlstd{}\hlnum{139}\hlstd{}\hlopt{,\ }\hlstd{}\hlnum{140}\hlstd{}\hlopt{,\ }\hlstd{}\hlnum{141}\hlstd{}\hlopt{,\ }\hlstd{}\hlnum{142}\hlstd{}\hlopt{,\ }\hlstd{}\hlnum{143}\\
\hllin{22\ }\hlstd{}\hlstd{\ \ \ \ \ \ \ \ }\hlstd{}\hlkwa{DB\ }\hlstd{}\hlnum{144}\hlstd{}\hlopt{,\ }\hlstd{}\hlnum{145}\hlstd{}\hlopt{,\ }\hlstd{}\hlnum{146}\hlstd{}\hlopt{,\ }\hlstd{}\hlnum{147}\hlstd{}\hlopt{,\ }\hlstd{}\hlnum{148}\hlstd{}\hlopt{,\ }\hlstd{}\hlnum{149}\hlstd{}\hlopt{,\ }\hlstd{}\hlnum{150}\hlstd{}\hlopt{,\ }\hlstd{}\hlnum{151}\hlstd{}\hlopt{,\ }\hlstd{}\hlnum{152}\hlstd{}\hlopt{,\ }\hlstd{}\hlnum{153}\hlstd{}\hlopt{,\ }\Righttorque\\
\hllin{23\ }\hlstd{}\hlstd{\ \ \ \ \ \ \ \ }\hlstd{}\hlnum{154}\hlstd{}\hlopt{,\ }\hlstd{}\hlnum{155}\hlstd{}\hlopt{,\ }\hlstd{}\hlnum{156}\hlstd{}\hlopt{,\ }\hlstd{}\hlnum{157}\hlstd{}\hlopt{,\ }\hlstd{}\hlnum{158}\hlstd{}\hlopt{,\ }\hlstd{}\hlnum{159}\\
\hllin{24\ }\hlstd{}\hlstd{\ \ \ \ \ \ \ \ }\hlstd{}\hlkwa{DB\ }\hlstd{}\hlnum{160}\hlstd{}\hlopt{,\ }\hlstd{}\hlnum{161}\hlstd{}\hlopt{,\ }\hlstd{}\hlnum{162}\hlstd{}\hlopt{,\ }\hlstd{}\hlnum{163}\hlstd{}\hlopt{,\ }\hlstd{}\hlnum{164}\hlstd{}\hlopt{,\ }\hlstd{}\hlnum{165}\hlstd{}\hlopt{,\ }\hlstd{}\hlnum{166}\hlstd{}\hlopt{,\ }\hlstd{}\hlnum{167}\hlstd{}\hlopt{,\ }\hlstd{}\hlnum{168}\hlstd{}\hlopt{,\ }\hlstd{}\hlnum{169}\hlstd{}\hlopt{,\ }\Righttorque\\
\hllin{25\ }\hlstd{}\hlstd{\ \ \ \ \ \ \ \ }\hlstd{}\hlnum{170}\hlstd{}\hlopt{,\ }\hlstd{}\hlnum{171}\hlstd{}\hlopt{,\ }\hlstd{}\hlnum{172}\hlstd{}\hlopt{,\ }\hlstd{}\hlnum{173}\hlstd{}\hlopt{,\ }\hlstd{}\hlnum{174}\hlstd{}\hlopt{,\ }\hlstd{}\hlnum{175}\\
\hllin{26\ }\hlstd{}\hlstd{\ \ \ \ \ \ \ \ }\hlstd{}\hlkwa{DB\ }\hlstd{}\hlnum{176}\hlstd{}\hlopt{,\ }\hlstd{}\hlnum{177}\hlstd{}\hlopt{,\ }\hlstd{}\hlnum{178}\hlstd{}\hlopt{,\ }\hlstd{}\hlnum{179}\hlstd{}\hlopt{,\ }\hlstd{}\hlnum{180}\hlstd{}\hlopt{,\ }\hlstd{}\hlnum{181}\hlstd{}\hlopt{,\ }\hlstd{}\hlnum{182}\hlstd{}\hlopt{,\ }\hlstd{}\hlnum{183}\hlstd{}\hlopt{,\ }\hlstd{}\hlnum{184}\hlstd{}\hlopt{,\ }\hlstd{}\hlnum{185}\hlstd{}\hlopt{,\ }\Righttorque\\
\hllin{27\ }\hlstd{}\hlstd{\ \ \ \ \ \ \ \ }\hlstd{}\hlnum{186}\hlstd{}\hlopt{,\ }\hlstd{}\hlnum{187}\hlstd{}\hlopt{,\ }\hlstd{}\hlnum{188}\hlstd{}\hlopt{,\ }\hlstd{}\hlnum{189}\hlstd{}\hlopt{,\ }\hlstd{}\hlnum{190}\hlstd{}\hlopt{,\ }\hlstd{}\hlnum{191}\\
\hllin{28\ }\hlstd{}\hlstd{\ \ \ \ \ \ \ \ }\hlstd{}\hlkwa{DB\ }\hlstd{}\hlnum{192}\hlstd{}\hlopt{,\ }\hlstd{}\hlnum{193}\hlstd{}\hlopt{,\ }\hlstd{}\hlnum{194}\hlstd{}\hlopt{,\ }\hlstd{}\hlnum{195}\hlstd{}\hlopt{,\ }\hlstd{}\hlnum{196}\hlstd{}\hlopt{,\ }\hlstd{}\hlnum{197}\hlstd{}\hlopt{,\ }\hlstd{}\hlnum{198}\hlstd{}\hlopt{,\ }\hlstd{}\hlnum{199}\hlstd{}\hlopt{,\ }\hlstd{}\hlnum{200}\hlstd{}\hlopt{,\ }\hlstd{}\hlnum{201}\hlstd{}\hlopt{,\ }\Righttorque\\
\hllin{29\ }\hlstd{}\hlstd{\ \ \ \ \ \ \ \ }\hlstd{}\hlnum{202}\hlstd{}\hlopt{,\ }\hlstd{}\hlnum{203}\hlstd{}\hlopt{,\ }\hlstd{}\hlnum{204}\hlstd{}\hlopt{,\ }\hlstd{}\hlnum{205}\hlstd{}\hlopt{,\ }\hlstd{}\hlnum{206}\hlstd{}\hlopt{,\ }\hlstd{}\hlnum{207}\\
\hllin{30\ }\hlstd{}\hlstd{\ \ \ \ \ \ \ \ }\hlstd{}\hlkwa{DB\ }\hlstd{}\hlnum{208}\hlstd{}\hlopt{,\ }\hlstd{}\hlnum{209}\hlstd{}\hlopt{,\ }\hlstd{}\hlnum{210}\hlstd{}\hlopt{,\ }\hlstd{}\hlnum{211}\hlstd{}\hlopt{,\ }\hlstd{}\hlnum{212}\hlstd{}\hlopt{,\ }\hlstd{}\hlnum{213}\hlstd{}\hlopt{,\ }\hlstd{}\hlnum{214}\hlstd{}\hlopt{,\ }\hlstd{}\hlnum{215}\hlstd{}\hlopt{,\ }\hlstd{}\hlnum{216}\hlstd{}\hlopt{,\ }\hlstd{}\hlnum{217}\hlstd{}\hlopt{,\ }\Righttorque\\
\hllin{31\ }\hlstd{}\hlstd{\ \ \ \ \ \ \ \ }\hlstd{}\hlnum{218}\hlstd{}\hlopt{,\ }\hlstd{}\hlnum{219}\hlstd{}\hlopt{,\ }\hlstd{}\hlnum{220}\hlstd{}\hlopt{,\ }\hlstd{}\hlnum{221}\hlstd{}\hlopt{,\ }\hlstd{}\hlnum{222}\hlstd{}\hlopt{,\ }\hlstd{}\hlnum{223}\\
\hllin{32\ }\hlstd{}\hlstd{\ \ \ \ \ \ \ \ }\hlstd{}\hlkwa{DB\ }\hlstd{}\hlnum{224}\hlstd{}\hlopt{,\ }\hlstd{}\hlnum{225}\hlstd{}\hlopt{,\ }\hlstd{}\hlnum{226}\hlstd{}\hlopt{,\ }\hlstd{}\hlnum{227}\hlstd{}\hlopt{,\ }\hlstd{}\hlnum{228}\hlstd{}\hlopt{,\ }\hlstd{}\hlnum{229}\hlstd{}\hlopt{,\ }\hlstd{}\hlnum{230}\hlstd{}\hlopt{,\ }\hlstd{}\hlnum{231}\hlstd{}\hlopt{,\ }\hlstd{}\hlnum{232}\hlstd{}\hlopt{,\ }\hlstd{}\hlnum{233}\hlstd{}\hlopt{,\ }\Righttorque\\
\hllin{33\ }\hlstd{}\hlstd{\ \ \ \ \ \ \ \ }\hlstd{}\hlnum{234}\hlstd{}\hlopt{,\ }\hlstd{}\hlnum{235}\hlstd{}\hlopt{,\ }\hlstd{}\hlnum{236}\hlstd{}\hlopt{,\ }\hlstd{}\hlnum{237}\hlstd{}\hlopt{,\ }\hlstd{}\hlnum{238}\hlstd{}\hlopt{,\ }\hlstd{}\hlnum{239}\\
\hllin{34\ }\hlstd{}\hlstd{\ \ \ \ \ \ \ \ }\hlstd{}\hlkwa{DB\ }\hlstd{}\hlnum{240}\hlstd{}\hlopt{,\ }\hlstd{}\hlnum{241}\hlstd{}\hlopt{,\ }\hlstd{}\hlnum{242}\hlstd{}\hlopt{,\ }\hlstd{}\hlnum{243}\hlstd{}\hlopt{,\ }\hlstd{}\hlnum{244}\hlstd{}\hlopt{,\ }\hlstd{}\hlnum{245}\hlstd{}\hlopt{,\ }\hlstd{}\hlnum{246}\hlstd{}\hlopt{,\ }\hlstd{}\hlnum{247}\hlstd{}\hlopt{,\ }\hlstd{}\hlnum{248}\hlstd{}\hlopt{,\ }\hlstd{}\hlnum{249}\hlstd{}\hlopt{,\ }\Righttorque\\
\hllin{35\ }\hlstd{}\hlstd{\ \ \ \ \ \ \ \ }\hlstd{}\hlnum{250}\hlstd{}\hlopt{,\ }\hlstd{}\hlnum{251}\hlstd{}\hlopt{,\ }\hlstd{}\hlnum{252}\hlstd{}\hlopt{,\ }\hlstd{}\hlnum{253}\hlstd{}\hlopt{,\ }\hlstd{}\hlnum{254}\hlstd{}\hlopt{,\ }\hlstd{}\hlnum{255}\\
\hllin{36\ }\hlstd{}\hlstd{\ \ \ \ }\hlstd{TRIANGLE\textunderscore WAVE\ }\hlkwa{DB\ }\hlstd{}\hlnum{1}\hlstd{}\hlopt{,\ }\hlstd{}\hlnum{3}\hlstd{}\hlopt{,\ }\hlstd{}\hlnum{5}\hlstd{}\hlopt{,\ }\hlstd{}\hlnum{7}\hlstd{}\hlopt{,\ }\hlstd{}\hlnum{9}\hlstd{}\hlopt{,\ }\hlstd{}\hlnum{11}\hlstd{}\hlopt{,\ }\hlstd{}\hlnum{13}\hlstd{}\hlopt{,\ }\hlstd{}\hlnum{15}\hlstd{}\hlopt{,\ }\hlstd{}\hlnum{17}\hlstd{}\hlopt{,\ }\hlstd{}\hlnum{19}\hlstd{}\hlopt{,\ }\hlstd{}\hlnum{21}\hlstd{}\hlopt{,\ }\Righttorque\\
\hllin{37\ }\hlstd{}\hlstd{\ \ \ \ }\hlstd{}\hlnum{23}\hlstd{}\hlopt{,\ }\hlstd{}\hlnum{25}\hlstd{}\hlopt{,\ }\hlstd{}\hlnum{27}\hlstd{}\hlopt{,\ }\hlstd{}\hlnum{29}\hlstd{}\hlopt{,\ }\hlstd{}\hlnum{31}\\
\hllin{38\ }\hlstd{}\hlstd{\ \ \ \ \ \ \ \ }\hlstd{}\hlkwa{DB\ }\hlstd{}\hlnum{33}\hlstd{}\hlopt{,\ }\hlstd{}\hlnum{35}\hlstd{}\hlopt{,\ }\hlstd{}\hlnum{37}\hlstd{}\hlopt{,\ }\hlstd{}\hlnum{39}\hlstd{}\hlopt{,\ }\hlstd{}\hlnum{41}\hlstd{}\hlopt{,\ }\hlstd{}\hlnum{43}\hlstd{}\hlopt{,\ }\hlstd{}\hlnum{45}\hlstd{}\hlopt{,\ }\hlstd{}\hlnum{47}\hlstd{}\hlopt{,\ }\hlstd{}\hlnum{49}\hlstd{}\hlopt{,\ }\hlstd{}\hlnum{51}\hlstd{}\hlopt{,\ }\hlstd{}\hlnum{53}\hlstd{}\hlopt{,\ }\hlstd{}\hlnum{55}\hlstd{}\hlopt{,\ }\Righttorque\\
\hllin{39\ }\hlstd{}\hlstd{\ \ \ \ \ \ \ \ }\hlstd{}\hlnum{57}\hlstd{}\hlopt{,\ }\hlstd{}\hlnum{59}\hlstd{}\hlopt{,\ }\hlstd{}\hlnum{61}\hlstd{}\hlopt{,\ }\hlstd{}\hlnum{63}\\
\hllin{40\ }\hlstd{}\hlstd{\ \ \ \ \ \ \ \ }\hlstd{}\hlkwa{DB\ }\hlstd{}\hlnum{65}\hlstd{}\hlopt{,\ }\hlstd{}\hlnum{67}\hlstd{}\hlopt{,\ }\hlstd{}\hlnum{69}\hlstd{}\hlopt{,\ }\hlstd{}\hlnum{71}\hlstd{}\hlopt{,\ }\hlstd{}\hlnum{73}\hlstd{}\hlopt{,\ }\hlstd{}\hlnum{75}\hlstd{}\hlopt{,\ }\hlstd{}\hlnum{77}\hlstd{}\hlopt{,\ }\hlstd{}\hlnum{79}\hlstd{}\hlopt{,\ }\hlstd{}\hlnum{81}\hlstd{}\hlopt{,\ }\hlstd{}\hlnum{83}\hlstd{}\hlopt{,\ }\hlstd{}\hlnum{85}\hlstd{}\hlopt{,\ }\hlstd{}\hlnum{87}\hlstd{}\hlopt{,\ }\Righttorque\\
\hllin{41\ }\hlstd{}\hlstd{\ \ \ \ \ \ \ \ }\hlstd{}\hlnum{89}\hlstd{}\hlopt{,\ }\hlstd{}\hlnum{91}\hlstd{}\hlopt{,\ }\hlstd{}\hlnum{93}\hlstd{}\hlopt{,\ }\hlstd{}\hlnum{95}\\
\hllin{42\ }\hlstd{}\hlstd{\ \ \ \ \ \ \ \ }\hlstd{}\hlkwa{DB\ }\hlstd{}\hlnum{97}\hlstd{}\hlopt{,\ }\hlstd{}\hlnum{99}\hlstd{}\hlopt{,\ }\hlstd{}\hlnum{101}\hlstd{}\hlopt{,\ }\hlstd{}\hlnum{103}\hlstd{}\hlopt{,\ }\hlstd{}\hlnum{105}\hlstd{}\hlopt{,\ }\hlstd{}\hlnum{107}\hlstd{}\hlopt{,\ }\hlstd{}\hlnum{109}\hlstd{}\hlopt{,\ }\hlstd{}\hlnum{111}\hlstd{}\hlopt{,\ }\hlstd{}\hlnum{113}\hlstd{}\hlopt{,\ }\hlstd{}\hlnum{115}\hlstd{}\hlopt{,\ }\Righttorque\\
\hllin{43\ }\hlstd{}\hlstd{\ \ \ \ \ \ \ \ }\hlstd{}\hlnum{117}\hlstd{}\hlopt{,\ }\hlstd{}\hlnum{119}\hlstd{}\hlopt{,\ }\hlstd{}\hlnum{121}\hlstd{}\hlopt{,\ }\hlstd{}\hlnum{123}\hlstd{}\hlopt{,\ }\hlstd{}\hlnum{125}\hlstd{}\hlopt{,\ }\hlstd{}\hlnum{127}\\
\hllin{44\ }\hlstd{}\hlstd{\ \ \ \ \ \ \ \ }\hlstd{}\hlkwa{DB\ }\hlstd{}\hlnum{129}\hlstd{}\hlopt{,\ }\hlstd{}\hlnum{131}\hlstd{}\hlopt{,\ }\hlstd{}\hlnum{133}\hlstd{}\hlopt{,\ }\hlstd{}\hlnum{135}\hlstd{}\hlopt{,\ }\hlstd{}\hlnum{137}\hlstd{}\hlopt{,\ }\hlstd{}\hlnum{139}\hlstd{}\hlopt{,\ }\hlstd{}\hlnum{141}\hlstd{}\hlopt{,\ }\hlstd{}\hlnum{143}\hlstd{}\hlopt{,\ }\hlstd{}\hlnum{145}\hlstd{}\hlopt{,\ }\hlstd{}\hlnum{147}\hlstd{}\hlopt{,\ }\Righttorque\\
\hllin{45\ }\hlstd{}\hlstd{\ \ \ \ \ \ \ \ }\hlstd{}\hlnum{149}\hlstd{}\hlopt{,\ }\hlstd{}\hlnum{151}\hlstd{}\hlopt{,\ }\hlstd{}\hlnum{153}\hlstd{}\hlopt{,\ }\hlstd{}\hlnum{155}\hlstd{}\hlopt{,\ }\hlstd{}\hlnum{157}\hlstd{}\hlopt{,\ }\hlstd{}\hlnum{159}\\
\hllin{46\ }\hlstd{}\hlstd{\ \ \ \ \ \ \ \ }\hlstd{}\hlkwa{DB\ }\hlstd{}\hlnum{161}\hlstd{}\hlopt{,\ }\hlstd{}\hlnum{163}\hlstd{}\hlopt{,\ }\hlstd{}\hlnum{165}\hlstd{}\hlopt{,\ }\hlstd{}\hlnum{167}\hlstd{}\hlopt{,\ }\hlstd{}\hlnum{169}\hlstd{}\hlopt{,\ }\hlstd{}\hlnum{171}\hlstd{}\hlopt{,\ }\hlstd{}\hlnum{173}\hlstd{}\hlopt{,\ }\hlstd{}\hlnum{175}\hlstd{}\hlopt{,\ }\hlstd{}\hlnum{177}\hlstd{}\hlopt{,\ }\hlstd{}\hlnum{179}\hlstd{}\hlopt{,\ }\Righttorque\\
\hllin{47\ }\hlstd{}\hlstd{\ \ \ \ \ \ \ \ }\hlstd{}\hlnum{181}\hlstd{}\hlopt{,\ }\hlstd{}\hlnum{183}\hlstd{}\hlopt{,\ }\hlstd{}\hlnum{185}\hlstd{}\hlopt{,\ }\hlstd{}\hlnum{187}\hlstd{}\hlopt{,\ }\hlstd{}\hlnum{189}\hlstd{}\hlopt{,\ }\hlstd{}\hlnum{191}\\
\hllin{48\ }\hlstd{}\hlstd{\ \ \ \ \ \ \ \ }\hlstd{}\hlkwa{DB\ }\hlstd{}\hlnum{193}\hlstd{}\hlopt{,\ }\hlstd{}\hlnum{195}\hlstd{}\hlopt{,\ }\hlstd{}\hlnum{197}\hlstd{}\hlopt{,\ }\hlstd{}\hlnum{199}\hlstd{}\hlopt{,\ }\hlstd{}\hlnum{201}\hlstd{}\hlopt{,\ }\hlstd{}\hlnum{203}\hlstd{}\hlopt{,\ }\hlstd{}\hlnum{205}\hlstd{}\hlopt{,\ }\hlstd{}\hlnum{207}\hlstd{}\hlopt{,\ }\hlstd{}\hlnum{209}\hlstd{}\hlopt{,\ }\hlstd{}\hlnum{211}\hlstd{}\hlopt{,\ }\Righttorque\\
\hllin{49\ }\hlstd{}\hlstd{\ \ \ \ \ \ \ \ }\hlstd{}\hlnum{213}\hlstd{}\hlopt{,\ }\hlstd{}\hlnum{215}\hlstd{}\hlopt{,\ }\hlstd{}\hlnum{217}\hlstd{}\hlopt{,\ }\hlstd{}\hlnum{219}\hlstd{}\hlopt{,\ }\hlstd{}\hlnum{221}\hlstd{}\hlopt{,\ }\hlstd{}\hlnum{223}\\
\hllin{50\ }\hlstd{}\hlstd{\ \ \ \ \ \ \ \ }\hlstd{}\hlkwa{DB\ }\hlstd{}\hlnum{225}\hlstd{}\hlopt{,\ }\hlstd{}\hlnum{227}\hlstd{}\hlopt{,\ }\hlstd{}\hlnum{229}\hlstd{}\hlopt{,\ }\hlstd{}\hlnum{231}\hlstd{}\hlopt{,\ }\hlstd{}\hlnum{233}\hlstd{}\hlopt{,\ }\hlstd{}\hlnum{235}\hlstd{}\hlopt{,\ }\hlstd{}\hlnum{237}\hlstd{}\hlopt{,\ }\hlstd{}\hlnum{239}\hlstd{}\hlopt{,\ }\hlstd{}\hlnum{241}\hlstd{}\hlopt{,\ }\hlstd{}\hlnum{243}\hlstd{}\hlopt{,\ }\Righttorque\\
\hllin{51\ }\hlstd{}\hlstd{\ \ \ \ \ \ \ \ }\hlstd{}\hlnum{245}\hlstd{}\hlopt{,\ }\hlstd{}\hlnum{247}\hlstd{}\hlopt{,\ }\hlstd{}\hlnum{249}\hlstd{}\hlopt{,\ }\hlstd{}\hlnum{251}\hlstd{}\hlopt{,\ }\hlstd{}\hlnum{253}\hlstd{}\hlopt{,\ }\hlstd{}\hlnum{255}\\
\hllin{52\ }\hlstd{}\hlstd{\ \ \ \ \ \ \ \ }\hlstd{}\hlkwa{DB\ }\hlstd{}\hlnum{255}\hlstd{}\hlopt{,\ }\hlstd{}\hlnum{253}\hlstd{}\hlopt{,\ }\hlstd{}\hlnum{251}\hlstd{}\hlopt{,\ }\hlstd{}\hlnum{249}\hlstd{}\hlopt{,\ }\hlstd{}\hlnum{247}\hlstd{}\hlopt{,\ }\hlstd{}\hlnum{245}\hlstd{}\hlopt{,\ }\hlstd{}\hlnum{243}\hlstd{}\hlopt{,\ }\hlstd{}\hlnum{241}\hlstd{}\hlopt{,\ }\hlstd{}\hlnum{239}\hlstd{}\hlopt{,\ }\hlstd{}\hlnum{237}\hlstd{}\hlopt{,\ }\Righttorque\\
\hllin{53\ }\hlstd{}\hlstd{\ \ \ \ \ \ \ \ }\hlstd{}\hlnum{235}\hlstd{}\hlopt{,\ }\hlstd{}\hlnum{233}\hlstd{}\hlopt{,\ }\hlstd{}\hlnum{231}\hlstd{}\hlopt{,\ }\hlstd{}\hlnum{229}\hlstd{}\hlopt{,\ }\hlstd{}\hlnum{227}\hlstd{}\hlopt{,\ }\hlstd{}\hlnum{225}\\
\hllin{54\ }\hlstd{}\hlstd{\ \ \ \ \ \ \ \ }\hlstd{}\hlkwa{DB\ }\hlstd{}\hlnum{223}\hlstd{}\hlopt{,\ }\hlstd{}\hlnum{221}\hlstd{}\hlopt{,\ }\hlstd{}\hlnum{219}\hlstd{}\hlopt{,\ }\hlstd{}\hlnum{217}\hlstd{}\hlopt{,\ }\hlstd{}\hlnum{215}\hlstd{}\hlopt{,\ }\hlstd{}\hlnum{213}\hlstd{}\hlopt{,\ }\hlstd{}\hlnum{211}\hlstd{}\hlopt{,\ }\hlstd{}\hlnum{209}\hlstd{}\hlopt{,\ }\hlstd{}\hlnum{207}\hlstd{}\hlopt{,\ }\hlstd{}\hlnum{205}\hlstd{}\hlopt{,\ }\Righttorque\\
\hllin{55\ }\hlstd{}\hlstd{\ \ \ \ \ \ \ \ }\hlstd{}\hlnum{203}\hlstd{}\hlopt{,\ }\hlstd{}\hlnum{201}\hlstd{}\hlopt{,\ }\hlstd{}\hlnum{199}\hlstd{}\hlopt{,\ }\hlstd{}\hlnum{197}\hlstd{}\hlopt{,\ }\hlstd{}\hlnum{195}\hlstd{}\hlopt{,\ }\hlstd{}\hlnum{193}\\
\hllin{56\ }\hlstd{}\hlstd{\ \ \ \ \ \ \ \ }\hlstd{}\hlkwa{DB\ }\hlstd{}\hlnum{191}\hlstd{}\hlopt{,\ }\hlstd{}\hlnum{189}\hlstd{}\hlopt{,\ }\hlstd{}\hlnum{187}\hlstd{}\hlopt{,\ }\hlstd{}\hlnum{185}\hlstd{}\hlopt{,\ }\hlstd{}\hlnum{183}\hlstd{}\hlopt{,\ }\hlstd{}\hlnum{181}\hlstd{}\hlopt{,\ }\hlstd{}\hlnum{179}\hlstd{}\hlopt{,\ }\hlstd{}\hlnum{177}\hlstd{}\hlopt{,\ }\hlstd{}\hlnum{175}\hlstd{}\hlopt{,\ }\hlstd{}\hlnum{173}\hlstd{}\hlopt{,\ }\Righttorque\\
\hllin{57\ }\hlstd{}\hlstd{\ \ \ \ \ \ \ \ }\hlstd{}\hlnum{171}\hlstd{}\hlopt{,\ }\hlstd{}\hlnum{169}\hlstd{}\hlopt{,\ }\hlstd{}\hlnum{167}\hlstd{}\hlopt{,\ }\hlstd{}\hlnum{165}\hlstd{}\hlopt{,\ }\hlstd{}\hlnum{163}\hlstd{}\hlopt{,\ }\hlstd{}\hlnum{161}\\
\hllin{58\ }\hlstd{}\hlstd{\ \ \ \ \ \ \ \ }\hlstd{}\hlkwa{DB\ }\hlstd{}\hlnum{159}\hlstd{}\hlopt{,\ }\hlstd{}\hlnum{157}\hlstd{}\hlopt{,\ }\hlstd{}\hlnum{155}\hlstd{}\hlopt{,\ }\hlstd{}\hlnum{153}\hlstd{}\hlopt{,\ }\hlstd{}\hlnum{151}\hlstd{}\hlopt{,\ }\hlstd{}\hlnum{149}\hlstd{}\hlopt{,\ }\hlstd{}\hlnum{147}\hlstd{}\hlopt{,\ }\hlstd{}\hlnum{145}\hlstd{}\hlopt{,\ }\hlstd{}\hlnum{143}\hlstd{}\hlopt{,\ }\hlstd{}\hlnum{141}\hlstd{}\hlopt{,\ }\Righttorque\\
\hllin{59\ }\hlstd{}\hlstd{\ \ \ \ \ \ \ \ }\hlstd{}\hlnum{139}\hlstd{}\hlopt{,\ }\hlstd{}\hlnum{137}\hlstd{}\hlopt{,\ }\hlstd{}\hlnum{135}\hlstd{}\hlopt{,\ }\hlstd{}\hlnum{133}\hlstd{}\hlopt{,\ }\hlstd{}\hlnum{131}\hlstd{}\hlopt{,\ }\hlstd{}\hlnum{129}\\
\hllin{60\ }\hlstd{}\hlstd{\ \ \ \ \ \ \ \ }\hlstd{}\hlkwa{DB\ }\hlstd{}\hlnum{127}\hlstd{}\hlopt{,\ }\hlstd{}\hlnum{125}\hlstd{}\hlopt{,\ }\hlstd{}\hlnum{123}\hlstd{}\hlopt{,\ }\hlstd{}\hlnum{121}\hlstd{}\hlopt{,\ }\hlstd{}\hlnum{119}\hlstd{}\hlopt{,\ }\hlstd{}\hlnum{117}\hlstd{}\hlopt{,\ }\hlstd{}\hlnum{115}\hlstd{}\hlopt{,\ }\hlstd{}\hlnum{113}\hlstd{}\hlopt{,\ }\hlstd{}\hlnum{111}\hlstd{}\hlopt{,\ }\hlstd{}\hlnum{109}\hlstd{}\hlopt{,\ }\Righttorque\\
\hllin{61\ }\hlstd{}\hlstd{\ \ \ \ \ \ \ \ }\hlstd{}\hlnum{107}\hlstd{}\hlopt{,\ }\hlstd{}\hlnum{105}\hlstd{}\hlopt{,\ }\hlstd{}\hlnum{103}\hlstd{}\hlopt{,\ }\hlstd{}\hlnum{101}\hlstd{}\hlopt{,\ }\hlstd{}\hlnum{99}\hlstd{}\hlopt{,\ }\hlstd{}\hlnum{97}\\
\hllin{62\ }\hlstd{}\hlstd{\ \ \ \ \ \ \ \ }\hlstd{}\hlkwa{DB\ }\hlstd{}\hlnum{95}\hlstd{}\hlopt{,\ }\hlstd{}\hlnum{93}\hlstd{}\hlopt{,\ }\hlstd{}\hlnum{91}\hlstd{}\hlopt{,\ }\hlstd{}\hlnum{89}\hlstd{}\hlopt{,\ }\hlstd{}\hlnum{87}\hlstd{}\hlopt{,\ }\hlstd{}\hlnum{85}\hlstd{}\hlopt{,\ }\hlstd{}\hlnum{83}\hlstd{}\hlopt{,\ }\hlstd{}\hlnum{81}\hlstd{}\hlopt{,\ }\hlstd{}\hlnum{79}\hlstd{}\hlopt{,\ }\hlstd{}\hlnum{77}\hlstd{}\hlopt{,\ }\hlstd{}\hlnum{75}\hlstd{}\hlopt{,\ }\hlstd{}\hlnum{73}\hlstd{}\hlopt{,\ }\Righttorque\\
\hllin{63\ }\hlstd{}\hlstd{\ \ \ \ \ \ \ \ }\hlstd{}\hlnum{71}\hlstd{}\hlopt{,\ }\hlstd{}\hlnum{69}\hlstd{}\hlopt{,\ }\hlstd{}\hlnum{67}\hlstd{}\hlopt{,\ }\hlstd{}\hlnum{65}\\
\hllin{64\ }\hlstd{}\hlstd{\ \ \ \ \ \ \ \ }\hlstd{}\hlkwa{DB\ }\hlstd{}\hlnum{63}\hlstd{}\hlopt{,\ }\hlstd{}\hlnum{61}\hlstd{}\hlopt{,\ }\hlstd{}\hlnum{59}\hlstd{}\hlopt{,\ }\hlstd{}\hlnum{57}\hlstd{}\hlopt{,\ }\hlstd{}\hlnum{55}\hlstd{}\hlopt{,\ }\hlstd{}\hlnum{53}\hlstd{}\hlopt{,\ }\hlstd{}\hlnum{51}\hlstd{}\hlopt{,\ }\hlstd{}\hlnum{49}\hlstd{}\hlopt{,\ }\hlstd{}\hlnum{47}\hlstd{}\hlopt{,\ }\hlstd{}\hlnum{45}\hlstd{}\hlopt{,\ }\hlstd{}\hlnum{43}\hlstd{}\hlopt{,\ }\hlstd{}\hlnum{41}\hlstd{}\hlopt{,\ }\Righttorque\\
\hllin{65\ }\hlstd{}\hlstd{\ \ \ \ \ \ \ \ }\hlstd{}\hlnum{39}\hlstd{}\hlopt{,\ }\hlstd{}\hlnum{37}\hlstd{}\hlopt{,\ }\hlstd{}\hlnum{35}\hlstd{}\hlopt{,\ }\hlstd{}\hlnum{33}\\
\hllin{66\ }\hlstd{}\hlstd{\ \ \ \ \ \ \ \ }\hlstd{}\hlkwa{DB\ }\hlstd{}\hlnum{31}\hlstd{}\hlopt{,\ }\hlstd{}\hlnum{29}\hlstd{}\hlopt{,\ }\hlstd{}\hlnum{27}\hlstd{}\hlopt{,\ }\hlstd{}\hlnum{25}\hlstd{}\hlopt{,\ }\hlstd{}\hlnum{23}\hlstd{}\hlopt{,\ }\hlstd{}\hlnum{21}\hlstd{}\hlopt{,\ }\hlstd{}\hlnum{19}\hlstd{}\hlopt{,\ }\hlstd{}\hlnum{17}\hlstd{}\hlopt{,\ }\hlstd{}\hlnum{15}\hlstd{}\hlopt{,\ }\hlstd{}\hlnum{13}\hlstd{}\hlopt{,\ }\hlstd{}\hlnum{11}\hlstd{}\hlopt{,\ }\hlstd{}\hlnum{9}\hlstd{}\hlopt{,\ }\hlstd{}\hlnum{7}\hlstd{}\hlopt{,\ }\Righttorque\\
\hllin{67\ }\hlstd{}\hlstd{\ \ \ \ \ \ \ \ }\hlstd{}\hlnum{5}\hlstd{}\hlopt{,\ }\hlstd{}\hlnum{3}\hlstd{}\hlopt{,\ }\hlstd{}\hlnum{1}\\
\hllin{68\ }\hlstd{}\hlstd{\ \ \ \ }\hlstd{SINE\textunderscore WAVE\ }\hlkwa{DB\ }\hlstd{}\hlnum{128}\hlstd{}\hlopt{,\ }\hlstd{}\hlnum{131}\hlstd{}\hlopt{,\ }\hlstd{}\hlnum{134}\hlstd{}\hlopt{,\ }\hlstd{}\hlnum{137}\hlstd{}\hlopt{,\ }\hlstd{}\hlnum{140}\hlstd{}\hlopt{,\ }\hlstd{}\hlnum{143}\hlstd{}\hlopt{,\ }\hlstd{}\hlnum{146}\hlstd{}\hlopt{,\ }\hlstd{}\hlnum{149}\hlstd{}\hlopt{,\ }\hlstd{}\hlnum{152}\hlstd{}\hlopt{,}\Righttorque\\
\hllin{69\ }\hlstd{}\hlstd{\ \ \ \ }\hlstd{}\hlnum{155}\hlstd{}\hlopt{,\ }\hlstd{}\hlnum{158}\hlstd{}\hlopt{,\ }\hlstd{}\hlnum{162}\hlstd{}\hlopt{,\ }\hlstd{}\hlnum{165}\hlstd{}\hlopt{,\ }\hlstd{}\hlnum{167}\hlstd{}\hlopt{,\ }\hlstd{}\hlnum{170}\hlstd{}\hlopt{,\ }\hlstd{}\hlnum{173}\\
\hllin{70\ }\hlstd{}\hlstd{\ \ \ \ \ \ \ \ }\hlstd{}\hlkwa{DB\ }\hlstd{}\hlnum{176}\hlstd{}\hlopt{,\ }\hlstd{}\hlnum{179}\hlstd{}\hlopt{,\ }\hlstd{}\hlnum{182}\hlstd{}\hlopt{,\ }\hlstd{}\hlnum{185}\hlstd{}\hlopt{,\ }\hlstd{}\hlnum{188}\hlstd{}\hlopt{,\ }\hlstd{}\hlnum{190}\hlstd{}\hlopt{,\ }\hlstd{}\hlnum{193}\hlstd{}\hlopt{,\ }\hlstd{}\hlnum{196}\hlstd{}\hlopt{,\ }\hlstd{}\hlnum{198}\hlstd{}\hlopt{,\ }\hlstd{}\hlnum{201}\hlstd{}\hlopt{,\ }\Righttorque\\
\hllin{71\ }\hlstd{}\hlstd{\ \ \ \ \ \ \ \ }\hlstd{}\hlnum{203}\hlstd{}\hlopt{,\ }\hlstd{}\hlnum{206}\hlstd{}\hlopt{,\ }\hlstd{}\hlnum{208}\hlstd{}\hlopt{,\ }\hlstd{}\hlnum{211}\hlstd{}\hlopt{,\ }\hlstd{}\hlnum{213}\hlstd{}\hlopt{,\ }\hlstd{}\hlnum{215}\\
\hllin{72\ }\hlstd{}\hlstd{\ \ \ \ \ \ \ \ }\hlstd{}\hlkwa{DB\ }\hlstd{}\hlnum{218}\hlstd{}\hlopt{,\ }\hlstd{}\hlnum{220}\hlstd{}\hlopt{,\ }\hlstd{}\hlnum{222}\hlstd{}\hlopt{,\ }\hlstd{}\hlnum{224}\hlstd{}\hlopt{,\ }\hlstd{}\hlnum{226}\hlstd{}\hlopt{,\ }\hlstd{}\hlnum{228}\hlstd{}\hlopt{,\ }\hlstd{}\hlnum{230}\hlstd{}\hlopt{,\ }\hlstd{}\hlnum{232}\hlstd{}\hlopt{,\ }\hlstd{}\hlnum{234}\hlstd{}\hlopt{,\ }\hlstd{}\hlnum{235}\hlstd{}\hlopt{,\ }\Righttorque\\
\hllin{73\ }\hlstd{}\hlstd{\ \ \ \ \ \ \ \ }\hlstd{}\hlnum{237}\hlstd{}\hlopt{,\ }\hlstd{}\hlnum{238}\hlstd{}\hlopt{,\ }\hlstd{}\hlnum{240}\hlstd{}\hlopt{,\ }\hlstd{}\hlnum{241}\hlstd{}\hlopt{,\ }\hlstd{}\hlnum{243}\hlstd{}\hlopt{,\ }\hlstd{}\hlnum{244}\\
\hllin{74\ }\hlstd{}\hlstd{\ \ \ \ \ \ \ \ }\hlstd{}\hlkwa{DB\ }\hlstd{}\hlnum{245}\hlstd{}\hlopt{,\ }\hlstd{}\hlnum{246}\hlstd{}\hlopt{,\ }\hlstd{}\hlnum{248}\hlstd{}\hlopt{,\ }\hlstd{}\hlnum{249}\hlstd{}\hlopt{,\ }\hlstd{}\hlnum{250}\hlstd{}\hlopt{,\ }\hlstd{}\hlnum{250}\hlstd{}\hlopt{,\ }\hlstd{}\hlnum{251}\hlstd{}\hlopt{,\ }\hlstd{}\hlnum{252}\hlstd{}\hlopt{,\ }\hlstd{}\hlnum{253}\hlstd{}\hlopt{,\ }\hlstd{}\hlnum{253}\hlstd{}\hlopt{,\ }\Righttorque\\
\hllin{75\ }\hlstd{}\hlstd{\ \ \ \ \ \ \ \ }\hlstd{}\hlnum{254}\hlstd{}\hlopt{,\ }\hlstd{}\hlnum{254}\hlstd{}\hlopt{,\ }\hlstd{}\hlnum{254}\hlstd{}\hlopt{,\ }\hlstd{}\hlnum{255}\hlstd{}\hlopt{,\ }\hlstd{}\hlnum{255}\hlstd{}\hlopt{,\ }\hlstd{}\hlnum{255}\\
\hllin{76\ }\hlstd{}\hlstd{\ \ \ \ \ \ \ \ }\hlstd{}\hlkwa{DB\ }\hlstd{}\hlnum{255}\hlstd{}\hlopt{,\ }\hlstd{}\hlnum{255}\hlstd{}\hlopt{,\ }\hlstd{}\hlnum{255}\hlstd{}\hlopt{,\ }\hlstd{}\hlnum{255}\hlstd{}\hlopt{,\ }\hlstd{}\hlnum{254}\hlstd{}\hlopt{,\ }\hlstd{}\hlnum{254}\hlstd{}\hlopt{,\ }\hlstd{}\hlnum{254}\hlstd{}\hlopt{,\ }\hlstd{}\hlnum{253}\hlstd{}\hlopt{,\ }\hlstd{}\hlnum{253}\hlstd{}\hlopt{,\ }\hlstd{}\hlnum{252}\hlstd{}\hlopt{,\ }\Righttorque\\
\hllin{77\ }\hlstd{}\hlstd{\ \ \ \ \ \ \ \ }\hlstd{}\hlnum{251}\hlstd{}\hlopt{,\ }\hlstd{}\hlnum{250}\hlstd{}\hlopt{,\ }\hlstd{}\hlnum{250}\hlstd{}\hlopt{,\ }\hlstd{}\hlnum{249}\hlstd{}\hlopt{,\ }\hlstd{}\hlnum{248}\hlstd{}\hlopt{,\ }\hlstd{}\hlnum{246}\\
\hllin{78\ }\hlstd{}\hlstd{\ \ \ \ \ \ \ \ }\hlstd{}\hlkwa{DB\ }\hlstd{}\hlnum{245}\hlstd{}\hlopt{,\ }\hlstd{}\hlnum{244}\hlstd{}\hlopt{,\ }\hlstd{}\hlnum{243}\hlstd{}\hlopt{,\ }\hlstd{}\hlnum{241}\hlstd{}\hlopt{,\ }\hlstd{}\hlnum{240}\hlstd{}\hlopt{,\ }\hlstd{}\hlnum{238}\hlstd{}\hlopt{,\ }\hlstd{}\hlnum{237}\hlstd{}\hlopt{,\ }\hlstd{}\hlnum{235}\hlstd{}\hlopt{,\ }\hlstd{}\hlnum{234}\hlstd{}\hlopt{,\ }\hlstd{}\hlnum{232}\hlstd{}\hlopt{,\ }\Righttorque\\
\hllin{79\ }\hlstd{}\hlstd{\ \ \ \ \ \ \ \ }\hlstd{}\hlnum{230}\hlstd{}\hlopt{,\ }\hlstd{}\hlnum{228}\hlstd{}\hlopt{,\ }\hlstd{}\hlnum{226}\hlstd{}\hlopt{,\ }\hlstd{}\hlnum{224}\hlstd{}\hlopt{,\ }\hlstd{}\hlnum{222}\hlstd{}\hlopt{,\ }\hlstd{}\hlnum{220}\\
\hllin{80\ }\hlstd{}\hlstd{\ \ \ \ \ \ \ \ }\hlstd{}\hlkwa{DB\ }\hlstd{}\hlnum{218}\hlstd{}\hlopt{,\ }\hlstd{}\hlnum{215}\hlstd{}\hlopt{,\ }\hlstd{}\hlnum{213}\hlstd{}\hlopt{,\ }\hlstd{}\hlnum{211}\hlstd{}\hlopt{,\ }\hlstd{}\hlnum{208}\hlstd{}\hlopt{,\ }\hlstd{}\hlnum{206}\hlstd{}\hlopt{,\ }\hlstd{}\hlnum{203}\hlstd{}\hlopt{,\ }\hlstd{}\hlnum{201}\hlstd{}\hlopt{,\ }\hlstd{}\hlnum{198}\hlstd{}\hlopt{,\ }\hlstd{}\hlnum{196}\hlstd{}\hlopt{,\ }\Righttorque\\
\hllin{81\ }\hlstd{}\hlstd{\ \ \ \ \ \ \ \ }\hlstd{}\hlnum{193}\hlstd{}\hlopt{,\ }\hlstd{}\hlnum{190}\hlstd{}\hlopt{,\ }\hlstd{}\hlnum{188}\hlstd{}\hlopt{,\ }\hlstd{}\hlnum{185}\hlstd{}\hlopt{,\ }\hlstd{}\hlnum{182}\hlstd{}\hlopt{,\ }\hlstd{}\hlnum{179}\\
\hllin{82\ }\hlstd{}\hlstd{\ \ \ \ \ \ \ \ }\hlstd{}\hlkwa{DB\ }\hlstd{}\hlnum{176}\hlstd{}\hlopt{,\ }\hlstd{}\hlnum{173}\hlstd{}\hlopt{,\ }\hlstd{}\hlnum{170}\hlstd{}\hlopt{,\ }\hlstd{}\hlnum{167}\hlstd{}\hlopt{,\ }\hlstd{}\hlnum{165}\hlstd{}\hlopt{,\ }\hlstd{}\hlnum{162}\hlstd{}\hlopt{,\ }\hlstd{}\hlnum{158}\hlstd{}\hlopt{,\ }\hlstd{}\hlnum{155}\hlstd{}\hlopt{,\ }\hlstd{}\hlnum{152}\hlstd{}\hlopt{,\ }\hlstd{}\hlnum{149}\hlstd{}\hlopt{,\ }\Righttorque\\
\hllin{83\ }\hlstd{}\hlstd{\ \ \ \ \ \ \ \ }\hlstd{}\hlnum{146}\hlstd{}\hlopt{,\ }\hlstd{}\hlnum{143}\hlstd{}\hlopt{,\ }\hlstd{}\hlnum{140}\hlstd{}\hlopt{,\ }\hlstd{}\hlnum{137}\hlstd{}\hlopt{,\ }\hlstd{}\hlnum{134}\hlstd{}\hlopt{,\ }\hlstd{}\hlnum{131}\\
\hllin{84\ }\hlstd{}\hlstd{\ \ \ \ \ \ \ \ }\hlstd{}\hlkwa{DB\ }\hlstd{}\hlnum{128}\hlstd{}\hlopt{,\ }\hlstd{}\hlnum{124}\hlstd{}\hlopt{,\ }\hlstd{}\hlnum{121}\hlstd{}\hlopt{,\ }\hlstd{}\hlnum{118}\hlstd{}\hlopt{,\ }\hlstd{}\hlnum{115}\hlstd{}\hlopt{,\ }\hlstd{}\hlnum{112}\hlstd{}\hlopt{,\ }\hlstd{}\hlnum{109}\hlstd{}\hlopt{,\ }\hlstd{}\hlnum{106}\hlstd{}\hlopt{,\ }\hlstd{}\hlnum{103}\hlstd{}\hlopt{,\ }\hlstd{}\hlnum{100}\hlstd{}\hlopt{,\ }\Righttorque\\
\hllin{85\ }\hlstd{}\hlstd{\ \ \ \ \ \ \ \ }\hlstd{}\hlnum{97}\hlstd{}\hlopt{,\ }\hlstd{}\hlnum{93}\hlstd{}\hlopt{,\ }\hlstd{}\hlnum{90}\hlstd{}\hlopt{,\ }\hlstd{}\hlnum{88}\hlstd{}\hlopt{,\ }\hlstd{}\hlnum{85}\hlstd{}\hlopt{,\ }\hlstd{}\hlnum{82}\\
\hllin{86\ }\hlstd{}\hlstd{\ \ \ \ \ \ \ \ }\hlstd{}\hlkwa{DB\ }\hlstd{}\hlnum{79}\hlstd{}\hlopt{,\ }\hlstd{}\hlnum{76}\hlstd{}\hlopt{,\ }\hlstd{}\hlnum{73}\hlstd{}\hlopt{,\ }\hlstd{}\hlnum{70}\hlstd{}\hlopt{,\ }\hlstd{}\hlnum{67}\hlstd{}\hlopt{,\ }\hlstd{}\hlnum{65}\hlstd{}\hlopt{,\ }\hlstd{}\hlnum{62}\hlstd{}\hlopt{,\ }\hlstd{}\hlnum{59}\hlstd{}\hlopt{,\ }\hlstd{}\hlnum{57}\hlstd{}\hlopt{,\ }\hlstd{}\hlnum{54}\hlstd{}\hlopt{,\ }\hlstd{}\hlnum{52}\hlstd{}\hlopt{,\ }\hlstd{}\hlnum{49}\hlstd{}\hlopt{,\ }\Righttorque\\
\hllin{87\ }\hlstd{}\hlstd{\ \ \ \ \ \ \ \ }\hlstd{}\hlnum{47}\hlstd{}\hlopt{,\ }\hlstd{}\hlnum{44}\hlstd{}\hlopt{,\ }\hlstd{}\hlnum{42}\hlstd{}\hlopt{,\ }\hlstd{}\hlnum{40}\\
\hllin{88\ }\hlstd{}\hlstd{\ \ \ \ \ \ \ \ }\hlstd{}\hlkwa{DB\ }\hlstd{}\hlnum{37}\hlstd{}\hlopt{,\ }\hlstd{}\hlnum{35}\hlstd{}\hlopt{,\ }\hlstd{}\hlnum{33}\hlstd{}\hlopt{,\ }\hlstd{}\hlnum{31}\hlstd{}\hlopt{,\ }\hlstd{}\hlnum{29}\hlstd{}\hlopt{,\ }\hlstd{}\hlnum{27}\hlstd{}\hlopt{,\ }\hlstd{}\hlnum{25}\hlstd{}\hlopt{,\ }\hlstd{}\hlnum{23}\hlstd{}\hlopt{,\ }\hlstd{}\hlnum{21}\hlstd{}\hlopt{,\ }\hlstd{}\hlnum{20}\hlstd{}\hlopt{,\ }\hlstd{}\hlnum{18}\hlstd{}\hlopt{,\ }\hlstd{}\hlnum{17}\hlstd{}\hlopt{,\ }\Righttorque\\
\hllin{89\ }\hlstd{}\hlstd{\ \ \ \ \ \ \ \ }\hlstd{}\hlnum{15}\hlstd{}\hlopt{,\ }\hlstd{}\hlnum{14}\hlstd{}\hlopt{,\ }\hlstd{}\hlnum{12}\hlstd{}\hlopt{,\ }\hlstd{}\hlnum{11}\\
\hllin{90\ }\hlstd{}\hlstd{\ \ \ \ \ \ \ \ }\hlstd{}\hlkwa{DB\ }\hlstd{}\hlnum{10}\hlstd{}\hlopt{,\ }\hlstd{}\hlnum{9}\hlstd{}\hlopt{,\ }\hlstd{}\hlnum{7}\hlstd{}\hlopt{,\ }\hlstd{}\hlnum{6}\hlstd{}\hlopt{,\ }\hlstd{}\hlnum{5}\hlstd{}\hlopt{,\ }\hlstd{}\hlnum{5}\hlstd{}\hlopt{,\ }\hlstd{}\hlnum{4}\hlstd{}\hlopt{,\ }\hlstd{}\hlnum{3}\hlstd{}\hlopt{,\ }\hlstd{}\hlnum{2}\hlstd{}\hlopt{,\ }\hlstd{}\hlnum{2}\hlstd{}\hlopt{,\ }\hlstd{}\hlnum{1}\hlstd{}\hlopt{,\ }\hlstd{}\hlnum{1}\hlstd{}\hlopt{,\ }\hlstd{}\hlnum{1}\hlstd{}\hlopt{,\ }\hlstd{}\hlnum{0}\hlstd{}\hlopt{,\ }\hlstd{}\hlnum{0}\hlstd{}\hlopt{,\ }\hlstd{}\hlnum{0}\\
\hllin{91\ }\hlstd{}\hlstd{\ \ \ \ \ \ \ \ }\hlstd{}\hlkwa{DB\ }\hlstd{}\hlnum{0}\hlstd{}\hlopt{,\ }\hlstd{}\hlnum{0}\hlstd{}\hlopt{,\ }\hlstd{}\hlnum{0}\hlstd{}\hlopt{,\ }\hlstd{}\hlnum{0}\hlstd{}\hlopt{,\ }\hlstd{}\hlnum{1}\hlstd{}\hlopt{,\ }\hlstd{}\hlnum{1}\hlstd{}\hlopt{,\ }\hlstd{}\hlnum{1}\hlstd{}\hlopt{,\ }\hlstd{}\hlnum{2}\hlstd{}\hlopt{,\ }\hlstd{}\hlnum{2}\hlstd{}\hlopt{,\ }\hlstd{}\hlnum{3}\hlstd{}\hlopt{,\ }\hlstd{}\hlnum{4}\hlstd{}\hlopt{,\ }\hlstd{}\hlnum{5}\hlstd{}\hlopt{,\ }\hlstd{}\hlnum{5}\hlstd{}\hlopt{,\ }\hlstd{}\hlnum{6}\hlstd{}\hlopt{,\ }\hlstd{}\hlnum{7}\hlstd{}\hlopt{,\ }\hlstd{}\hlnum{9}\\
\hllin{92\ }\hlstd{}\hlstd{\ \ \ \ \ \ \ \ }\hlstd{}\hlkwa{DB\ }\hlstd{}\hlnum{10}\hlstd{}\hlopt{,\ }\hlstd{}\hlnum{11}\hlstd{}\hlopt{,\ }\hlstd{}\hlnum{12}\hlstd{}\hlopt{,\ }\hlstd{}\hlnum{14}\hlstd{}\hlopt{,\ }\hlstd{}\hlnum{15}\hlstd{}\hlopt{,\ }\hlstd{}\hlnum{17}\hlstd{}\hlopt{,\ }\hlstd{}\hlnum{18}\hlstd{}\hlopt{,\ }\hlstd{}\hlnum{20}\hlstd{}\hlopt{,\ }\hlstd{}\hlnum{21}\hlstd{}\hlopt{,\ }\hlstd{}\hlnum{23}\hlstd{}\hlopt{,\ }\hlstd{}\hlnum{25}\hlstd{}\hlopt{,\ }\hlstd{}\hlnum{27}\hlstd{}\hlopt{,\ }\Righttorque\\
\hllin{93\ }\hlstd{}\hlstd{\ \ \ \ \ \ \ \ }\hlstd{}\hlnum{29}\hlstd{}\hlopt{,\ }\hlstd{}\hlnum{31}\hlstd{}\hlopt{,\ }\hlstd{}\hlnum{33}\hlstd{}\hlopt{,\ }\hlstd{}\hlnum{35}\\
\hllin{94\ }\hlstd{}\hlstd{\ \ \ \ \ \ \ \ }\hlstd{}\hlkwa{DB\ }\hlstd{}\hlnum{37}\hlstd{}\hlopt{,\ }\hlstd{}\hlnum{40}\hlstd{}\hlopt{,\ }\hlstd{}\hlnum{42}\hlstd{}\hlopt{,\ }\hlstd{}\hlnum{44}\hlstd{}\hlopt{,\ }\hlstd{}\hlnum{47}\hlstd{}\hlopt{,\ }\hlstd{}\hlnum{49}\hlstd{}\hlopt{,\ }\hlstd{}\hlnum{52}\hlstd{}\hlopt{,\ }\hlstd{}\hlnum{54}\hlstd{}\hlopt{,\ }\hlstd{}\hlnum{57}\hlstd{}\hlopt{,\ }\hlstd{}\hlnum{59}\hlstd{}\hlopt{,\ }\hlstd{}\hlnum{62}\hlstd{}\hlopt{,\ }\hlstd{}\hlnum{65}\hlstd{}\hlopt{,\ }\Righttorque\\
\hllin{95\ }\hlstd{}\hlstd{\ \ \ \ \ \ \ \ }\hlstd{}\hlnum{67}\hlstd{}\hlopt{,\ }\hlstd{}\hlnum{70}\hlstd{}\hlopt{,\ }\hlstd{}\hlnum{73}\hlstd{}\hlopt{,\ }\hlstd{}\hlnum{76}\\
\hllin{96\ }\hlstd{}\hlstd{\ \ \ \ \ \ \ \ }\hlstd{}\hlkwa{DB\ }\hlstd{}\hlnum{79}\hlstd{}\hlopt{,\ }\hlstd{}\hlnum{82}\hlstd{}\hlopt{,\ }\hlstd{}\hlnum{85}\hlstd{}\hlopt{,\ }\hlstd{}\hlnum{88}\hlstd{}\hlopt{,\ }\hlstd{}\hlnum{90}\hlstd{}\hlopt{,\ }\hlstd{}\hlnum{93}\hlstd{}\hlopt{,\ }\hlstd{}\hlnum{97}\hlstd{}\hlopt{,\ }\hlstd{}\hlnum{100}\hlstd{}\hlopt{,\ }\hlstd{}\hlnum{103}\hlstd{}\hlopt{,\ }\hlstd{}\hlnum{106}\hlstd{}\hlopt{,\ }\hlstd{}\hlnum{109}\hlstd{}\hlopt{,\ }\Righttorque\\
\hllin{97\ }\hlstd{}\hlstd{\ \ \ \ \ \ \ \ }\hlstd{}\hlnum{112}\hlstd{}\hlopt{,\ }\hlstd{}\hlnum{115}\hlstd{}\hlopt{,\ }\hlstd{}\hlnum{118}\hlstd{}\hlopt{,\ }\hlstd{}\hlnum{121}\hlstd{}\hlopt{,\ }\hlstd{}\hlnum{124}\\
\hllin{98\ }\hlstd{}\hlstd{\ \ \ \ }\hlstd{WAVE\textunderscore TYPE\ }\hlkwa{DB\ }\hlstd{}\hlnum{00h}\\
\hllin{99\ }\hlstd{DATA\ }\hlkwa{ENDS}\\
\hllin{100\ }\hlstd{\\
\hllin{101\ }STACKS\ }\hlkwa{SEGMENT\ }\hlstd{STACK\\
\hllin{102\ }}\hlstd{\ \ \ \ }\hlstd{}\hlkwa{DB\ }\hlstd{}\hlnum{100\ }\hlstd{dup}\hlopt{(}\hlstd{?}\hlopt{)}\\
\hllin{103\ }\hlstd{STACKS\ }\hlkwa{ENDS}\\
\hllin{104\ }\hlstd{\\
\hllin{105\ }CODE\ }\hlkwa{SEGMENT}\\
\hllin{106\ }\hlstd{}\hlstd{\ \ \ \ }\hlstd{}\hlkwa{ASSUME\ }\hlstd{}\hlkwb{CS}\hlstd{}\hlopt{:}\hlstd{CODE}\hlopt{,\ }\hlstd{}\hlkwb{DS}\hlstd{}\hlopt{:}\hlstd{DATA}\hlopt{,\ }\hlstd{}\hlkwb{SS}\hlstd{}\hlopt{:}\hlstd{STACKS\\
\hllin{107\ }\\
\hllin{108\ }}\hlstd{\ \ \ \ }\hlstd{\\
\hllin{109\ }MAIN\ }\hlkwa{PROC\ FAR}\\
\hllin{110\ }\hlstd{}\\
\hllin{111\ }\hlkwc{START:}\\
\hllin{112\ }\hlstd{}\hlstd{\ \ \ \ }\hlstd{}\hlkwa{MOV\ }\hlstd{}\hlkwb{AX}\hlstd{}\hlopt{,\ }\hlstd{DATA\\
\hllin{113\ }}\hlstd{\ \ \ \ }\hlstd{}\hlkwa{MOV\ }\hlstd{}\hlkwb{DS}\hlstd{}\hlopt{,\ }\hlstd{}\hlkwb{AX}\\
\hllin{114\ }\hlstd{}\hlstd{\ \ \ \ }\hlstd{}\hlkwa{MOV\ }\hlstd{}\hlkwb{AX}\hlstd{}\hlopt{,\ }\hlstd{}\hlnum{0}\\
\hllin{115\ }\hlstd{}\hlstd{\ \ \ \ }\hlstd{}\hlkwa{MOV\ }\hlstd{}\hlkwb{DI}\hlstd{}\hlopt{,\ }\hlstd{}\hlkwb{AX}\\
\hllin{116\ }\hlstd{}\hlstd{\ \ \ \ }\hlstd{}\\
\hllin{117\ }\hlkwc{MAIN\textunderscore LOOP:}\\
\hllin{118\ }\hlstd{}\hlstd{\ \ \ \ }\hlstd{}\hlkwa{MOV\ }\hlstd{}\hlkwb{AH}\hlstd{}\hlopt{,\ }\hlstd{}\hlnum{1}\\
\hllin{119\ }\hlstd{}\hlstd{\ \ \ \ }\hlstd{}\hlkwa{INT\ }\hlstd{}\hlnum{16h}\\
\hllin{120\ }\hlstd{}\hlstd{\ \ \ \ }\hlstd{}\hlkwa{JNZ\ }\hlstd{JUDGE\\
\hllin{121\ }}\hlstd{\ \ \ \ \ \ \ \ }\hlstd{}\\
\hllin{122\ }\hlkwc{START\textunderscore CONVERSION:\ }\\
\hllin{123\ }\hlstd{}\hlstd{\ \ \ \ }\hlstd{}\hlkwa{MOV\ }\hlstd{}\hlkwb{AX}\hlstd{}\hlopt{,\ }\hlstd{}\hlkwb{DI}\\
\hllin{124\ }\hlstd{}\hlstd{\ \ \ \ }\hlstd{}\hlkwa{CMP\ }\hlstd{}\hlkwb{AX}\hlstd{}\hlopt{,\ }\hlstd{}\hlnum{0}\hlstd{ffh\\
\hllin{125\ }}\hlstd{\ \ \ \ }\hlstd{}\hlkwa{JNZ\ }\hlstd{P1\\
\hllin{126\ }}\hlstd{\ \ \ \ }\hlstd{}\hlkwa{MOV\ }\hlstd{}\hlkwb{AX}\hlstd{}\hlopt{,\ }\hlstd{}\hlnum{0h}\\
\hllin{127\ }\hlstd{}\hlstd{\ \ \ \ }\hlstd{}\hlkwa{MOV\ }\hlstd{}\hlkwb{DI}\hlstd{}\hlopt{,\ }\hlstd{}\hlkwb{AX}\\
\hllin{128\ }\hlstd{}\hlkwc{P1:\ }\\
\hllin{129\ }\hlstd{}\hlstd{\ \ \ \ }\hlstd{}\hlkwa{MOV\ }\hlstd{}\hlkwb{AL}\hlstd{}\hlopt{,\ }\hlstd{WAVE\textunderscore TYPE\\
\hllin{130\ }}\hlstd{\ \ \ \ }\hlstd{}\hlkwa{CMP\ }\hlstd{}\hlkwb{AL}\hlstd{}\hlopt{,\ }\hlstd{}\hlnum{00h}\\
\hllin{131\ }\hlstd{}\hlstd{\ \ \ \ }\hlstd{}\hlkwa{JZ\ }\hlstd{WAVE\textunderscore SAW\\
\hllin{132\ }}\hlstd{\ \ \ \ }\hlstd{}\hlkwa{CMP\ }\hlstd{}\hlkwb{AL}\hlstd{}\hlopt{,\ }\hlstd{}\hlnum{01h}\\
\hllin{133\ }\hlstd{}\hlstd{\ \ \ \ }\hlstd{}\hlkwa{JZ\ }\hlstd{WAVE\textunderscore TRIANGLE\\
\hllin{134\ }}\hlstd{\ \ \ \ }\hlstd{}\hlkwa{CMP\ }\hlstd{}\hlkwb{AL}\hlstd{}\hlopt{,\ }\hlstd{}\hlnum{02h}\\
\hllin{135\ }\hlstd{}\hlstd{\ \ \ \ }\hlstd{}\hlkwa{JZ\ }\hlstd{WAVE\textunderscore SINE\\
\hllin{136\ }}\hlstd{\ \ \ \ }\hlstd{}\\
\hllin{137\ }\hlkwc{FLAG:}\\
\hllin{138\ }\hlstd{}\hlstd{\ \ \ \ }\hlstd{\\
\hllin{139\ }}\hlstd{\ \ \ \ }\hlstd{}\hlkwa{PUSH\ }\hlstd{}\hlkwb{AX}\\
\hllin{140\ }\hlstd{}\hlstd{\ \ \ \ }\hlstd{}\hlkwa{PUSH\ }\hlstd{}\hlkwb{BX}\\
\hllin{141\ }\hlstd{}\hlstd{\ \ \ \ }\hlstd{}\hlkwa{PUSH\ }\hlstd{}\hlkwb{CX}\\
\hllin{142\ }\hlstd{}\hlkwc{DELAY:\ }\\
\hllin{143\ }\hlstd{}\hlstd{\ \ \ \ }\hlstd{}\hlkwa{MOV\ }\hlstd{}\hlkwb{CX}\hlstd{}\hlopt{,\ }\hlstd{}\hlnum{03}\hlstd{ffh}\\
\hllin{144\ }\hlkwc{P2:\ }\\
\hllin{145\ }\hlstd{}\hlstd{\ \ \ \ }\hlstd{}\hlkwa{LOOP\ }\hlstd{P2\\
\hllin{146\ }}\hlstd{\ \ \ \ }\hlstd{}\hlkwa{POP\ }\hlstd{}\hlkwb{CX}\\
\hllin{147\ }\hlstd{}\hlstd{\ \ \ \ }\hlstd{}\hlkwa{POP\ }\hlstd{}\hlkwb{BX}\\
\hllin{148\ }\hlstd{}\hlstd{\ \ \ \ }\hlstd{}\hlkwa{POP\ }\hlstd{}\hlkwb{AX}\\
\hllin{149\ }\hlstd{}\hlstd{\ \ \ \ \ \ \ \ }\hlstd{\\
\hllin{150\ }}\hlstd{\ \ \ \ }\hlstd{}\hlkwa{INC\ }\hlstd{}\hlkwb{DI}\\
\hllin{151\ }\hlstd{}\hlstd{\ \ \ \ }\hlstd{}\hlkwa{JMP\ }\hlstd{MAIN\textunderscore LOOP\\
\hllin{152\ }}\hlstd{\ \ \ \ }\hlstd{}\\
\hllin{153\ }\hlkwc{JUDGE:\ }\\
\hllin{154\ }\hlstd{}\hlstd{\ \ \ \ }\hlstd{}\hlkwa{MOV\ }\hlstd{}\hlkwb{AH}\hlstd{}\hlopt{,\ }\hlstd{}\hlnum{1}\\
\hllin{155\ }\hlstd{}\hlstd{\ \ \ \ }\hlstd{}\hlkwa{INT\ }\hlstd{}\hlnum{21h}\\
\hllin{156\ }\hlstd{}\hlstd{\ \ \ \ }\hlstd{\\
\hllin{157\ }}\hlstd{\ \ \ \ }\hlstd{}\hlkwa{CMP\ }\hlstd{}\hlkwb{AL}\hlstd{}\hlopt{,\ }\hlstd{}\hlnum{34h}\\
\hllin{158\ }\hlstd{}\hlstd{\ \ \ \ }\hlstd{}\hlkwa{JZ\ }\hlstd{EXIT\\
\hllin{159\ }}\hlstd{\ \ \ \ }\hlstd{\\
\hllin{160\ }}\hlstd{\ \ \ \ }\hlstd{}\hlkwa{SUB\ }\hlstd{}\hlkwb{AL}\hlstd{}\hlopt{,\ }\hlstd{}\hlnum{31h}\\
\hllin{161\ }\hlstd{}\hlstd{\ \ \ \ }\hlstd{}\hlkwa{MOV\ }\hlstd{WAVE\textunderscore TYPE}\hlopt{,\ }\hlstd{}\hlkwb{AL}\hlstd{\ \ \ \ }\hlkwb{}\\
\hllin{162\ }\hlstd{}\hlstd{\ \ \ \ }\hlstd{}\hlkwa{JMP\ }\hlstd{START\textunderscore CONVERSION\\
\hllin{163\ }}\hlstd{\ \ \ \ }\hlstd{}\\
\hllin{164\ }\hlkwc{EXIT:\ }\\
\hllin{165\ }\hlstd{}\hlstd{\ \ \ \ }\hlstd{}\hlkwa{MOV\ }\hlstd{}\hlkwb{AH}\hlstd{}\hlopt{,\ }\hlstd{}\hlnum{4}\hlstd{}\hlkwb{CH}\\
\hllin{166\ }\hlstd{}\hlstd{\ \ \ \ }\hlstd{}\hlkwa{INT\ }\hlstd{}\hlnum{21h}\\
\hllin{167\ }\hlstd{}\hlstd{\ \ \ \ }\hlstd{}\\
\hllin{168\ }\hlkwc{WAVE\textunderscore SAW:\ }\\
\hllin{169\ }\hlstd{}\hlstd{\ \ \ \ }\hlstd{}\hlkwa{MOV\ }\hlstd{}\hlkwb{DX}\hlstd{}\hlopt{,\ }\hlstd{}\hlnum{280h}\\
\hllin{170\ }\hlstd{}\hlstd{\ \ \ \ }\hlstd{}\hlkwa{MOV\ }\hlstd{}\hlkwb{AL}\hlstd{}\hlopt{,\ }\hlstd{SAW\textunderscore WAVE}\hlopt{{[}}\hlstd{}\hlkwb{DI}\hlstd{}\hlopt{{]}}\\
\hllin{171\ }\hlstd{}\hlstd{\ \ \ \ }\hlstd{}\hlkwa{OUT\ }\hlstd{}\hlkwb{DX}\hlstd{}\hlopt{,\ }\hlstd{}\hlkwb{AL}\\
\hllin{172\ }\hlstd{}\hlstd{\ \ \ \ }\hlstd{}\hlkwa{JMP\ }\hlstd{FLAG}\\
\hllin{173\ }\\
\hllin{174\ }\hlkwc{WAVE\textunderscore TRIANGLE:\ }\\
\hllin{175\ }\hlstd{}\hlstd{\ \ \ \ }\hlstd{}\hlkwa{MOV\ }\hlstd{}\hlkwb{DX}\hlstd{}\hlopt{,\ }\hlstd{}\hlnum{280h}\\
\hllin{176\ }\hlstd{}\hlstd{\ \ \ \ }\hlstd{}\hlkwa{MOV\ }\hlstd{}\hlkwb{AL}\hlstd{}\hlopt{,\ }\hlstd{TRIANGLE\textunderscore WAVE}\hlopt{{[}}\hlstd{}\hlkwb{DI}\hlstd{}\hlopt{{]}}\\
\hllin{177\ }\hlstd{}\hlstd{\ \ \ \ }\hlstd{}\hlkwa{OUT\ }\hlstd{}\hlkwb{DX}\hlstd{}\hlopt{,\ }\hlstd{}\hlkwb{AL}\\
\hllin{178\ }\hlstd{}\hlstd{\ \ \ \ }\hlstd{}\hlkwa{JMP\ }\hlstd{FLAG}\\
\hllin{179\ }\\
\hllin{180\ }\\
\hllin{181\ }\hlkwc{WAVE\textunderscore SINE:\ }\\
\hllin{182\ }\hlstd{}\hlstd{\ \ \ \ }\hlstd{}\hlkwa{MOV\ }\hlstd{}\hlkwb{DX}\hlstd{}\hlopt{,\ }\hlstd{}\hlnum{280h}\\
\hllin{183\ }\hlstd{}\hlstd{\ \ \ \ }\hlstd{}\hlkwa{MOV\ }\hlstd{}\hlkwb{AL}\hlstd{}\hlopt{,\ }\hlstd{SINE\textunderscore WAVE}\hlopt{{[}}\hlstd{}\hlkwb{DI}\hlstd{}\hlopt{{]}}\\
\hllin{184\ }\hlstd{}\hlstd{\ \ \ \ }\hlstd{}\hlkwa{OUT\ }\hlstd{}\hlkwb{DX}\hlstd{}\hlopt{,\ }\hlstd{}\hlkwb{AL}\\
\hllin{185\ }\hlstd{}\hlstd{\ \ \ \ }\hlstd{}\hlkwa{JMP\ }\hlstd{FLAG\\
\hllin{186\ }\\
\hllin{187\ }MAIN\ }\hlkwa{ENDP}\\
\hllin{188\ }\hlstd{CODE\ }\hlkwa{ENDS}\\
\hllin{189\ }\hlstd{}\\
\hllin{190\ }\hlkwa{END\ }\hlstd{START}\\
\mbox{}
\normalfont
\normalsize

\subsection{任务二}
用ADC0809实现A/D转换,用汇编语言程序自动对一个模拟
信号重复采集20组不同的数据,在CRT上将每组数据对应显
示成要求的形式,代码如下所示:
\noindent
\ttfamily
\hlstd{\hllin{01\ }DATA\ }\hlkwa{SEGMENT}\\
\hllin{02\ }\hlstd{}\hlstd{\ \ \ \ }\hlstd{HEAD\ }\hlkwa{DB\ }\hlstd{}\hlnum{0}\hlstd{}\hlkwb{DH}\hlstd{}\hlopt{,\ }\hlstd{}\hlnum{0}\hlstd{}\hlkwb{AH}\hlstd{}\hlopt{,\ }\hlstd{}\hlstr{'D/A}\hlstd{\ \ \ \ \ \ \ \ \ }\hlstr{A/D'}\hlstd{}\hlopt{,\ }\hlstd{}\hlnum{0}\hlstd{}\hlkwb{DH}\hlstd{}\hlopt{,\ }\hlstd{}\hlnum{0}\hlstd{}\hlkwb{AH}\hlstd{}\hlopt{,\ }\hlstd{}\hlstr{'\$'}\hlstd{\\
\hllin{03\ }}\hlstd{\ \ \ \ }\hlstd{INFO\ }\hlkwa{DB\ }\hlstd{}\hlstr{'Please\ input\ c\ to\ get\ the\ datas,\ e\ to\ exit:'}\hlstd{\\
\hllin{04\ }}\hlstd{\ \ \ \ }\hlstd{LINE\ }\hlkwa{DB\ }\hlstd{}\hlnum{0}\hlstd{}\hlkwb{DH}\hlstd{}\hlopt{,\ }\hlstd{}\hlnum{0}\hlstd{}\hlkwb{AH}\hlstd{}\hlopt{,\ }\hlstd{}\hlstr{'\$'}\hlstd{\\
\hllin{05\ }}\hlstd{\ \ \ \ }\hlstd{SPACE\ }\hlkwa{DB\ }\hlstd{}\hlstr{'}\hlstd{\ \ \ \ \ \ \ \ \ }\hlstr{\$'}\hlstd{\\
\hllin{06\ }DATA\ }\hlkwa{ENDS}\\
\hllin{07\ }\hlstd{\\
\hllin{08\ }STACKS\ }\hlkwa{SEGMENT}\\
\hllin{09\ }\hlstd{STACKS\ }\hlkwa{ENDS}\\
\hllin{10\ }\hlstd{\\
\hllin{11\ }CODE\ }\hlkwa{SEGMENT}\\
\hllin{12\ }\hlstd{}\hlstd{\ \ \ \ }\hlstd{}\hlkwa{ASSUME\ }\hlstd{}\hlkwb{CS}\hlstd{}\hlopt{:}\hlstd{CODE}\hlopt{,}\hlstd{}\hlkwb{DS}\hlstd{}\hlopt{:}\hlstd{DATA}\hlopt{,}\hlstd{}\hlkwb{SS}\hlstd{}\hlopt{:}\hlstd{STACKS\\
\hllin{13\ }MAIN\ }\hlkwa{PROC\ FAR}\\
\hllin{14\ }\hlstd{}\\
\hllin{15\ }\hlkwc{START:}\\
\hllin{16\ }\hlstd{}\hlstd{\ \ \ \ }\hlstd{}\hlkwa{MOV\ }\hlstd{}\hlkwb{AX}\hlstd{}\hlopt{,}\hlstd{DATA\\
\hllin{17\ }}\hlstd{\ \ \ \ }\hlstd{}\hlkwa{MOV\ }\hlstd{}\hlkwb{DS}\hlstd{}\hlopt{,}\hlstd{}\hlkwb{AX}\\
\hllin{18\ }\hlstd{}\hlstd{\ \ \ \ }\hlstd{\\
\hllin{19\ }}\hlstd{\ \ \ \ }\hlstd{}\hlkwa{MOV\ }\hlstd{}\hlkwb{BX}\hlstd{}\hlopt{,\ }\hlstd{}\hlnum{0}\\
\hllin{20\ }\hlstd{}\\
\hllin{21\ }\hlkwc{DISPLAY:}\\
\hllin{22\ }\hlstd{\\
\hllin{23\ }}\hlstd{\ \ \ \ }\hlstd{}\hlkwa{MOV\ }\hlstd{}\hlkwb{BL}\hlstd{}\hlopt{,}\hlstd{}\hlkwb{BH\ }\\
\hllin{24\ }\hlstd{}\hlstd{\ \ \ \ }\hlstd{}\hlkwa{INC\ }\hlstd{}\hlkwb{BH}\\
\hllin{25\ }\hlstd{}\hlstd{\ \ \ \ }\hlstd{}\hlkwa{MOV\ }\hlstd{}\hlkwb{CX}\hlstd{}\hlopt{,\ }\hlstd{}\hlnum{20}\\
\hllin{26\ }\hlstd{}\hlstd{\ \ \ \ }\hlstd{}\hlkwa{MOV\ }\hlstd{}\hlkwb{AH}\hlstd{}\hlopt{,\ }\hlstd{}\hlnum{9}\\
\hllin{27\ }\hlstd{}\hlstd{\ \ \ \ }\hlstd{}\hlkwa{LEA\ }\hlstd{}\hlkwb{DX}\hlstd{}\hlopt{,\ }\hlstd{HEAD\\
\hllin{28\ }}\hlstd{\ \ \ \ }\hlstd{}\hlkwa{INT\ }\hlstd{}\hlnum{21H}\\
\hllin{29\ }\hlstd{}\hlkwc{ONE\textunderscore LINE:}\\
\hllin{30\ }\hlstd{}\hlstd{\ \ \ \ }\hlstd{}\hlkwa{MOV\ }\hlstd{}\hlkwb{AL}\hlstd{}\hlopt{,\ }\hlstd{}\hlkwb{BL}\hlstd{\ \ \ }\hlkwb{}\\
\hllin{31\ }\hlstd{}\hlstd{\ \ \ \ }\hlstd{}\hlkwa{MOV\ }\hlstd{}\hlkwb{DX}\hlstd{}\hlopt{,\ }\hlstd{}\hlnum{280H}\hlstd{\ \ \ \ \ \ \ \ }\hlnum{}\\
\hllin{32\ }\hlstd{}\hlstd{\ \ \ \ }\hlstd{}\hlkwa{OUT\ }\hlstd{}\hlkwb{DX}\hlstd{}\hlopt{,\ }\hlstd{}\hlkwb{AL}\hlstd{\ \ \ \ \ \ \ }\hlkwb{}\\
\hllin{33\ }\hlstd{}\hlstd{\ \ \ \ }\hlstd{\\
\hllin{34\ }}\hlstd{\ \ \ \ }\hlstd{}\hlkwa{CALL\ }\hlstd{SHOW}\hlstd{\ \ \ \ \ \ \ \ }\hlstd{\\
\hllin{35\ }}\hlstd{\ \ \ \ }\hlstd{}\hlkwa{CALL\ }\hlstd{DELAY}\hlstd{\ \ \ \ \ \ \ }\hlstd{\\
\hllin{36\ }}\hlstd{\ \ \ \ }\hlstd{\\
\hllin{37\ }}\hlstd{\ \ \ \ }\hlstd{}\hlkwa{MOV\ }\hlstd{}\hlkwb{AH}\hlstd{}\hlopt{,\ }\hlstd{}\hlnum{9}\\
\hllin{38\ }\hlstd{}\hlstd{\ \ \ \ }\hlstd{}\hlkwa{LEA\ }\hlstd{}\hlkwb{DX}\hlstd{}\hlopt{,\ }\hlstd{SPACE\\
\hllin{39\ }}\hlstd{\ \ \ \ }\hlstd{}\hlkwa{INT\ }\hlstd{}\hlnum{21H}\hlstd{\ \ \ \ \ \ \ \ \ \ }\hlnum{}\\
\hllin{40\ }\hlstd{}\hlstd{\ \ \ \ }\hlstd{\\
\hllin{41\ }}\hlstd{\ \ \ \ }\hlstd{}\hlkwa{MOV\ }\hlstd{}\hlkwb{DX}\hlstd{}\hlopt{,\ }\hlstd{}\hlnum{289H}\\
\hllin{42\ }\hlstd{}\hlstd{\ \ \ \ }\hlstd{}\hlkwa{OUT\ }\hlstd{}\hlkwb{DX}\hlstd{}\hlopt{,\ }\hlstd{}\hlkwb{AL}\hlstd{\ \ \ \ \ \ \ }\hlkwb{}\\
\hllin{43\ }\hlstd{}\hlstd{\ \ \ \ }\hlstd{\\
\hllin{44\ }}\hlstd{\ \ \ \ }\hlstd{}\hlkwa{CALL\ }\hlstd{DELAY}\hlstd{\ \ \ \ \ \ \ }\hlstd{\\
\hllin{45\ }}\hlstd{\ \ \ \ }\hlstd{\\
\hllin{46\ }}\hlstd{\ \ \ \ }\hlstd{}\hlkwa{MOV\ }\hlstd{}\hlkwb{DX}\hlstd{}\hlopt{,\ }\hlstd{}\hlnum{289H}\\
\hllin{47\ }\hlstd{}\hlstd{\ \ \ \ }\hlstd{}\hlkwa{IN\ }\hlstd{}\hlkwb{AL}\hlstd{}\hlopt{,\ }\hlstd{}\hlkwb{DX}\hlstd{\ \ \ \ \ \ \ \ }\hlkwb{}\\
\hllin{48\ }\hlstd{}\hlstd{\ \ \ \ }\hlstd{\\
\hllin{49\ }}\hlstd{\ \ \ \ }\hlstd{}\hlkwa{CALL\ }\hlstd{SHOW}\hlstd{\ \ \ \ \ \ \ \ }\hlstd{\\
\hllin{50\ }}\hlstd{\ \ \ \ }\hlstd{\\
\hllin{51\ }}\hlstd{\ \ \ \ }\hlstd{}\hlkwa{MOV\ }\hlstd{}\hlkwb{AH}\hlstd{}\hlopt{,\ }\hlstd{}\hlnum{9}\\
\hllin{52\ }\hlstd{}\hlstd{\ \ \ \ }\hlstd{}\hlkwa{LEA\ }\hlstd{}\hlkwb{DX}\hlstd{}\hlopt{,\ }\hlstd{LINE\\
\hllin{53\ }}\hlstd{\ \ \ \ }\hlstd{}\hlkwa{INT\ }\hlstd{}\hlnum{21H}\hlstd{\ \ \ \ \ \ \ \ \ \ }\hlnum{}\\
\hllin{54\ }\hlstd{}\hlstd{\ \ \ \ }\hlstd{\\
\hllin{55\ }}\hlstd{\ \ \ \ }\hlstd{}\hlkwa{ADD\ }\hlstd{}\hlkwb{BL}\hlstd{}\hlopt{,\ }\hlstd{}\hlnum{0}\hlstd{FH}\hlstd{\ \ \ \ \ \ }\hlstd{\\
\hllin{56\ }}\hlstd{\ \ \ \ }\hlstd{}\hlkwa{LOOP\ }\hlstd{ONE\textunderscore LINE}\hlstd{\ \ \ \ }\hlstd{\\
\hllin{57\ }}\hlstd{\ \ \ \ }\hlstd{\\
\hllin{58\ }}\hlstd{\ \ \ \ }\hlstd{}\hlkwa{MOV\ }\hlstd{}\hlkwb{AH}\hlstd{}\hlopt{,\ }\hlstd{}\hlnum{9}\\
\hllin{59\ }\hlstd{}\hlstd{\ \ \ \ }\hlstd{}\hlkwa{LEA\ }\hlstd{}\hlkwb{DX}\hlstd{}\hlopt{,\ }\hlstd{INFO\\
\hllin{60\ }}\hlstd{\ \ \ \ }\hlstd{}\hlkwa{INT\ }\hlstd{}\hlnum{21H}\hlstd{\ \ \ \ \ \ \ \ \ \ }\hlnum{}\\
\hllin{61\ }\hlstd{}\hlstd{\ \ \ \ }\hlstd{}\\
\hllin{62\ }\hlkwc{ENTER:}\hlstd{\ \ \ \ }\hlkwc{}\\
\hllin{63\ }\hlstd{}\hlstd{\ \ \ \ }\hlstd{}\hlkwa{MOV\ }\hlstd{}\hlkwb{AH}\hlstd{}\hlopt{,\ }\hlstd{}\hlnum{1}\\
\hllin{64\ }\hlstd{}\hlstd{\ \ \ \ }\hlstd{}\hlkwa{INT\ }\hlstd{}\hlnum{21H}\hlstd{\ \ \ \ \ \ \ \ \ \ }\hlnum{}\\
\hllin{65\ }\hlstd{}\hlstd{\ \ \ \ }\hlstd{}\hlkwa{CMP\ }\hlstd{}\hlkwb{AL}\hlstd{}\hlopt{,\ }\hlstd{}\hlstr{'C'}\hlstd{\\
\hllin{66\ }}\hlstd{\ \ \ \ }\hlstd{}\hlkwa{JZ\ }\hlstd{DISPLAY\\
\hllin{67\ }}\hlstd{\ \ \ \ }\hlstd{}\hlkwa{CMP\ }\hlstd{}\hlkwb{AL}\hlstd{}\hlopt{,\ }\hlstd{}\hlstr{'c'}\hlstd{}\hlstd{\ \ \ \ }\hlstd{\\
\hllin{68\ }}\hlstd{\ \ \ \ }\hlstd{}\hlkwa{JZ\ }\hlstd{DISPLAY}\hlstd{\ \ \ \ \ \ \ }\hlstd{\\
\hllin{69\ }}\hlstd{\ \ \ \ }\hlstd{}\hlkwa{CMP\ }\hlstd{}\hlkwb{AL}\hlstd{}\hlopt{,\ }\hlstd{}\hlstr{'E'}\hlstd{\\
\hllin{70\ }}\hlstd{\ \ \ \ }\hlstd{}\hlkwa{JZ\ }\hlstd{EXIT\\
\hllin{71\ }}\hlstd{\ \ \ \ }\hlstd{}\hlkwa{CMP\ }\hlstd{}\hlkwb{AL}\hlstd{}\hlopt{,\ }\hlstd{}\hlstr{'e'}\hlstd{\\
\hllin{72\ }}\hlstd{\ \ \ \ }\hlstd{}\hlkwa{JZ\ }\hlstd{EXIT}\hlstd{\ \ \ \ \ \ \ \ \ \ }\hlstd{\\
\hllin{73\ }}\hlstd{\ \ \ \ }\hlstd{}\hlkwa{JMP\ ENTER}\\
\hllin{74\ }\hlstd{}\hlstd{\ \ \ \ }\hlstd{}\\
\hllin{75\ }\hlkwc{EXIT:}\hlstd{\ \ \ \ }\hlkwc{}\\
\hllin{76\ }\hlstd{}\hlstd{\ \ \ \ }\hlstd{}\hlkwa{MOV\ }\hlstd{}\hlkwb{AH}\hlstd{}\hlopt{,}\hlstd{}\hlnum{4}\hlstd{}\hlkwb{CH}\\
\hllin{77\ }\hlstd{}\hlstd{\ \ \ \ }\hlstd{}\hlkwa{INT\ }\hlstd{}\hlnum{21H}\hlstd{\ \ \ \ \ \ \ \ \ \ }\hlnum{}\\
\hllin{78\ }\hlstd{}\hlstd{\ \ \ \ }\hlstd{\\
\hllin{79\ }DELAY\ }\hlkwa{PROC}\hlstd{\ \ \ \ \ \ \ \ \ \ \ \ \ \ \ \ \ \ \ \ }\hlkwa{}\\
\hllin{80\ }\hlstd{}\hlstd{\ \ \ \ }\hlstd{}\hlkwa{PUSH\ }\hlstd{}\hlkwb{AX}\\
\hllin{81\ }\hlstd{}\hlstd{\ \ \ \ }\hlstd{}\hlkwa{PUSH\ }\hlstd{}\hlkwb{BX}\\
\hllin{82\ }\hlstd{}\hlstd{\ \ \ \ }\hlstd{}\hlkwa{PUSH\ }\hlstd{}\hlkwb{CX}\\
\hllin{83\ }\hlstd{}\hlstd{\ \ \ \ }\hlstd{}\hlkwa{MOV\ }\hlstd{}\hlkwb{CX}\hlstd{}\hlopt{,\ }\hlstd{}\hlnum{0}\hlstd{FFFH}\hlstd{\ \ \ \ }\hlstd{}\\
\hllin{84\ }\hlkwc{WAIT:}\hlstd{\ \ \ }\hlkwc{}\\
\hllin{85\ }\hlstd{}\hlstd{\ \ \ \ }\hlstd{}\hlkwa{LOOP\ WAIT}\\
\hllin{86\ }\hlstd{}\hlstd{\ \ \ \ }\hlstd{\\
\hllin{87\ }}\hlstd{\ \ \ \ }\hlstd{}\hlkwa{POP\ }\hlstd{}\hlkwb{CX}\\
\hllin{88\ }\hlstd{}\hlstd{\ \ \ \ }\hlstd{}\hlkwa{POP\ }\hlstd{}\hlkwb{BX}\\
\hllin{89\ }\hlstd{}\hlstd{\ \ \ \ }\hlstd{}\hlkwa{POP\ }\hlstd{}\hlkwb{AX}\\
\hllin{90\ }\hlstd{}\hlstd{\ \ \ \ }\hlstd{}\hlkwa{RET}\\
\hllin{91\ }\hlstd{DELAY\ }\hlkwa{ENDP}\\
\hllin{92\ }\hlstd{\\
\hllin{93\ }\\
\hllin{94\ }SHOW\ }\hlkwa{PROC\ }\\
\hllin{95\ }\hlstd{}\hlstd{\ \ \ \ }\hlstd{}\hlkwa{PUSH\ }\hlstd{}\hlkwb{AX}\\
\hllin{96\ }\hlstd{}\hlstd{\ \ \ \ }\hlstd{}\hlkwa{AND\ }\hlstd{}\hlkwb{AL}\hlstd{}\hlopt{,\ }\hlstd{}\hlnum{0}\hlstd{F0H}\hlstd{\ \ \ \ }\hlstd{\\
\hllin{97\ }}\hlstd{\ \ \ \ }\hlstd{}\hlkwa{SHR\ }\hlstd{}\hlkwb{AL}\hlstd{}\hlopt{,\ }\hlstd{}\hlnum{1}\\
\hllin{98\ }\hlstd{}\hlstd{\ \ \ \ }\hlstd{}\hlkwa{SHR\ }\hlstd{}\hlkwb{AL}\hlstd{}\hlopt{,\ }\hlstd{}\hlnum{1}\\
\hllin{99\ }\hlstd{}\hlstd{\ \ \ \ }\hlstd{}\hlkwa{SHR\ }\hlstd{}\hlkwb{AL}\hlstd{}\hlopt{,\ }\hlstd{}\hlnum{1}\\
\hllin{100\ }\hlstd{}\hlstd{\ \ \ \ }\hlstd{}\hlkwa{SHR\ }\hlstd{}\hlkwb{AL}\hlstd{}\hlopt{,\ }\hlstd{}\hlnum{1}\hlstd{\ \ \ \ \ \ \ }\hlnum{}\\
\hllin{101\ }\hlstd{}\hlstd{\ \ \ \ }\hlstd{}\hlkwa{CMP\ }\hlstd{}\hlkwb{AL}\hlstd{}\hlopt{,\ }\hlstd{}\hlnum{09H}\hlstd{\ \ \ \ \ \ \ \ }\hlnum{}\\
\hllin{102\ }\hlstd{}\hlstd{\ \ \ \ }\hlstd{}\hlkwa{JBE\ }\hlstd{DIG2}\hlstd{\ \ \ \ \ \ \ \ }\hlstd{\\
\hllin{103\ }}\hlstd{\ \ \ \ }\hlstd{}\hlkwa{ADD\ }\hlstd{}\hlkwb{AL}\hlstd{}\hlopt{,\ }\hlstd{}\hlnum{07H}\hlstd{\ \ \ \ \ }\hlnum{}\\
\hllin{104\ }\hlstd{}\hlkwc{DIG2:}\\
\hllin{105\ }\hlstd{}\hlstd{\ \ \ \ }\hlstd{}\hlkwa{ADD\ }\hlstd{}\hlkwb{AL}\hlstd{}\hlopt{,\ }\hlstd{}\hlnum{30H}\hlstd{\ \ \ \ \ }\hlnum{}\\
\hllin{106\ }\hlstd{}\hlstd{\ \ \ \ }\hlstd{}\hlkwa{MOV\ }\hlstd{}\hlkwb{DL}\hlstd{}\hlopt{,\ }\hlstd{}\hlkwb{AL}\\
\hllin{107\ }\hlstd{}\hlstd{\ \ \ \ }\hlstd{}\hlkwa{MOV\ }\hlstd{}\hlkwb{AH}\hlstd{}\hlopt{,\ }\hlstd{}\hlnum{2}\\
\hllin{108\ }\hlstd{}\hlstd{\ \ \ \ }\hlstd{}\hlkwa{INT\ }\hlstd{}\hlnum{21H}\hlstd{\ \ \ \ \ \ \ \ \ }\hlnum{}\\
\hllin{109\ }\hlstd{}\hlstd{\ \ \ \ }\hlstd{}\hlkwa{POP\ }\hlstd{}\hlkwb{AX}\\
\hllin{110\ }\hlstd{}\hlstd{\ \ \ \ }\hlstd{\\
\hllin{111\ }}\hlstd{\ \ \ \ }\hlstd{}\hlkwa{AND\ }\hlstd{}\hlkwb{AL}\hlstd{}\hlopt{,\ }\hlstd{}\hlnum{0}\hlstd{FH}\hlstd{\ \ \ \ \ }\hlstd{\\
\hllin{112\ }}\hlstd{\ \ \ \ }\hlstd{}\hlkwa{CMP\ }\hlstd{}\hlkwb{AL}\hlstd{}\hlopt{,\ }\hlstd{}\hlnum{09H}\\
\hllin{113\ }\hlstd{}\hlstd{\ \ \ \ }\hlstd{}\hlkwa{JBE\ }\hlstd{DIG1\\
\hllin{114\ }}\hlstd{\ \ \ \ }\hlstd{}\hlkwa{ADD\ }\hlstd{}\hlkwb{AL}\hlstd{}\hlopt{,\ }\hlstd{}\hlnum{07H}\hlstd{\ \ \ \ \ }\hlnum{}\\
\hllin{115\ }\hlstd{}\hlkwc{DIG1:}\\
\hllin{116\ }\hlstd{}\hlstd{\ \ \ \ }\hlstd{}\hlkwa{ADD\ }\hlstd{}\hlkwb{AL}\hlstd{}\hlopt{,\ }\hlstd{}\hlnum{30H}\hlstd{\ \ \ \ \ }\hlnum{}\\
\hllin{117\ }\hlstd{}\hlstd{\ \ \ \ }\hlstd{}\hlkwa{MOV\ }\hlstd{}\hlkwb{DL}\hlstd{}\hlopt{,\ }\hlstd{}\hlkwb{AL}\\
\hllin{118\ }\hlstd{}\hlstd{\ \ \ \ }\hlstd{}\hlkwa{MOV\ }\hlstd{}\hlkwb{AH}\hlstd{}\hlopt{,\ }\hlstd{}\hlnum{2}\\
\hllin{119\ }\hlstd{}\hlstd{\ \ \ \ }\hlstd{}\hlkwa{INT\ }\hlstd{}\hlnum{21H}\hlstd{\ \ \ \ \ \ \ \ \ }\hlnum{}\\
\hllin{120\ }\hlstd{}\hlstd{\ \ \ \ }\hlstd{}\hlkwa{MOV\ }\hlstd{}\hlkwb{DL}\hlstd{}\hlopt{,\ }\hlstd{}\hlstr{'H'}\hlstd{}\hlstd{\ \ \ \ \ }\hlstd{\\
\hllin{121\ }}\hlstd{\ \ \ \ }\hlstd{}\hlkwa{MOV\ }\hlstd{}\hlkwb{AH}\hlstd{}\hlopt{,\ }\hlstd{}\hlnum{2}\\
\hllin{122\ }\hlstd{}\hlstd{\ \ \ \ }\hlstd{}\hlkwa{INT\ }\hlstd{}\hlnum{21H}\hlstd{\ \ \ \ \ \ \ \ \ }\hlnum{}\\
\hllin{123\ }\hlstd{}\hlstd{\ \ \ \ }\hlstd{}\hlkwa{RET}\\
\hllin{124\ }\hlstd{SHOW\ }\hlkwa{ENDP}\\
\hllin{125\ }\hlstd{\\
\hllin{126\ }MAIN\ }\hlkwa{ENDP}\\
\hllin{127\ }\hlstd{CODE\ }\hlkwa{ENDS}\\
\hllin{128\ }\hlstd{}\hlstd{\ \ \ \ }\hlstd{}\hlkwa{END\ }\hlstd{START}\\
\mbox{}
\normalfont
\normalsize

\subsection{完成情况及心得体会}
本次实验使用了AD/DA转换,通过具体代码的编写了解了计
算机IO速度和CPU运算速度的差异,并提升了自己的汇编编
程水平
\section{第七次实验}
\subsection{实验目的}
综合汇编语言编程及I/O接口的知识,提高实际应用的能力
\subsection{任务一}
将TPC实验台上的8255电路A口设置成方式0输入,检测8只
开关的状态;将C口设置成方式0输出,控制8只LED灯。程序
运行后不断地读入8只开关的状态,送往对应的LED灯显示,
直至在计算机键盘上敲入空格键退回DOS
\subsubsection{必做任务}
代码如下所示
\noindent
\ttfamily
\hlstd{\hllin{01\ }DATA\ }\hlkwa{SEGMENT}\\
\hllin{02\ }\hlstd{DATA\ }\hlkwa{ENDS}\\
\hllin{03\ }\hlstd{\\
\hllin{04\ }STACKS\ }\hlkwa{SEGMENT}\\
\hllin{05\ }\hlstd{STACKS\ }\hlkwa{ENDS}\\
\hllin{06\ }\hlstd{\\
\hllin{07\ }CODE\ }\hlkwa{SEGMENT}\\
\hllin{08\ }\hlstd{}\hlstd{\ \ \ \ }\hlstd{}\hlkwa{ASSUME\ }\hlstd{}\hlkwb{CS}\hlstd{}\hlopt{:}\hlstd{CODE}\hlopt{,}\hlstd{}\hlkwb{DS}\hlstd{}\hlopt{:}\hlstd{DATA}\hlopt{,}\hlstd{}\hlkwb{SS}\hlstd{}\hlopt{:}\hlstd{STACKS\\
\hllin{09\ }\ \\
\hllin{10\ }MAIN\ }\hlkwa{PROC\ FAR}\\
\hllin{11\ }\hlstd{}\hlkwc{START:}\\
\hllin{12\ }\hlstd{}\hlstd{\ \ \ \ }\hlstd{}\hlkwa{MOV\ }\hlstd{}\hlkwb{AX}\hlstd{}\hlopt{,}\hlstd{DATA\\
\hllin{13\ }}\hlstd{\ \ \ \ }\hlstd{}\hlkwa{MOV\ }\hlstd{}\hlkwb{DS}\hlstd{}\hlopt{,}\hlstd{}\hlkwb{AX}\\
\hllin{14\ }\hlstd{}\hlstd{\ \ \ \ }\hlstd{\\
\hllin{15\ }}\hlstd{\ \ \ \ }\hlstd{}\hlkwa{MOV\ }\hlstd{}\hlkwb{DX}\hlstd{}\hlopt{,\ }\hlstd{}\hlnum{283H}\\
\hllin{16\ }\hlstd{}\hlstd{\ \ \ \ }\hlstd{}\hlkwa{MOV\ }\hlstd{}\hlkwb{AL}\hlstd{}\hlopt{,\ }\hlstd{}\hlnum{10010000}\hlstd{B}\hlstd{\ \ \ \ }\hlstd{\\
\hllin{17\ }}\hlstd{\ \ \ \ }\hlstd{}\hlkwa{OUT\ }\hlstd{}\hlkwb{DX}\hlstd{}\hlopt{,\ }\hlstd{}\hlkwb{AL}\hlstd{\ \ \ \ \ \ \ \ \ }\hlkwb{}\\
\hllin{18\ }\hlstd{}\hlstd{\ \ \ \ }\hlstd{}\\
\hllin{19\ }\hlkwc{NEXT:}\hlstd{\ \ \ \ }\hlkwc{}\\
\hllin{20\ }\hlstd{}\hlstd{\ \ \ \ }\hlstd{}\hlkwa{MOV\ }\hlstd{}\hlkwb{DX}\hlstd{}\hlopt{,\ }\hlstd{}\hlnum{280H}\\
\hllin{21\ }\hlstd{}\hlstd{\ \ \ \ }\hlstd{}\hlkwa{IN\ }\hlstd{}\hlkwb{AL}\hlstd{}\hlopt{,\ }\hlstd{}\hlkwb{DX}\hlstd{\ \ \ \ \ \ \ \ \ \ }\hlkwb{}\\
\hllin{22\ }\hlstd{}\hlstd{\ \ \ \ }\hlstd{}\hlkwa{MOV\ }\hlstd{}\hlkwb{DX}\hlstd{}\hlopt{,\ }\hlstd{}\hlnum{282H}\\
\hllin{23\ }\hlstd{}\hlstd{\ \ \ \ }\hlstd{}\hlkwa{OUT\ }\hlstd{}\hlkwb{DX}\hlstd{}\hlopt{,\ }\hlstd{}\hlkwb{AL}\hlstd{\ \ \ \ \ \ \ \ \ }\hlkwb{}\\
\hllin{24\ }\hlstd{}\hlstd{\ \ \ \ }\hlstd{}\hlkwa{MOV\ }\hlstd{}\hlkwb{AH}\hlstd{}\hlopt{,\ }\hlstd{}\hlnum{1}\\
\hllin{25\ }\hlstd{}\hlstd{\ \ \ \ }\hlstd{}\hlkwa{INT\ }\hlstd{}\hlnum{16H}\hlstd{\ \ \ \ \ \ \ \ \ \ \ \ }\hlnum{}\\
\hllin{26\ }\hlstd{}\hlstd{\ \ \ \ }\hlstd{}\hlkwa{JZ\ }\hlstd{NEXT}\hlstd{\ \ \ \ \ \ \ \ \ \ \ \ }\hlstd{\\
\hllin{27\ }}\hlstd{\ \ \ \ }\hlstd{}\hlkwa{MOV\ }\hlstd{}\hlkwb{AH}\hlstd{}\hlopt{,\ }\hlstd{}\hlnum{0}\\
\hllin{28\ }\hlstd{}\hlstd{\ \ \ \ }\hlstd{}\hlkwa{INT\ }\hlstd{}\hlnum{16H}\hlstd{\ \ \ \ \ \ \ \ \ \ \ \ }\hlnum{}\\
\hllin{29\ }\hlstd{}\hlstd{\ \ \ \ }\hlstd{}\hlkwa{CMP\ }\hlstd{}\hlkwb{AL}\hlstd{}\hlopt{,\ }\hlstd{}\hlnum{20H}\\
\hllin{30\ }\hlstd{}\hlstd{\ \ \ \ }\hlstd{}\hlkwa{JZ\ }\hlstd{EXIT}\hlstd{\ \ \ \ \ \ \ \ \ \ \ \ }\hlstd{\\
\hllin{31\ }}\hlstd{\ \ \ \ }\hlstd{}\hlkwa{JMP\ }\hlstd{NEXT\\
\hllin{32\ }}\hlstd{\ \ \ \ \ \ \ \ \ \ \ \ \ \ \ \ \ }\hlstd{}\\
\hllin{33\ }\hlkwc{EXIT:}\hlstd{\ \ \ \ }\hlkwc{}\\
\hllin{34\ }\hlstd{}\hlstd{\ \ \ \ }\hlstd{}\hlkwa{MOV\ }\hlstd{}\hlkwb{AH}\hlstd{}\hlopt{,}\hlstd{}\hlnum{4}\hlstd{}\hlkwb{CH}\\
\hllin{35\ }\hlstd{}\hlstd{\ \ \ \ }\hlstd{}\hlkwa{INT\ }\hlstd{}\hlnum{21H}\\
\hllin{36\ }\hlstd{\\
\hllin{37\ }MAIN\ }\hlkwa{ENDP}\\
\hllin{38\ }\hlstd{CODE\ }\hlkwa{ENDS}\\
\hllin{39\ }\hlstd{}\hlstd{\ \ \ \ }\hlstd{}\hlkwa{END\ }\hlstd{START}\\
\mbox{}
\normalfont
\normalsize

\subsubsection{选做任务}
A口仍保持方式0输入开关状态,C口仍以方式0输出LED灯显
示,满足要求的条件,代码如下所示:
\noindent
\ttfamily
\hlstd{\hllin{01\ }DATA\ }\hlkwa{SEGMENT}\\
\hllin{02\ }\hlstd{DATA\ }\hlkwa{ENDS}\\
\hllin{03\ }\hlstd{\\
\hllin{04\ }STACKS\ }\hlkwa{SEGMENT}\\
\hllin{05\ }\hlstd{STACKS\ }\hlkwa{ENDS}\\
\hllin{06\ }\hlstd{\\
\hllin{07\ }CODE\ }\hlkwa{SEGMENT}\\
\hllin{08\ }\hlstd{}\hlstd{\ \ \ \ }\hlstd{}\hlkwa{ASSUME\ }\hlstd{}\hlkwb{CS}\hlstd{}\hlopt{:}\hlstd{CODE}\hlopt{,}\hlstd{}\hlkwb{DS}\hlstd{}\hlopt{:}\hlstd{DATA}\hlopt{,}\hlstd{}\hlkwb{SS}\hlstd{}\hlopt{:}\hlstd{STACKS\\
\hllin{09\ }}\hlstd{\ \ \ \ }\hlstd{\\
\hllin{10\ }MAIN\ }\hlkwa{PROC\ FAR}\\
\hllin{11\ }\hlstd{}\hlkwc{START:}\\
\hllin{12\ }\hlstd{}\hlstd{\ \ \ \ }\hlstd{}\hlkwa{MOV\ }\hlstd{}\hlkwb{AX}\hlstd{}\hlopt{,}\hlstd{DATA\\
\hllin{13\ }}\hlstd{\ \ \ \ }\hlstd{}\hlkwa{MOV\ }\hlstd{}\hlkwb{DS}\hlstd{}\hlopt{,}\hlstd{}\hlkwb{AX}\\
\hllin{14\ }\hlstd{}\hlstd{\ \ \ \ }\hlstd{\\
\hllin{15\ }}\hlstd{\ \ \ \ }\hlstd{}\hlkwa{MOV\ }\hlstd{}\hlkwb{DX}\hlstd{}\hlopt{,\ }\hlstd{}\hlnum{283H}\\
\hllin{16\ }\hlstd{}\hlstd{\ \ \ \ }\hlstd{}\hlkwa{MOV\ }\hlstd{}\hlkwb{AL}\hlstd{}\hlopt{,\ }\hlstd{}\hlnum{10010000}\hlstd{B\\
\hllin{17\ }}\hlstd{\ \ \ \ }\hlstd{}\hlkwa{OUT\ }\hlstd{}\hlkwb{DX}\hlstd{}\hlopt{,\ }\hlstd{}\hlkwb{AL}\hlstd{\ \ \ }\hlkwb{}\\
\hllin{18\ }\hlstd{}\hlkwc{SCAN:}\hlstd{\ \ \ \ }\hlkwc{}\\
\hllin{19\ }\hlstd{}\hlstd{\ \ \ \ }\hlstd{}\hlkwa{MOV\ }\hlstd{}\hlkwb{DX}\hlstd{}\hlopt{,\ }\hlstd{}\hlnum{280H}\\
\hllin{20\ }\hlstd{}\hlstd{\ \ \ \ }\hlstd{}\hlkwa{IN\ }\hlstd{}\hlkwb{AL}\hlstd{}\hlopt{,\ }\hlstd{}\hlkwb{DX}\hlstd{\ \ \ \ }\hlkwb{}\\
\hllin{21\ }\hlstd{}\hlstd{\ \ \ \ }\hlstd{\\
\hllin{22\ }}\hlstd{\ \ \ \ }\hlstd{}\hlkwa{CMP\ }\hlstd{}\hlkwb{AL}\hlstd{}\hlopt{,\ }\hlstd{}\hlnum{11000000}\hlstd{B\\
\hllin{23\ }}\hlstd{\ \ \ \ }\hlstd{}\hlkwa{JZ\ }\hlstd{TOFLASH}\hlstd{\ \ \ }\hlstd{\\
\hllin{24\ }}\hlstd{\ \ \ \ }\hlstd{\\
\hllin{25\ }}\hlstd{\ \ \ \ }\hlstd{}\hlkwa{CMP\ }\hlstd{}\hlkwb{AL}\hlstd{}\hlopt{,\ }\hlstd{}\hlnum{10000000}\hlstd{B\\
\hllin{26\ }}\hlstd{\ \ \ \ }\hlstd{}\hlkwa{JZ\ }\hlstd{L\textunderscore MOVE}\hlstd{\ \ \ \ }\hlstd{\\
\hllin{27\ }}\hlstd{\ \ \ \ }\hlstd{\\
\hllin{28\ }}\hlstd{\ \ \ \ }\hlstd{}\hlkwa{CMP\ }\hlstd{}\hlkwb{AL}\hlstd{}\hlopt{,\ }\hlstd{}\hlnum{01000000}\hlstd{B\\
\hllin{29\ }}\hlstd{\ \ \ \ }\hlstd{}\hlkwa{JZ\ }\hlstd{R\textunderscore MOVE}\hlstd{\ \ \ \ }\hlstd{\\
\hllin{30\ }}\hlstd{\ \ \ \ }\hlstd{\\
\hllin{31\ }}\hlstd{\ \ \ \ }\hlstd{}\hlkwa{MOV\ }\hlstd{}\hlkwb{DX}\hlstd{}\hlopt{,\ }\hlstd{}\hlnum{282H}\\
\hllin{32\ }\hlstd{}\hlstd{\ \ \ \ }\hlstd{}\hlkwa{OUT\ }\hlstd{}\hlkwb{DX}\hlstd{}\hlopt{,\ }\hlstd{}\hlkwb{AL}\hlstd{\ \ \ }\hlkwb{}\\
\hllin{33\ }\hlstd{}\hlstd{\ \ \ \ }\hlstd{\\
\hllin{34\ }}\hlstd{\ \ \ \ }\hlstd{}\hlkwa{MOV\ }\hlstd{}\hlkwb{AH}\hlstd{}\hlopt{,\ }\hlstd{}\hlnum{1}\\
\hllin{35\ }\hlstd{}\hlstd{\ \ \ \ }\hlstd{}\hlkwa{INT\ }\hlstd{}\hlnum{16H}\hlstd{\ \ \ \ \ \ }\hlnum{}\\
\hllin{36\ }\hlstd{}\hlstd{\ \ \ \ }\hlstd{}\hlkwa{JZ\ }\hlstd{SCAN}\hlstd{\ \ \ \ \ \ }\hlstd{\\
\hllin{37\ }}\hlstd{\ \ \ \ }\hlstd{\\
\hllin{38\ }}\hlstd{\ \ \ \ }\hlstd{}\hlkwa{MOV\ }\hlstd{}\hlkwb{AH}\hlstd{}\hlopt{,\ }\hlstd{}\hlnum{0}\\
\hllin{39\ }\hlstd{}\hlstd{\ \ \ \ }\hlstd{}\hlkwa{INT\ }\hlstd{}\hlnum{16H}\hlstd{\ \ \ \ \ \ }\hlnum{}\\
\hllin{40\ }\hlstd{}\hlstd{\ \ \ \ }\hlstd{}\hlkwa{CMP\ }\hlstd{}\hlkwb{AL}\hlstd{}\hlopt{,\ }\hlstd{}\hlnum{20H}\\
\hllin{41\ }\hlstd{}\hlstd{\ \ \ \ }\hlstd{}\hlkwa{JZ\ }\hlstd{TOEXIT}\hlstd{\ \ \ \ }\hlstd{\\
\hllin{42\ }}\hlstd{\ \ \ \ }\hlstd{}\hlkwa{JMP\ }\hlstd{SCAN\ \\
\hllin{43\ }}\hlstd{\ \ \ \ \ \ \ \ }\hlstd{}\\
\hllin{44\ }\hlkwc{L\textunderscore MOVE:}\hlstd{\ \ \ \ \ \ \ \ \ \ }\hlkwc{}\\
\hllin{45\ }\hlstd{}\hlstd{\ \ \ \ }\hlstd{}\hlkwa{MOV\ }\hlstd{}\hlkwb{DX}\hlstd{}\hlopt{,\ }\hlstd{}\hlnum{282H}\\
\hllin{46\ }\hlstd{}\hlstd{\ \ \ \ }\hlstd{}\hlkwa{MOV\ }\hlstd{}\hlkwb{BL}\hlstd{}\hlopt{,\ }\hlstd{}\hlnum{10000000}\hlstd{B\\
\hllin{47\ }}\hlstd{\ \ \ \ }\hlstd{}\hlkwa{MOV\ }\hlstd{}\hlkwb{AL}\hlstd{}\hlopt{,\ }\hlstd{}\hlkwb{BL}\\
\hllin{48\ }\hlstd{}\hlstd{\ \ \ \ }\hlstd{}\hlkwa{OUT\ }\hlstd{}\hlkwb{DX}\hlstd{}\hlopt{,\ }\hlstd{}\hlkwb{AL}\hlstd{\ \ \ }\hlkwb{}\\
\hllin{49\ }\hlstd{}\hlkwc{R\textunderscore LEFT:}\hlstd{\ \ \ \ }\hlkwc{}\\
\hllin{50\ }\hlstd{}\hlstd{\ \ \ \ }\hlstd{}\hlkwa{ROL\ }\hlstd{}\hlkwb{BL}\hlstd{}\hlopt{,\ }\hlstd{}\hlnum{1}\hlstd{\ \ \ \ }\hlnum{}\\
\hllin{51\ }\hlstd{}\hlstd{\ \ \ \ }\hlstd{}\hlkwa{MOV\ }\hlstd{}\hlkwb{AL}\hlstd{}\hlopt{,\ }\hlstd{}\hlkwb{BL}\\
\hllin{52\ }\hlstd{}\hlstd{\ \ \ \ }\hlstd{}\hlkwa{MOV\ }\hlstd{}\hlkwb{DX}\hlstd{}\hlopt{,\ }\hlstd{}\hlnum{282H}\\
\hllin{53\ }\hlstd{}\hlstd{\ \ \ \ }\hlstd{}\hlkwa{OUT\ }\hlstd{}\hlkwb{DX}\hlstd{}\hlopt{,\ }\hlstd{}\hlkwb{AL}\hlstd{\ \ \ }\hlkwb{}\\
\hllin{54\ }\hlstd{}\hlstd{\ \ \ \ }\hlstd{\\
\hllin{55\ }}\hlstd{\ \ \ \ }\hlstd{}\hlkwa{CALL\ }\hlstd{DELAY}\hlstd{\ \ \ }\hlstd{\\
\hllin{56\ }}\hlstd{\ \ \ \ }\hlstd{\\
\hllin{57\ }}\hlstd{\ \ \ \ }\hlstd{}\hlkwa{MOV\ }\hlstd{}\hlkwb{DX}\hlstd{}\hlopt{,\ }\hlstd{}\hlnum{280H}\\
\hllin{58\ }\hlstd{}\hlstd{\ \ \ \ }\hlstd{}\hlkwa{IN\ }\hlstd{}\hlkwb{AL}\hlstd{}\hlopt{,\ }\hlstd{}\hlkwb{DX}\hlstd{\ \ \ \ }\hlkwb{}\\
\hllin{59\ }\hlstd{}\hlstd{\ \ \ \ }\hlstd{}\hlkwa{CMP\ }\hlstd{}\hlkwb{AL}\hlstd{}\hlopt{,\ }\hlstd{}\hlnum{10000000}\hlstd{B\\
\hllin{60\ }}\hlstd{\ \ \ \ }\hlstd{}\hlkwa{JNZ\ }\hlstd{SCAN}\hlstd{\ \ \ \ \ \ \ \ \ \ }\hlstd{\\
\hllin{61\ }}\hlstd{\ \ \ \ }\hlstd{\\
\hllin{62\ }}\hlstd{\ \ \ \ }\hlstd{}\hlkwa{MOV\ }\hlstd{}\hlkwb{DX}\hlstd{}\hlopt{,\ }\hlstd{}\hlnum{282H}\\
\hllin{63\ }\hlstd{}\hlstd{\ \ \ \ }\hlstd{}\hlkwa{OUT\ }\hlstd{}\hlkwb{DX}\hlstd{}\hlopt{,\ }\hlstd{}\hlkwb{AL}\hlstd{\ \ \ \ \ \ \ \ }\hlkwb{}\\
\hllin{64\ }\hlstd{}\hlstd{\ \ \ \ }\hlstd{}\hlkwa{MOV\ }\hlstd{}\hlkwb{AH}\hlstd{}\hlopt{,\ }\hlstd{}\hlnum{1}\\
\hllin{65\ }\hlstd{}\hlstd{\ \ \ \ }\hlstd{}\hlkwa{INT\ }\hlstd{}\hlnum{16H}\hlstd{\ \ \ \ \ \ \ \ \ \ \ }\hlnum{}\\
\hllin{66\ }\hlstd{}\hlstd{\ \ \ \ }\hlstd{}\hlkwa{JZ\ }\hlstd{R\textunderscore LEFT}\hlstd{\ \ \ \ \ \ \ \ \ }\hlstd{\\
\hllin{67\ }}\hlstd{\ \ \ \ }\hlstd{\\
\hllin{68\ }}\hlstd{\ \ \ \ }\hlstd{}\hlkwa{MOV\ }\hlstd{}\hlkwb{AH}\hlstd{}\hlopt{,\ }\hlstd{}\hlnum{0}\\
\hllin{69\ }\hlstd{}\hlstd{\ \ \ \ }\hlstd{}\hlkwa{INT\ }\hlstd{}\hlnum{16H}\\
\hllin{70\ }\hlstd{}\hlstd{\ \ \ \ }\hlstd{}\hlkwa{CMP\ }\hlstd{}\hlkwb{AL}\hlstd{}\hlopt{,\ }\hlstd{}\hlnum{20H}\hlstd{\ \ \ \ \ \ \ }\hlnum{}\\
\hllin{71\ }\hlstd{}\hlstd{\ \ \ \ }\hlstd{}\hlkwa{JZ\ }\hlstd{EXIT\\
\hllin{72\ }}\hlstd{\ \ \ \ }\hlstd{\\
\hllin{73\ }}\hlstd{\ \ \ \ }\hlstd{}\hlkwa{JMP\ }\hlstd{R\textunderscore LEFT}\\
\hllin{74\ }\\
\hllin{75\ }\hlkwc{TOFLASH:}\hlstd{\ \ \ \ \ \ \ \ \ \ \ \ \ \ }\hlkwc{}\\
\hllin{76\ }\hlstd{}\hlstd{\ \ \ \ }\hlstd{}\hlkwa{JMP\ }\hlstd{FLASH}\\
\hllin{77\ }\hlkwc{TOEXIT:}\\
\hllin{78\ }\hlstd{}\hlstd{\ \ \ \ }\hlstd{}\hlkwa{JMP\ }\hlstd{EXIT\\
\hllin{79\ }}\hlstd{\ \ \ \ \ \ }\hlstd{}\\
\hllin{80\ }\hlkwc{R\textunderscore MOVE:}\hlstd{\ \ \ \ \ \ \ \ \ \ \ \ \ \ \ }\hlkwc{}\\
\hllin{81\ }\hlstd{}\hlstd{\ \ \ \ }\hlstd{}\hlkwa{MOV\ }\hlstd{}\hlkwb{DX}\hlstd{}\hlopt{,\ }\hlstd{}\hlnum{282H}\\
\hllin{82\ }\hlstd{}\hlstd{\ \ \ \ }\hlstd{}\hlkwa{MOV\ }\hlstd{}\hlkwb{BL}\hlstd{}\hlopt{,\ }\hlstd{}\hlnum{10000000}\hlstd{B}\hlstd{\ \ \ \ \ \ \ \ }\hlstd{\\
\hllin{83\ }}\hlstd{\ \ \ \ }\hlstd{}\hlkwa{MOV\ }\hlstd{}\hlkwb{AL}\hlstd{}\hlopt{,\ }\hlstd{}\hlkwb{BL}\\
\hllin{84\ }\hlstd{}\hlstd{\ \ \ \ }\hlstd{}\hlkwa{OUT\ }\hlstd{}\hlkwb{DX}\hlstd{}\hlopt{,\ }\hlstd{}\hlkwb{AL}\hlstd{\ \ \ \ \ \ \ \ }\hlkwb{}\\
\hllin{85\ }\hlstd{}\hlstd{\ \ \ \ }\hlstd{}\\
\hllin{86\ }\hlkwc{R\textunderscore RIGHT:\ }\\
\hllin{87\ }\hlstd{}\hlstd{\ \ \ \ }\hlstd{}\hlkwa{ROR\ }\hlstd{}\hlkwb{BL}\hlstd{}\hlopt{,\ }\hlstd{}\hlnum{1}\hlstd{\ \ \ \ \ \ \ \ \ }\hlnum{}\\
\hllin{88\ }\hlstd{}\hlstd{\ \ \ \ }\hlstd{}\hlkwa{MOV\ }\hlstd{}\hlkwb{AL}\hlstd{}\hlopt{,\ }\hlstd{}\hlkwb{BL}\\
\hllin{89\ }\hlstd{}\hlstd{\ \ \ \ }\hlstd{}\hlkwa{MOV\ }\hlstd{}\hlkwb{DX}\hlstd{}\hlopt{,\ }\hlstd{}\hlnum{282H}\\
\hllin{90\ }\hlstd{}\hlstd{\ \ \ \ }\hlstd{\\
\hllin{91\ }}\hlstd{\ \ \ \ }\hlstd{}\hlkwa{CALL\ }\hlstd{DELAY}\hlstd{\ \ \ \ \ \ \ \ \ }\hlstd{\\
\hllin{92\ }}\hlstd{\ \ \ \ }\hlstd{\\
\hllin{93\ }}\hlstd{\ \ \ \ }\hlstd{}\hlkwa{MOV\ }\hlstd{}\hlkwb{DX}\hlstd{}\hlopt{,\ }\hlstd{}\hlnum{280H}\\
\hllin{94\ }\hlstd{}\hlstd{\ \ \ \ }\hlstd{}\hlkwa{IN\ }\hlstd{}\hlkwb{AL}\hlstd{}\hlopt{,\ }\hlstd{}\hlkwb{DX}\hlstd{\ \ \ \ \ \ \ \ \ \ }\hlkwb{}\\
\hllin{95\ }\hlstd{}\hlstd{\ \ \ \ }\hlstd{}\hlkwa{CMP\ }\hlstd{}\hlkwb{AL}\hlstd{}\hlopt{,\ }\hlstd{}\hlnum{01000000}\hlstd{B}\hlstd{\ \ \ \ \ \ \ \ }\hlstd{\\
\hllin{96\ }}\hlstd{\ \ \ \ }\hlstd{}\hlkwa{JNZ\ }\hlstd{SCAN}\hlstd{\ \ \ \ \ \ \ \ \ \ \ }\hlstd{\\
\hllin{97\ }}\hlstd{\ \ \ \ }\hlstd{\\
\hllin{98\ }}\hlstd{\ \ \ \ }\hlstd{}\hlkwa{MOV\ }\hlstd{}\hlkwb{DX}\hlstd{}\hlopt{,\ }\hlstd{}\hlnum{282H}\\
\hllin{99\ }\hlstd{}\hlstd{\ \ \ \ }\hlstd{}\hlkwa{OUT\ }\hlstd{}\hlkwb{DX}\hlstd{}\hlopt{,\ }\hlstd{}\hlkwb{AL}\hlstd{\ \ \ \ \ \ \ \ \ }\hlkwb{}\\
\hllin{100\ }\hlstd{}\hlstd{\ \ \ \ }\hlstd{}\hlkwa{MOV\ }\hlstd{}\hlkwb{AH}\hlstd{}\hlopt{,\ }\hlstd{}\hlnum{1}\hlstd{\ \ \ \ \ \ \ \ }\hlnum{}\\
\hllin{101\ }\hlstd{}\hlstd{\ \ \ \ }\hlstd{}\hlkwa{INT\ }\hlstd{}\hlnum{16H}\hlstd{\ \ \ \ \ \ \ \ \ \ \ \ }\hlnum{}\\
\hllin{102\ }\hlstd{}\hlstd{\ \ \ \ }\hlstd{}\hlkwa{JZ\ }\hlstd{R\textunderscore RIGHT}\hlstd{\ \ \ \ \ \ \ \ \ }\hlstd{\\
\hllin{103\ }}\hlstd{\ \ \ \ }\hlstd{}\hlkwa{MOV\ }\hlstd{}\hlkwb{AH}\hlstd{}\hlopt{,\ }\hlstd{}\hlnum{0}\\
\hllin{104\ }\hlstd{}\hlstd{\ \ \ \ }\hlstd{}\hlkwa{INT\ }\hlstd{}\hlnum{16H}\\
\hllin{105\ }\hlstd{}\hlstd{\ \ \ \ }\hlstd{}\hlkwa{CMP\ }\hlstd{}\hlkwb{AL}\hlstd{}\hlopt{,\ }\hlstd{}\hlnum{20H}\hlstd{\ \ \ \ \ \ \ \ }\hlnum{}\\
\hllin{106\ }\hlstd{}\hlstd{\ \ \ \ }\hlstd{}\hlkwa{JZ\ }\hlstd{EXIT}\hlstd{\ \ \ \ \ \ \ \ \ \ \ \ }\hlstd{\\
\hllin{107\ }}\hlstd{\ \ \ \ }\hlstd{\\
\hllin{108\ }}\hlstd{\ \ \ \ }\hlstd{}\hlkwa{JMP\ }\hlstd{R\textunderscore RIGHT}\\
\hllin{109\ }\\
\hllin{110\ }\hlkwc{TOSCAN:}\\
\hllin{111\ }\hlstd{}\hlstd{\ \ \ \ }\hlstd{}\hlkwa{JMP\ }\hlstd{SCAN}\\
\hllin{112\ }\hlkwc{FLASH:}\hlstd{\ \ \ \ \ \ \ \ \ \ \ \ \ \ \ \ \ }\hlkwc{}\\
\hllin{113\ }\hlstd{}\hlstd{\ \ \ \ }\hlstd{}\hlkwa{MOV\ }\hlstd{}\hlkwb{DX}\hlstd{}\hlopt{,\ }\hlstd{}\hlnum{282H}\\
\hllin{114\ }\hlstd{}\hlstd{\ \ \ \ }\hlstd{}\hlkwa{MOV\ }\hlstd{}\hlkwb{AL}\hlstd{}\hlopt{,\ }\hlstd{}\hlnum{0}\\
\hllin{115\ }\hlstd{}\hlstd{\ \ \ \ }\hlstd{}\hlkwa{OUT\ }\hlstd{}\hlkwb{DX}\hlstd{}\hlopt{,\ }\hlstd{}\hlkwb{AL}\hlstd{\ \ \ \ \ \ \ \ \ }\hlkwb{}\\
\hllin{116\ }\hlstd{}\hlstd{\ \ \ \ }\hlstd{\\
\hllin{117\ }}\hlstd{\ \ \ \ }\hlstd{}\hlkwa{CALL\ }\hlstd{DELAY\\
\hllin{118\ }}\hlstd{\ \ \ \ }\hlstd{\\
\hllin{119\ }}\hlstd{\ \ \ \ }\hlstd{}\hlkwa{MOV\ }\hlstd{}\hlkwb{DX}\hlstd{}\hlopt{,\ }\hlstd{}\hlnum{282H}\\
\hllin{120\ }\hlstd{}\hlstd{\ \ \ \ }\hlstd{}\hlkwa{MOV\ }\hlstd{}\hlkwb{AL}\hlstd{}\hlopt{,\ }\hlstd{}\hlnum{0}\hlstd{FFH\\
\hllin{121\ }}\hlstd{\ \ \ \ }\hlstd{}\hlkwa{OUT\ }\hlstd{}\hlkwb{DX}\hlstd{}\hlopt{,\ }\hlstd{}\hlkwb{AL}\hlstd{\ \ \ \ \ \ \ \ \ }\hlkwb{}\\
\hllin{122\ }\hlstd{\\
\hllin{123\ }}\hlstd{\ \ \ \ }\hlstd{}\hlkwa{MOV\ }\hlstd{}\hlkwb{DX}\hlstd{}\hlopt{,\ }\hlstd{}\hlnum{280H}\\
\hllin{124\ }\hlstd{}\hlstd{\ \ \ \ }\hlstd{}\hlkwa{IN\ }\hlstd{}\hlkwb{AL}\hlstd{}\hlopt{,\ }\hlstd{}\hlkwb{DX}\\
\hllin{125\ }\hlstd{}\hlstd{\ \ \ \ }\hlstd{}\hlkwa{CMP\ }\hlstd{}\hlkwb{AL}\hlstd{}\hlopt{,\ }\hlstd{}\hlnum{11000000}\hlstd{B\\
\hllin{126\ }}\hlstd{\ \ \ \ }\hlstd{}\hlkwa{JNZ\ }\hlstd{TOSCAN}\hlstd{\ \ \ \ \ \ \ \ \ }\hlstd{\\
\hllin{127\ }}\hlstd{\ \ \ \ }\hlstd{\\
\hllin{128\ }}\hlstd{\ \ \ \ }\hlstd{}\hlkwa{CALL\ }\hlstd{DELAY\\
\hllin{129\ }}\hlstd{\ \ \ \ }\hlstd{\\
\hllin{130\ }}\hlstd{\ \ \ \ }\hlstd{}\hlkwa{MOV\ }\hlstd{}\hlkwb{DX}\hlstd{}\hlopt{,\ }\hlstd{}\hlnum{282H}\\
\hllin{131\ }\hlstd{}\hlstd{\ \ \ \ }\hlstd{}\hlkwa{OUT\ }\hlstd{}\hlkwb{DX}\hlstd{}\hlopt{,\ }\hlstd{}\hlkwb{AL}\hlstd{\ \ \ \ \ \ \ \ \ }\hlkwb{}\\
\hllin{132\ }\hlstd{}\hlstd{\ \ \ \ }\hlstd{}\hlkwa{MOV\ }\hlstd{}\hlkwb{AH}\hlstd{}\hlopt{,\ }\hlstd{}\hlnum{1}\\
\hllin{133\ }\hlstd{}\hlstd{\ \ \ \ }\hlstd{}\hlkwa{INT\ }\hlstd{}\hlnum{16H}\hlstd{\ \ \ \ \ \ \ \ \ \ \ \ \ }\hlnum{}\\
\hllin{134\ }\hlstd{}\hlstd{\ \ \ \ }\hlstd{}\hlkwa{JZ\ }\hlstd{FLASH}\hlstd{\ \ \ \ \ \ \ \ \ \ \ \ }\hlstd{\\
\hllin{135\ }}\hlstd{\ \ \ \ }\hlstd{}\hlkwa{MOV\ }\hlstd{}\hlkwb{AH}\hlstd{}\hlopt{,\ }\hlstd{}\hlnum{0}\\
\hllin{136\ }\hlstd{}\hlstd{\ \ \ \ }\hlstd{}\hlkwa{INT\ }\hlstd{}\hlnum{16H}\\
\hllin{137\ }\hlstd{}\hlstd{\ \ \ \ }\hlstd{}\hlkwa{CMP\ }\hlstd{}\hlkwb{AL}\hlstd{}\hlopt{,\ }\hlstd{}\hlnum{20H}\hlstd{\ \ \ \ \ \ \ \ \ }\hlnum{}\\
\hllin{138\ }\hlstd{}\hlstd{\ \ \ \ }\hlstd{}\hlkwa{JZ\ }\hlstd{EXIT}\hlstd{\ \ \ \ \ \ \ \ \ \ \ \ \ }\hlstd{\\
\hllin{139\ }}\hlstd{\ \ \ \ }\hlstd{\\
\hllin{140\ }}\hlstd{\ \ \ \ }\hlstd{}\hlkwa{JMP\ }\hlstd{FLASH\\
\hllin{141\ }}\hlstd{\ \ \ \ \ \ \ \ \ \ \ \ \ \ \ \ \ \ \ \ \ }\hlstd{}\\
\hllin{142\ }\hlkwc{EXIT:}\hlstd{\ \ \ \ \ \ \ \ \ \ \ \ \ \ \ \ \ \ \ }\hlkwc{}\\
\hllin{143\ }\hlstd{}\hlstd{\ \ \ \ }\hlstd{}\hlkwa{MOV\ }\hlstd{}\hlkwb{AH}\hlstd{}\hlopt{,}\hlstd{}\hlnum{4}\hlstd{}\hlkwb{CH}\\
\hllin{144\ }\hlstd{}\hlstd{\ \ \ \ }\hlstd{}\hlkwa{INT\ }\hlstd{}\hlnum{21H}\\
\hllin{145\ }\hlstd{}\hlstd{\ \ \ \ }\hlstd{\\
\hllin{146\ }DELAY\ }\hlkwa{PROC}\hlstd{\ \ \ \ \ \ \ \ \ \ \ \ \ \ }\hlkwa{}\\
\hllin{147\ }\hlstd{}\hlstd{\ \ \ \ }\hlstd{}\hlkwa{PUSH\ }\hlstd{}\hlkwb{BX}\\
\hllin{148\ }\hlstd{}\hlstd{\ \ \ \ }\hlstd{}\hlkwa{PUSH\ }\hlstd{}\hlkwb{CX}\\
\hllin{149\ }\hlstd{}\hlstd{\ \ \ \ }\hlstd{}\hlkwa{MOV\ }\hlstd{}\hlkwb{BX}\hlstd{}\hlopt{,\ }\hlstd{}\hlnum{0}\hlstd{FH}\\
\hllin{150\ }\hlkwc{WAITB:}\\
\hllin{151\ }\hlstd{}\hlstd{\ \ \ \ }\hlstd{}\hlkwa{MOV\ }\hlstd{}\hlkwb{CX}\hlstd{}\hlopt{,\ }\hlstd{}\hlnum{0}\hlstd{FFFFH}\\
\hllin{152\ }\hlkwc{WAITC:}\\
\hllin{153\ }\hlstd{}\hlstd{\ \ \ \ }\hlstd{}\hlkwa{DEC\ }\hlstd{}\hlkwb{CX}\\
\hllin{154\ }\hlstd{}\hlstd{\ \ \ \ }\hlstd{}\hlkwa{JNZ\ }\hlstd{WAITC}\hlstd{\ \ \ \ \ \ \ \ \ \ \ }\hlstd{\\
\hllin{155\ }}\hlstd{\ \ \ \ }\hlstd{}\hlkwa{DEC\ }\hlstd{}\hlkwb{BX}\\
\hllin{156\ }\hlstd{}\hlstd{\ \ \ \ }\hlstd{}\hlkwa{JNZ\ }\hlstd{WAITB}\hlstd{\ \ \ \ \ \ \ \ \ \ \ }\hlstd{\\
\hllin{157\ }}\hlstd{\ \ \ \ }\hlstd{}\hlkwa{POP\ }\hlstd{}\hlkwb{CX}\\
\hllin{158\ }\hlstd{}\hlstd{\ \ \ \ }\hlstd{}\hlkwa{POP\ }\hlstd{}\hlkwb{BX}\\
\hllin{159\ }\hlstd{}\hlstd{\ \ \ \ }\hlstd{}\hlkwa{RET}\\
\hllin{160\ }\hlstd{DELAY\ }\hlkwa{ENDP}\\
\hllin{161\ }\hlstd{\\
\hllin{162\ }MAIN\ }\hlkwa{ENDP}\\
\hllin{163\ }\hlstd{CODE\ }\hlkwa{ENDS}\\
\hllin{164\ }\hlstd{}\hlstd{\ \ \ \ }\hlstd{}\hlkwa{END\ }\hlstd{START}\\
\mbox{}
\normalfont
\normalsize

\subsection{任务二}
实验中每按一次单脉冲按键,通过8255电路发一次中断请
求。CRT上显示一个A口的ASCII码字符,直到A口数据为FFH
退出。
\subsubsection{必做任务}
代码如下所示:
\noindent
\ttfamily
\hlstd{\hllin{01\ }DATA\ }\hlkwa{SEGMENT}\\
\hllin{02\ }\hlstd{KEEP\textunderscore IP\ }\hlkwa{DW\ }\hlstd{}\hlnum{0}\\
\hllin{03\ }\hlstd{KEEP\textunderscore CS\ }\hlkwa{DW\ }\hlstd{}\hlnum{0}\hlstd{\ \ }\hlnum{}\\
\hllin{04\ }\hlstd{FLAG\ }\hlkwa{DB\ }\hlstd{}\hlnum{0}\hlstd{\ \ \ \ \ \ }\hlnum{}\\
\hllin{05\ }\hlstd{DATA\ }\hlkwa{ENDS}\\
\hllin{06\ }\hlstd{\\
\hllin{07\ }STACKS\ }\hlkwa{SEGMENT}\\
\hllin{08\ }\hlstd{STACKS\ }\hlkwa{ENDS}\\
\hllin{09\ }\hlstd{\\
\hllin{10\ }CODE\ }\hlkwa{SEGMENT}\\
\hllin{11\ }\hlstd{}\hlstd{\ \ \ \ }\hlstd{}\hlkwa{ASSUME\ }\hlstd{}\hlkwb{CS}\hlstd{}\hlopt{:}\hlstd{CODE}\hlopt{,}\hlstd{}\hlkwb{DS}\hlstd{}\hlopt{:}\hlstd{DATA}\hlopt{,}\hlstd{}\hlkwb{SS}\hlstd{}\hlopt{:}\hlstd{STACKS\\
\hllin{12\ }\\
\hllin{13\ }MAIN\ }\hlkwa{PROC\ FAR}\\
\hllin{14\ }\hlstd{}\hlkwc{START:}\\
\hllin{15\ }\hlstd{}\hlstd{\ \ \ \ }\hlstd{}\hlkwa{MOV\ }\hlstd{}\hlkwb{AX}\hlstd{}\hlopt{,}\hlstd{DATA\\
\hllin{16\ }}\hlstd{\ \ \ \ }\hlstd{}\hlkwa{MOV\ }\hlstd{}\hlkwb{DS}\hlstd{}\hlopt{,}\hlstd{}\hlkwb{AX}\\
\hllin{17\ }\hlstd{\\
\hllin{18\ }}\hlstd{\ \ \ \ }\hlstd{\\
\hllin{19\ }}\hlstd{\ \ \ \ }\hlstd{}\hlkwa{MOV\ }\hlstd{}\hlkwb{DX}\hlstd{}\hlopt{,\ }\hlstd{}\hlnum{283H}\\
\hllin{20\ }\hlstd{}\hlstd{\ \ \ \ }\hlstd{}\hlkwa{MOV\ }\hlstd{}\hlkwb{AL}\hlstd{}\hlopt{,\ }\hlstd{}\hlnum{10110000}\hlstd{B\\
\hllin{21\ }}\hlstd{\ \ \ \ }\hlstd{}\hlkwa{OUT\ }\hlstd{}\hlkwb{DX}\hlstd{}\hlopt{,\ }\hlstd{}\hlkwb{AL}\\
\hllin{22\ }\hlstd{\\
\hllin{23\ }}\hlstd{\ \ \ \ }\hlstd{}\hlkwa{MOV\ }\hlstd{}\hlkwb{DX}\hlstd{}\hlopt{,\ }\hlstd{}\hlnum{283H}\\
\hllin{24\ }\hlstd{}\hlstd{\ \ \ \ }\hlstd{}\hlkwa{MOV\ }\hlstd{}\hlkwb{AL}\hlstd{}\hlopt{,\ }\hlstd{}\hlnum{00001001}\hlstd{B\\
\hllin{25\ }}\hlstd{\ \ \ \ }\hlstd{}\hlkwa{OUT\ }\hlstd{}\hlkwb{DX}\hlstd{}\hlopt{,\ }\hlstd{}\hlkwb{AL}\hlstd{\ \ \ \ \ \ \ }\hlkwb{}\\
\hllin{26\ }\hlstd{}\hlstd{\ \ \ \ }\hlstd{\\
\hllin{27\ }}\hlstd{\ \ \ \ }\hlstd{\\
\hllin{28\ }}\hlstd{\ \ \ \ }\hlstd{}\hlkwa{MOV\ }\hlstd{}\hlkwb{AH}\hlstd{}\hlopt{,\ }\hlstd{}\hlnum{35H}\hlstd{\ \ \ \ \ \ }\hlnum{}\\
\hllin{29\ }\hlstd{}\hlstd{\ \ \ \ }\hlstd{}\hlkwa{MOV\ }\hlstd{}\hlkwb{AL}\hlstd{}\hlopt{,\ }\hlstd{}\hlnum{0}\hlstd{}\hlkwb{BH\ }\\
\hllin{30\ }\hlstd{}\hlstd{\ \ \ \ }\hlstd{}\hlkwa{INT\ }\hlstd{}\hlnum{21H}\\
\hllin{31\ }\hlstd{}\hlstd{\ \ \ \ }\hlstd{}\hlkwa{MOV\ }\hlstd{KEEP\textunderscore IP}\hlopt{,\ }\hlstd{}\hlkwb{BX}\hlstd{\ \ }\hlkwb{}\\
\hllin{32\ }\hlstd{}\hlstd{\ \ \ \ }\hlstd{}\hlkwa{MOV\ }\hlstd{KEEP\textunderscore CS}\hlopt{,\ }\hlstd{}\hlkwb{ES}\hlstd{\ \ }\hlkwb{}\\
\hllin{33\ }\hlstd{}\hlstd{\ \ \ \ }\hlstd{\\
\hllin{34\ }}\hlstd{\ \ \ }\hlstd{\\
\hllin{35\ }}\hlstd{\ \ \ \ }\hlstd{}\hlkwa{PUSH\ }\hlstd{}\hlkwb{DS\ }\\
\hllin{36\ }\hlstd{}\hlstd{\ \ \ \ }\hlstd{}\hlkwa{MOV\ }\hlstd{}\hlkwb{DX}\hlstd{}\hlopt{,\ }\hlstd{}\hlkwb{OFFSET\ }\hlstd{INTR\\
\hllin{37\ }}\hlstd{\ \ \ \ }\hlstd{}\hlkwa{MOV\ }\hlstd{}\hlkwb{AX}\hlstd{}\hlopt{,\ }\hlstd{}\hlkwa{SEG\ }\hlstd{INTR\ \\
\hllin{38\ }}\hlstd{\ \ \ \ }\hlstd{}\hlkwa{MOV\ }\hlstd{}\hlkwb{DS}\hlstd{}\hlopt{,\ }\hlstd{}\hlkwb{AX\ }\\
\hllin{39\ }\hlstd{}\hlstd{\ \ \ \ }\hlstd{}\hlkwa{MOV\ }\hlstd{}\hlkwb{AH}\hlstd{}\hlopt{,\ }\hlstd{}\hlnum{25H}\hlstd{\ \ \ }\hlnum{}\\
\hllin{40\ }\hlstd{}\hlstd{\ \ \ \ }\hlstd{}\hlkwa{MOV\ }\hlstd{}\hlkwb{AL}\hlstd{}\hlopt{,\ }\hlstd{}\hlnum{0}\hlstd{}\hlkwb{BH\ }\\
\hllin{41\ }\hlstd{}\hlstd{\ \ \ \ }\hlstd{}\hlkwa{INT\ }\hlstd{}\hlnum{21H\ }\\
\hllin{42\ }\hlstd{}\hlstd{\ \ \ \ }\hlstd{}\hlkwa{POP\ }\hlstd{}\hlkwb{DS\ }\\
\hllin{43\ }\hlstd{\\
\hllin{44\ }}\hlstd{\ \ \ \ }\hlstd{}\hlkwa{MOV\ }\hlstd{}\hlkwb{AL}\hlstd{}\hlopt{,\ }\hlstd{}\hlnum{0}\hlstd{F7H}\hlstd{\ \ \ \ \ \ \ }\hlstd{\\
\hllin{45\ }}\hlstd{\ \ \ \ }\hlstd{}\hlkwa{OUT\ }\hlstd{}\hlnum{21H}\hlstd{}\hlopt{,\ }\hlstd{}\hlkwb{AL}\\
\hllin{46\ }\hlstd{}\\
\hllin{47\ }\hlkwc{WAIT\textunderscore FOR:}\hlstd{\ \ \ \ }\hlkwc{}\\
\hllin{48\ }\hlstd{}\hlstd{\ \ \ \ }\hlstd{}\hlkwa{MOV\ }\hlstd{}\hlkwb{BL}\hlstd{}\hlopt{,\ }\hlstd{FLAG\\
\hllin{49\ }}\hlstd{\ \ \ \ }\hlstd{}\hlkwa{CMP\ }\hlstd{}\hlkwb{BL}\hlstd{}\hlopt{,\ }\hlstd{}\hlnum{1}\\
\hllin{50\ }\hlstd{}\hlstd{\ \ \ \ }\hlstd{}\hlkwa{JZ\ }\hlstd{ISINT}\hlstd{\ \ \ \ \ \ \ \ \ \ \ }\hlstd{\\
\hllin{51\ }}\hlstd{\ \ \ \ }\hlstd{}\hlkwa{JMP\ }\hlstd{WAIT\textunderscore FOR\\
\hllin{52\ }}\hlstd{\ \ \ }\hlstd{}\\
\hllin{53\ }\hlkwc{ISINT:}\hlstd{\ \ \ \ \ \ \ \ \ \ \ \ \ \ \ \ \ }\hlkwc{}\\
\hllin{54\ }\hlstd{}\hlstd{\ \ \ \ }\hlstd{}\hlkwa{CMP\ }\hlstd{}\hlkwb{CL}\hlstd{}\hlopt{,\ }\hlstd{}\hlnum{0}\hlstd{FFH\\
\hllin{55\ }}\hlstd{\ \ \ \ }\hlstd{}\hlkwa{JZ\ }\hlstd{EXIT}\hlstd{\ \ \ \ \ \ \ \ \ \ \ \ }\hlstd{\\
\hllin{56\ }}\hlstd{\ \ \ \ }\hlstd{}\hlkwa{MOV\ }\hlstd{}\hlkwb{DL}\hlstd{}\hlopt{,\ }\hlstd{}\hlkwb{CL}\hlstd{\ \ \ \ \ \ \ \ \ }\hlkwb{}\\
\hllin{57\ }\hlstd{}\hlstd{\ \ \ \ }\hlstd{}\hlkwa{MOV\ }\hlstd{}\hlkwb{AH}\hlstd{}\hlopt{,\ }\hlstd{}\hlnum{2}\\
\hllin{58\ }\hlstd{}\hlstd{\ \ \ \ }\hlstd{}\hlkwa{INT\ }\hlstd{}\hlnum{21H}\hlstd{\ \ \ \ }\hlnum{}\\
\hllin{59\ }\hlstd{}\hlstd{\ \ \ \ }\hlstd{}\hlkwa{MOV\ }\hlstd{FLAG}\hlopt{,\ }\hlstd{}\hlnum{0}\hlstd{\ \ \ \ \ \ \ \ }\hlnum{}\\
\hllin{60\ }\hlstd{}\hlstd{\ \ \ \ }\hlstd{}\hlkwa{JMP\ }\hlstd{WAIT\textunderscore FOR}\hlstd{\ \ \ \ \ \ \ }\hlstd{\\
\hllin{61\ }}\hlstd{\ \ \ \ \ \ \ \ }\hlstd{}\\
\hllin{62\ }\hlkwc{EXIT:}\\
\hllin{63\ }\hlstd{}\hlstd{\ \ \ \ }\hlstd{\\
\hllin{64\ }}\hlstd{\ \ \ \ }\hlstd{}\hlkwa{MOV\ }\hlstd{}\hlkwb{AL}\hlstd{}\hlopt{,\ }\hlstd{}\hlnum{0}\hlstd{FFH}\hlstd{\ \ \ \ \ \ \ }\hlstd{\\
\hllin{65\ }}\hlstd{\ \ \ \ }\hlstd{}\hlkwa{OUT\ }\hlstd{}\hlnum{21H}\hlstd{}\hlopt{,\ }\hlstd{}\hlkwb{AL}\hlstd{\ \ \ \ \ }\hlkwb{}\\
\hllin{66\ }\hlstd{\\
\hllin{67\ }}\hlstd{\ \ \ \ }\hlstd{\\
\hllin{68\ }}\hlstd{\ \ \ \ }\hlstd{}\hlkwa{PUSH\ }\hlstd{}\hlkwb{DS}\hlstd{\ \ }\hlkwb{}\\
\hllin{69\ }\hlstd{}\hlstd{\ \ \ \ }\hlstd{}\hlkwa{MOV\ }\hlstd{}\hlkwb{DX}\hlstd{}\hlopt{,\ }\hlstd{KEEP\textunderscore IP}\hlstd{\ \ \ \ }\hlstd{\\
\hllin{70\ }}\hlstd{\ \ \ \ }\hlstd{}\hlkwa{MOV\ }\hlstd{}\hlkwb{AX}\hlstd{}\hlopt{,\ }\hlstd{KEEP\textunderscore CS\ \\
\hllin{71\ }}\hlstd{\ \ \ \ }\hlstd{}\hlkwa{MOV\ }\hlstd{}\hlkwb{DS}\hlstd{}\hlopt{,\ }\hlstd{}\hlkwb{AX\ }\\
\hllin{72\ }\hlstd{}\hlstd{\ \ \ \ }\hlstd{}\hlkwa{MOV\ }\hlstd{}\hlkwb{AH}\hlstd{}\hlopt{,\ }\hlstd{}\hlnum{25H\ }\\
\hllin{73\ }\hlstd{}\hlstd{\ \ \ \ }\hlstd{}\hlkwa{MOV\ }\hlstd{}\hlkwb{AL}\hlstd{}\hlopt{,\ }\hlstd{}\hlnum{0}\hlstd{}\hlkwb{BH\ }\\
\hllin{74\ }\hlstd{}\hlstd{\ \ \ \ }\hlstd{}\hlkwa{INT\ }\hlstd{}\hlnum{21H\ }\\
\hllin{75\ }\hlstd{}\hlstd{\ \ \ \ }\hlstd{}\hlkwa{POP\ }\hlstd{}\hlkwb{DS\ }\\
\hllin{76\ }\hlstd{}\hlstd{\ \ \ \ }\hlstd{}\hlkwa{MOV\ }\hlstd{}\hlkwb{AH}\hlstd{}\hlopt{,}\hlstd{}\hlnum{4}\hlstd{}\hlkwb{CH}\\
\hllin{77\ }\hlstd{}\hlstd{\ \ \ \ }\hlstd{}\hlkwa{INT\ }\hlstd{}\hlnum{21H}\\
\hllin{78\ }\hlstd{}\hlstd{\ \ \ \ }\hlstd{\\
\hllin{79\ }INTR\ }\hlkwa{PROC}\\
\hllin{80\ }\hlstd{}\hlstd{\ \ \ \ }\hlstd{}\hlkwa{MOV\ }\hlstd{FLAG}\hlopt{,\ }\hlstd{}\hlnum{1}\hlstd{\ \ \ \ \ \ \ \ }\hlnum{}\\
\hllin{81\ }\hlstd{}\hlstd{\ \ \ \ }\hlstd{}\hlkwa{MOV\ }\hlstd{}\hlkwb{DX}\hlstd{}\hlopt{,\ }\hlstd{}\hlnum{280H}\\
\hllin{82\ }\hlstd{}\hlstd{\ \ \ \ }\hlstd{}\hlkwa{IN\ }\hlstd{}\hlkwb{AL}\hlstd{}\hlopt{,\ }\hlstd{}\hlkwb{DX}\\
\hllin{83\ }\hlstd{}\hlstd{\ \ \ \ }\hlstd{}\hlkwa{MOV\ }\hlstd{}\hlkwb{CL}\hlstd{}\hlopt{,\ }\hlstd{}\hlkwb{AL}\hlstd{\ \ \ \ \ \ \ \ \ }\hlkwb{}\\
\hllin{84\ }\hlstd{}\hlstd{\ \ \ \ }\hlstd{}\hlkwa{MOV\ }\hlstd{}\hlkwb{AL}\hlstd{}\hlopt{,\ }\hlstd{}\hlnum{20H}\hlstd{\ \ \ \ \ \ \ \ }\hlnum{}\\
\hllin{85\ }\hlstd{}\hlstd{\ \ \ \ }\hlstd{}\hlkwa{OUT\ }\hlstd{}\hlnum{20H}\hlstd{}\hlopt{,\ }\hlstd{}\hlkwb{AL}\hlstd{\ \ \ \ \ \ }\hlkwb{}\\
\hllin{86\ }\hlstd{}\hlstd{\ \ \ \ }\hlstd{}\hlkwa{IRET}\hlstd{\ \ \ \ \ \ \ \ }\hlkwa{}\\
\hllin{87\ }\hlstd{INTR\ }\hlkwa{ENDP}\\
\hllin{88\ }\hlstd{}\hlstd{\ \ \ \ }\hlstd{\\
\hllin{89\ }MAIN\ }\hlkwa{ENDP}\\
\hllin{90\ }\hlstd{CODE\ }\hlkwa{ENDS}\\
\hllin{91\ }\hlstd{}\hlstd{\ \ \ \ }\hlstd{}\hlkwa{END\ }\hlstd{START}\\
\mbox{}
\normalfont
\normalsize

\subsubsection{选做任务一}
修改主程序实现密码检测功能,连续两次从A口拨入数据,
与计算机内部事先存放的两字节数比较,相符则在CRT上显
示“OK”,否则重新输入,代码如下所示:
\noindent
\ttfamily
\hlstd{\hllin{01\ }DATAS\ }\hlkwa{SEGMENT}\\
\hllin{02\ }\hlstd{KEEP\textunderscore IP\ }\hlkwa{DW\ }\hlstd{}\hlnum{0}\\
\hllin{03\ }\hlstd{KEEP\textunderscore CS\ }\hlkwa{DW\ }\hlstd{}\hlnum{0}\hlstd{\ \ }\hlnum{}\\
\hllin{04\ }\hlstd{FLAG\ }\hlkwa{DB\ }\hlstd{}\hlnum{0}\hlstd{\ \ \ }\hlnum{}\\
\hllin{05\ }\hlstd{PASSWORD\ }\hlkwa{DB\ }\hlstd{}\hlnum{0}\hlstd{F0H}\hlopt{,\ }\hlstd{}\hlnum{0}\hlstd{FH\ \\
\hllin{06\ }OK\ }\hlkwa{DB\ }\hlstd{}\hlstr{'OK'}\hlstd{}\hlopt{,}\hlstd{}\hlnum{0}\hlstd{}\hlkwb{DH}\hlstd{}\hlopt{,\ }\hlstd{}\hlnum{0}\hlstd{}\hlkwb{AH}\hlstd{}\hlopt{,\ }\hlstd{}\hlstr{'\$'}\hlstd{\\
\hllin{07\ }NO\ }\hlkwa{DB\ }\hlstd{}\hlstr{'NO'}\hlstd{}\hlopt{,}\hlstd{}\hlnum{0}\hlstd{}\hlkwb{DH}\hlstd{}\hlopt{,\ }\hlstd{}\hlnum{0}\hlstd{}\hlkwb{AH}\hlstd{}\hlopt{,\ }\hlstd{}\hlstr{'\$'}\hlstd{\\
\hllin{08\ }DATAS\ }\hlkwa{ENDS}\\
\hllin{09\ }\hlstd{\\
\hllin{10\ }STACKS\ }\hlkwa{SEGMENT}\\
\hllin{11\ }\hlstd{STACKS\ }\hlkwa{ENDS}\\
\hllin{12\ }\hlstd{\\
\hllin{13\ }CODES\ }\hlkwa{SEGMENT}\\
\hllin{14\ }\hlstd{}\hlstd{\ \ \ \ }\hlstd{}\hlkwa{ASSUME\ }\hlstd{}\hlkwb{CS}\hlstd{}\hlopt{:}\hlstd{CODES}\hlopt{,}\hlstd{}\hlkwb{DS}\hlstd{}\hlopt{:}\hlstd{DATAS}\hlopt{,}\hlstd{}\hlkwb{SS}\hlstd{}\hlopt{:}\hlstd{STACKS\\
\hllin{15\ }\\
\hllin{16\ }MAIN\ }\hlkwa{PROC\ FAR}\\
\hllin{17\ }\hlstd{}\hlkwc{START:}\\
\hllin{18\ }\hlstd{}\hlstd{\ \ \ \ }\hlstd{}\hlkwa{MOV\ }\hlstd{}\hlkwb{AX}\hlstd{}\hlopt{,}\hlstd{DATAS\\
\hllin{19\ }}\hlstd{\ \ \ \ }\hlstd{}\hlkwa{MOV\ }\hlstd{}\hlkwb{DS}\hlstd{}\hlopt{,}\hlstd{}\hlkwb{AX}\\
\hllin{20\ }\hlstd{\\
\hllin{21\ }}\hlstd{\ \ \ \ }\hlstd{\\
\hllin{22\ }}\hlstd{\ \ \ \ }\hlstd{}\hlkwa{MOV\ }\hlstd{}\hlkwb{DX}\hlstd{}\hlopt{,\ }\hlstd{}\hlnum{283H\ }\\
\hllin{23\ }\hlstd{}\hlstd{\ \ \ \ }\hlstd{}\hlkwa{MOV\ }\hlstd{}\hlkwb{AL}\hlstd{}\hlopt{,\ }\hlstd{}\hlnum{10110000}\hlstd{B\\
\hllin{24\ }}\hlstd{\ \ \ \ }\hlstd{}\hlkwa{OUT\ }\hlstd{}\hlkwb{DX}\hlstd{}\hlopt{,\ }\hlstd{}\hlkwb{AL}\\
\hllin{25\ }\hlstd{\\
\hllin{26\ }}\hlstd{\ \ \ \ }\hlstd{}\hlkwa{MOV\ }\hlstd{}\hlkwb{DX}\hlstd{}\hlopt{,\ }\hlstd{}\hlnum{283H}\\
\hllin{27\ }\hlstd{}\hlstd{\ \ \ \ }\hlstd{}\hlkwa{MOV\ }\hlstd{}\hlkwb{AL}\hlstd{}\hlopt{,\ }\hlstd{}\hlnum{00001001}\hlstd{B\\
\hllin{28\ }}\hlstd{\ \ \ \ }\hlstd{}\hlkwa{OUT\ }\hlstd{}\hlkwb{DX}\hlstd{}\hlopt{,\ }\hlstd{}\hlkwb{AL}\hlstd{\ \ \ \ \ \ \ }\hlkwb{}\\
\hllin{29\ }\hlstd{}\hlstd{\ \ \ \ }\hlstd{\\
\hllin{30\ }}\hlstd{\ \ \ \ }\hlstd{\\
\hllin{31\ }}\hlstd{\ \ \ \ }\hlstd{}\hlkwa{MOV\ }\hlstd{}\hlkwb{AH}\hlstd{}\hlopt{,\ }\hlstd{}\hlnum{35H}\\
\hllin{32\ }\hlstd{}\hlstd{\ \ \ \ }\hlstd{}\hlkwa{MOV\ }\hlstd{}\hlkwb{AL}\hlstd{}\hlopt{,\ }\hlstd{}\hlnum{0}\hlstd{}\hlkwb{BH\ }\\
\hllin{33\ }\hlstd{}\hlstd{\ \ \ \ }\hlstd{}\hlkwa{INT\ }\hlstd{}\hlnum{21H}\\
\hllin{34\ }\hlstd{}\hlstd{\ \ \ \ }\hlstd{}\hlkwa{MOV\ }\hlstd{KEEP\textunderscore IP}\hlopt{,\ }\hlstd{}\hlkwb{BX}\\
\hllin{35\ }\hlstd{}\hlstd{\ \ \ \ }\hlstd{}\hlkwa{MOV\ }\hlstd{KEEP\textunderscore CS}\hlopt{,\ }\hlstd{}\hlkwb{ES}\\
\hllin{36\ }\hlstd{}\hlstd{\ \ \ \ }\hlstd{\\
\hllin{37\ }}\hlstd{\ \ \ }\hlstd{\\
\hllin{38\ }}\hlstd{\ \ \ \ }\hlstd{}\hlkwa{PUSH\ }\hlstd{}\hlkwb{DS\ }\\
\hllin{39\ }\hlstd{}\hlstd{\ \ \ \ }\hlstd{}\hlkwa{MOV\ }\hlstd{}\hlkwb{DX}\hlstd{}\hlopt{,\ }\hlstd{}\hlkwb{OFFSET\ }\hlstd{INTR\\
\hllin{40\ }}\hlstd{\ \ \ \ }\hlstd{}\hlkwa{MOV\ }\hlstd{}\hlkwb{AX}\hlstd{}\hlopt{,\ }\hlstd{}\hlkwa{SEG\ }\hlstd{INTR\ \\
\hllin{41\ }}\hlstd{\ \ \ \ }\hlstd{}\hlkwa{MOV\ }\hlstd{}\hlkwb{DS}\hlstd{}\hlopt{,\ }\hlstd{}\hlkwb{AX\ }\\
\hllin{42\ }\hlstd{}\hlstd{\ \ \ \ }\hlstd{}\hlkwa{MOV\ }\hlstd{}\hlkwb{AH}\hlstd{}\hlopt{,\ }\hlstd{}\hlnum{25H}\hlstd{\ \ \ }\hlnum{}\\
\hllin{43\ }\hlstd{}\hlstd{\ \ \ \ }\hlstd{}\hlkwa{MOV\ }\hlstd{}\hlkwb{AL}\hlstd{}\hlopt{,\ }\hlstd{}\hlnum{0}\hlstd{}\hlkwb{BH\ }\\
\hllin{44\ }\hlstd{}\hlstd{\ \ \ \ }\hlstd{}\hlkwa{INT\ }\hlstd{}\hlnum{21H\ }\\
\hllin{45\ }\hlstd{}\hlstd{\ \ \ \ }\hlstd{}\hlkwa{POP\ }\hlstd{}\hlkwb{DS\ }\\
\hllin{46\ }\hlstd{\\
\hllin{47\ }}\hlstd{\ \ \ \ }\hlstd{}\hlkwa{MOV\ }\hlstd{}\hlkwb{AL}\hlstd{}\hlopt{,\ }\hlstd{}\hlnum{0}\hlstd{F7H}\hlstd{\ \ \ }\hlstd{\\
\hllin{48\ }}\hlstd{\ \ \ \ }\hlstd{}\hlkwa{OUT\ }\hlstd{}\hlnum{21H}\hlstd{}\hlopt{,\ }\hlstd{}\hlkwb{AL}\\
\hllin{49\ }\hlstd{}\\
\hllin{50\ }\hlkwc{WAIT\textunderscore FOR1:}\hlstd{\ \ \ \ \ \ \ \ \ }\hlkwc{}\\
\hllin{51\ }\hlstd{}\hlstd{\ \ \ \ }\hlstd{}\hlkwa{MOV\ }\hlstd{}\hlkwb{BL}\hlstd{}\hlopt{,\ }\hlstd{FLAG\\
\hllin{52\ }}\hlstd{\ \ \ \ }\hlstd{}\hlkwa{CMP\ }\hlstd{}\hlkwb{BL}\hlstd{}\hlopt{,\ }\hlstd{}\hlnum{1}\\
\hllin{53\ }\hlstd{}\hlstd{\ \ \ \ }\hlstd{}\hlkwa{JZ\ }\hlstd{WAIT\textunderscore FOR2}\hlstd{\ \ \ }\hlstd{\\
\hllin{54\ }}\hlstd{\ \ \ \ }\hlstd{}\hlkwa{JMP\ }\hlstd{WAIT\textunderscore FOR1\\
\hllin{55\ }}\hlstd{\ \ \ }\hlstd{}\\
\hllin{56\ }\hlkwc{WAIT\textunderscore FOR2:}\hlstd{\ \ \ \ \ \ \ \ \ }\hlkwc{}\\
\hllin{57\ }\hlstd{}\hlstd{\ \ \ \ }\hlstd{}\hlkwa{CMP\ }\hlstd{}\hlkwb{CL}\hlstd{}\hlopt{,\ }\hlstd{}\hlnum{0}\hlstd{FFH\\
\hllin{58\ }}\hlstd{\ \ \ \ }\hlstd{}\hlkwa{JZ\ }\hlstd{EXIT\\
\hllin{59\ }}\hlstd{\ \ \ \ }\hlstd{}\hlkwa{MOV\ }\hlstd{}\hlkwb{DL}\hlstd{}\hlopt{,\ }\hlstd{}\hlkwb{CL}\hlstd{\ \ \ \ \ }\hlkwb{}\\
\hllin{60\ }\hlstd{}\hlstd{\ \ \ \ }\hlstd{}\hlkwa{MOV\ }\hlstd{}\hlkwb{AH}\hlstd{}\hlopt{,\ }\hlstd{}\hlnum{2}\\
\hllin{61\ }\hlstd{}\hlstd{\ \ \ \ }\hlstd{}\hlkwa{INT\ }\hlstd{}\hlnum{21H}\hlstd{\ \ }\hlnum{}\\
\hllin{62\ }\hlstd{}\hlstd{\ \ \ \ }\hlstd{}\hlkwa{MOV\ }\hlstd{}\hlkwb{BH}\hlstd{}\hlopt{,}\hlstd{\ \ \ \ }\hlopt{}\hlstd{}\hlkwb{CL}\hlstd{\ \ }\hlkwb{}\\
\hllin{63\ }\hlstd{}\hlstd{\ \ \ \ }\hlstd{}\hlkwa{MOV\ }\hlstd{FLAG}\hlopt{,\ }\hlstd{}\hlnum{0}\hlstd{\ \ \ \ }\hlnum{}\\
\hllin{64\ }\hlstd{}\hlstd{\ \ \ \ }\hlstd{}\hlkwa{JMP\ }\hlstd{WAIT2}\\
\hllin{65\ }\hlkwc{WAIT2:}\\
\hllin{66\ }\hlstd{}\hlstd{\ \ \ \ }\hlstd{}\hlkwa{MOV\ }\hlstd{}\hlkwb{BL}\hlstd{}\hlopt{,\ }\hlstd{FLAG\\
\hllin{67\ }}\hlstd{\ \ \ \ }\hlstd{}\hlkwa{CMP\ }\hlstd{}\hlkwb{BL}\hlstd{}\hlopt{,\ }\hlstd{}\hlnum{1}\\
\hllin{68\ }\hlstd{}\hlstd{\ \ \ \ }\hlstd{}\hlkwa{JZ\ }\hlstd{CHECK}\hlstd{\ \ \ \ \ \ \ }\hlstd{\\
\hllin{69\ }}\hlstd{\ \ \ \ }\hlstd{}\hlkwa{JMP\ }\hlstd{WAIT2}\hlstd{\ \ \ }\hlstd{}\\
\hllin{70\ }\\
\hllin{71\ }\hlkwc{CHECK:}\\
\hllin{72\ }\hlstd{\\
\hllin{73\ }}\hlstd{\ \ \ \ }\hlstd{}\hlkwa{CMP\ }\hlstd{}\hlkwb{CL}\hlstd{}\hlopt{,\ }\hlstd{}\hlnum{0}\hlstd{FFH\\
\hllin{74\ }}\hlstd{\ \ \ \ }\hlstd{}\hlkwa{JZ\ }\hlstd{EXIT\\
\hllin{75\ }}\hlstd{\ \ \ \ }\hlstd{}\hlkwa{MOV\ }\hlstd{}\hlkwb{DL}\hlstd{}\hlopt{,\ }\hlstd{}\hlkwb{CL}\hlstd{\ \ \ \ \ }\hlkwb{}\\
\hllin{76\ }\hlstd{}\hlstd{\ \ \ \ }\hlstd{}\hlkwa{MOV\ }\hlstd{}\hlkwb{AH}\hlstd{}\hlopt{,\ }\hlstd{}\hlnum{2}\\
\hllin{77\ }\hlstd{}\hlstd{\ \ \ \ }\hlstd{}\hlkwa{INT\ }\hlstd{}\hlnum{21H}\hlstd{\ \ }\hlnum{}\\
\hllin{78\ }\hlstd{}\hlstd{\ \ \ \ }\hlstd{}\hlkwa{CMP\ }\hlstd{}\hlkwb{BH}\hlstd{}\hlopt{,}\hlstd{PASSWORD\\
\hllin{79\ }}\hlstd{\ \ \ \ }\hlstd{}\hlkwa{JNZ\ }\hlstd{ERROR\\
\hllin{80\ }}\hlstd{\ \ \ \ }\hlstd{}\hlkwa{CMP\ }\hlstd{}\hlkwb{CL}\hlstd{}\hlopt{,}\hlstd{PASSWORD}\hlopt{+}\hlstd{}\hlnum{1}\\
\hllin{81\ }\hlstd{}\hlstd{\ \ \ \ }\hlstd{}\hlkwa{JNZ\ }\hlstd{ERROR\\
\hllin{82\ }}\hlstd{\ \ \ \ }\hlstd{\\
\hllin{83\ }}\hlstd{\ \ \ \ }\hlstd{}\hlkwa{MOV\ }\hlstd{}\hlkwb{AH}\hlstd{}\hlopt{,\ }\hlstd{}\hlnum{9}\\
\hllin{84\ }\hlstd{}\hlstd{\ \ \ \ }\hlstd{}\hlkwa{LEA\ }\hlstd{}\hlkwb{DX}\hlstd{}\hlopt{,\ }\hlstd{OK}\hlstd{\ \ \ \ \ \ \ \ \ \ \ \ \ \ \ \ \ }\hlstd{\\
\hllin{85\ }}\hlstd{\ \ \ \ }\hlstd{}\hlkwa{INT\ }\hlstd{}\hlnum{21H}\\
\hllin{86\ }\hlstd{}\hlstd{\ \ \ \ }\hlstd{}\hlkwa{JMP\ }\hlstd{WAIT\textunderscore FOR1}\\
\hllin{87\ }\\
\hllin{88\ }\hlkwc{ERROR:}\\
\hllin{89\ }\hlstd{}\hlstd{\ \ \ \ }\hlstd{}\hlkwa{MOV\ }\hlstd{}\hlkwb{AH}\hlstd{}\hlopt{,\ }\hlstd{}\hlnum{9}\\
\hllin{90\ }\hlstd{}\hlstd{\ \ \ \ }\hlstd{}\hlkwa{LEA\ }\hlstd{}\hlkwb{DX}\hlstd{}\hlopt{,\ }\hlstd{NO}\hlstd{\ \ \ \ \ \ \ \ \ \ \ \ \ \ \ \ \ }\hlstd{\\
\hllin{91\ }}\hlstd{\ \ \ \ }\hlstd{}\hlkwa{INT\ }\hlstd{}\hlnum{21H}\\
\hllin{92\ }\hlstd{}\hlstd{\ \ \ \ }\hlstd{}\hlkwa{JMP\ }\hlstd{WAIT\textunderscore FOR1\\
\hllin{93\ }}\hlstd{\ \ \ \ \ \ \ \ }\hlstd{}\\
\hllin{94\ }\hlkwc{EXIT:}\\
\hllin{95\ }\hlstd{}\hlstd{\ \ \ \ }\hlstd{\\
\hllin{96\ }}\hlstd{\ \ \ \ }\hlstd{}\hlkwa{MOV\ }\hlstd{}\hlkwb{AL}\hlstd{}\hlopt{,\ }\hlstd{}\hlnum{0}\hlstd{FFH}\hlstd{\ \ \ \ \ }\hlstd{\\
\hllin{97\ }}\hlstd{\ \ \ \ }\hlstd{}\hlkwa{OUT\ }\hlstd{}\hlnum{21H}\hlstd{}\hlopt{,\ }\hlstd{}\hlkwb{AL}\hlstd{\ \ \ \ \ }\hlkwb{}\\
\hllin{98\ }\hlstd{\\
\hllin{99\ }}\hlstd{\ \ \ \ }\hlstd{\\
\hllin{100\ }}\hlstd{\ \ \ \ }\hlstd{}\hlkwa{PUSH\ }\hlstd{}\hlkwb{DS}\hlstd{\ \ }\hlkwb{}\\
\hllin{101\ }\hlstd{}\hlstd{\ \ \ \ }\hlstd{}\hlkwa{MOV\ }\hlstd{}\hlkwb{DX}\hlstd{}\hlopt{,\ }\hlstd{KEEP\textunderscore IP}\hlstd{\ \ }\hlstd{\\
\hllin{102\ }}\hlstd{\ \ \ \ }\hlstd{}\hlkwa{MOV\ }\hlstd{}\hlkwb{AX}\hlstd{}\hlopt{,\ }\hlstd{KEEP\textunderscore CS\ \\
\hllin{103\ }}\hlstd{\ \ \ \ }\hlstd{}\hlkwa{MOV\ }\hlstd{}\hlkwb{DS}\hlstd{}\hlopt{,\ }\hlstd{}\hlkwb{AX\ }\\
\hllin{104\ }\hlstd{}\hlstd{\ \ \ \ }\hlstd{}\hlkwa{MOV\ }\hlstd{}\hlkwb{AH}\hlstd{}\hlopt{,\ }\hlstd{}\hlnum{25H\ }\\
\hllin{105\ }\hlstd{}\hlstd{\ \ \ \ }\hlstd{}\hlkwa{MOV\ }\hlstd{}\hlkwb{AL}\hlstd{}\hlopt{,\ }\hlstd{}\hlnum{0}\hlstd{}\hlkwb{BH\ }\\
\hllin{106\ }\hlstd{}\hlstd{\ \ \ \ }\hlstd{}\hlkwa{INT\ }\hlstd{}\hlnum{21H\ }\\
\hllin{107\ }\hlstd{}\hlstd{\ \ \ \ }\hlstd{}\hlkwa{POP\ }\hlstd{}\hlkwb{DS\ }\\
\hllin{108\ }\hlstd{}\hlstd{\ \ \ \ }\hlstd{}\hlkwa{MOV\ }\hlstd{}\hlkwb{AH}\hlstd{}\hlopt{,}\hlstd{}\hlnum{4}\hlstd{}\hlkwb{CH}\\
\hllin{109\ }\hlstd{}\hlstd{\ \ \ \ }\hlstd{}\hlkwa{INT\ }\hlstd{}\hlnum{21H}\\
\hllin{110\ }\hlstd{}\hlstd{\ \ \ \ }\hlstd{\\
\hllin{111\ }INTR\ }\hlkwa{PROC}\\
\hllin{112\ }\hlstd{}\hlstd{\ \ \ \ }\hlstd{}\hlkwa{MOV\ }\hlstd{FLAG}\hlopt{,\ }\hlstd{}\hlnum{1}\hlstd{\ \ \ \ \ \ }\hlnum{}\\
\hllin{113\ }\hlstd{}\hlstd{\ \ \ \ }\hlstd{}\hlkwa{MOV\ }\hlstd{}\hlkwb{DX}\hlstd{}\hlopt{,\ }\hlstd{}\hlnum{280H}\\
\hllin{114\ }\hlstd{}\hlstd{\ \ \ \ }\hlstd{}\hlkwa{IN\ }\hlstd{}\hlkwb{AL}\hlstd{}\hlopt{,\ }\hlstd{}\hlkwb{DX}\\
\hllin{115\ }\hlstd{}\hlstd{\ \ \ \ }\hlstd{}\hlkwa{MOV\ }\hlstd{}\hlkwb{CL}\hlstd{}\hlopt{,\ }\hlstd{}\hlkwb{AL}\\
\hllin{116\ }\hlstd{}\hlstd{\ \ \ \ }\hlstd{}\hlkwa{MOV\ }\hlstd{}\hlkwb{AL}\hlstd{}\hlopt{,\ }\hlstd{}\hlnum{20H}\\
\hllin{117\ }\hlstd{}\hlstd{\ \ \ \ }\hlstd{}\hlkwa{OUT\ }\hlstd{}\hlnum{20H}\hlstd{}\hlopt{,\ }\hlstd{}\hlkwb{AL}\hlstd{\ \ \ \ \ \ }\hlkwb{}\\
\hllin{118\ }\hlstd{}\hlstd{\ \ \ \ }\hlstd{}\hlkwa{IRET}\hlstd{\ \ \ \ \ \ \ \ }\hlkwa{}\\
\hllin{119\ }\hlstd{INTR\ }\hlkwa{ENDP}\\
\hllin{120\ }\hlstd{}\hlstd{\ \ \ \ }\hlstd{\\
\hllin{121\ }MAIN\ }\hlkwa{ENDP}\\
\hllin{122\ }\hlstd{CODES\ }\hlkwa{ENDS}\\
\hllin{123\ }\hlstd{}\hlstd{\ \ \ \ }\hlstd{}\hlkwa{END\ }\hlstd{START}\\
\mbox{}
\normalfont
\normalsize

\subsubsection{选做任务二}
将8255电路A口改成方式1输出(仅将PA7接一只LED示范即
可),修改前面的程序实现每次中断后,通过A口输出数据
控制LED状态在0,1之间翻转,代码如下所示。
\noindent
\ttfamily
\hlstd{\hllin{01\ }DATA\ }\hlkwa{SEGMENT}\\
\hllin{02\ }\hlstd{SIG\ }\hlkwa{DB\ }\hlstd{}\hlnum{0}\\
\hllin{03\ }\hlstd{KEEP\textunderscore IP\ }\hlkwa{DW\ }\hlstd{}\hlnum{0}\\
\hllin{04\ }\hlstd{KEEP\textunderscore CS\ }\hlkwa{DW\ }\hlstd{}\hlnum{0}\hlstd{\ \ }\hlnum{}\\
\hllin{05\ }\hlstd{FLAG\ }\hlkwa{DB\ }\hlstd{}\hlnum{0}\\
\hllin{06\ }\hlstd{DATA\ }\hlkwa{ENDS}\\
\hllin{07\ }\hlstd{\\
\hllin{08\ }STACKS\ }\hlkwa{SEGMENT}\\
\hllin{09\ }\hlstd{STACKS\ }\hlkwa{ENDS}\\
\hllin{10\ }\hlstd{\\
\hllin{11\ }CODE\ }\hlkwa{SEGMENT}\\
\hllin{12\ }\hlstd{}\hlstd{\ \ \ \ }\hlstd{}\hlkwa{ASSUME\ }\hlstd{}\hlkwb{CS}\hlstd{}\hlopt{:}\hlstd{CODE}\hlopt{,}\hlstd{}\hlkwb{DS}\hlstd{}\hlopt{:}\hlstd{DATA}\hlopt{,}\hlstd{}\hlkwb{SS}\hlstd{}\hlopt{:}\hlstd{STACKS\\
\hllin{13\ }\\
\hllin{14\ }MAIN\ }\hlkwa{PROC\ FAR}\\
\hllin{15\ }\hlstd{}\hlkwc{START:}\\
\hllin{16\ }\hlstd{}\hlstd{\ \ \ \ }\hlstd{}\hlkwa{MOV\ }\hlstd{}\hlkwb{AX}\hlstd{}\hlopt{,}\hlstd{DATA\\
\hllin{17\ }}\hlstd{\ \ \ \ }\hlstd{}\hlkwa{MOV\ }\hlstd{}\hlkwb{DS}\hlstd{}\hlopt{,}\hlstd{}\hlkwb{AX}\\
\hllin{18\ }\hlstd{\\
\hllin{19\ }}\hlstd{\ \ \ \ }\hlstd{\\
\hllin{20\ }}\hlstd{\ \ \ \ }\hlstd{}\hlkwa{MOV\ }\hlstd{}\hlkwb{DX}\hlstd{}\hlopt{,\ }\hlstd{}\hlnum{283H}\\
\hllin{21\ }\hlstd{}\hlstd{\ \ \ \ }\hlstd{}\hlkwa{MOV\ }\hlstd{}\hlkwb{AL}\hlstd{}\hlopt{,\ }\hlstd{}\hlnum{10101000}\hlstd{B\\
\hllin{22\ }}\hlstd{\ \ \ \ }\hlstd{}\hlkwa{OUT\ }\hlstd{}\hlkwb{DX}\hlstd{}\hlopt{,\ }\hlstd{}\hlkwb{AL}\\
\hllin{23\ }\hlstd{\\
\hllin{24\ }}\hlstd{\ \ \ \ }\hlstd{}\hlkwa{MOV\ }\hlstd{}\hlkwb{DX}\hlstd{}\hlopt{,\ }\hlstd{}\hlnum{283H}\\
\hllin{25\ }\hlstd{}\hlstd{\ \ \ \ }\hlstd{}\hlkwa{MOV\ }\hlstd{}\hlkwb{AL}\hlstd{}\hlopt{,\ }\hlstd{}\hlnum{11001000}\hlstd{B\\
\hllin{26\ }}\hlstd{\ \ \ \ }\hlstd{}\hlkwa{OUT\ }\hlstd{}\hlkwb{DX}\hlstd{}\hlopt{,\ }\hlstd{}\hlkwb{AL}\hlstd{\ \ \ \ \ \ \ }\hlkwb{}\\
\hllin{27\ }\hlstd{}\hlstd{\ \ \ \ }\hlstd{\\
\hllin{28\ }}\hlstd{\ \ \ \ }\hlstd{\\
\hllin{29\ }}\hlstd{\ \ \ \ }\hlstd{}\hlkwa{MOV\ }\hlstd{}\hlkwb{AH}\hlstd{}\hlopt{,\ }\hlstd{}\hlnum{35H}\hlstd{\ \ \ \ \ \ }\hlnum{}\\
\hllin{30\ }\hlstd{}\hlstd{\ \ \ \ }\hlstd{}\hlkwa{MOV\ }\hlstd{}\hlkwb{AL}\hlstd{}\hlopt{,\ }\hlstd{}\hlnum{0}\hlstd{}\hlkwb{BH\ }\\
\hllin{31\ }\hlstd{}\hlstd{\ \ \ \ }\hlstd{}\hlkwa{INT\ }\hlstd{}\hlnum{21H}\\
\hllin{32\ }\hlstd{}\hlstd{\ \ \ \ }\hlstd{}\hlkwa{MOV\ }\hlstd{KEEP\textunderscore IP}\hlopt{,\ }\hlstd{}\hlkwb{BX}\hlstd{\ \ }\hlkwb{}\\
\hllin{33\ }\hlstd{}\hlstd{\ \ \ \ }\hlstd{}\hlkwa{MOV\ }\hlstd{KEEP\textunderscore CS}\hlopt{,\ }\hlstd{}\hlkwb{ES}\hlstd{\ \ }\hlkwb{}\\
\hllin{34\ }\hlstd{}\hlstd{\ \ \ \ }\hlstd{\\
\hllin{35\ }}\hlstd{\ \ \ \ }\hlstd{\\
\hllin{36\ }}\hlstd{\ \ \ }\hlstd{\\
\hllin{37\ }}\hlstd{\ \ \ \ }\hlstd{}\hlkwa{PUSH\ }\hlstd{}\hlkwb{DS\ }\\
\hllin{38\ }\hlstd{}\hlstd{\ \ \ \ }\hlstd{}\hlkwa{MOV\ }\hlstd{}\hlkwb{DX}\hlstd{}\hlopt{,\ }\hlstd{}\hlkwb{OFFSET\ }\hlstd{INTR\\
\hllin{39\ }}\hlstd{\ \ \ \ }\hlstd{}\hlkwa{MOV\ }\hlstd{}\hlkwb{AX}\hlstd{}\hlopt{,\ }\hlstd{}\hlkwa{SEG\ }\hlstd{INTR\ \\
\hllin{40\ }}\hlstd{\ \ \ \ }\hlstd{}\hlkwa{MOV\ }\hlstd{}\hlkwb{DS}\hlstd{}\hlopt{,\ }\hlstd{}\hlkwb{AX\ }\\
\hllin{41\ }\hlstd{}\hlstd{\ \ \ \ }\hlstd{}\hlkwa{MOV\ }\hlstd{}\hlkwb{AH}\hlstd{}\hlopt{,\ }\hlstd{}\hlnum{25H}\hlstd{\ \ \ }\hlnum{}\\
\hllin{42\ }\hlstd{}\hlstd{\ \ \ \ }\hlstd{}\hlkwa{MOV\ }\hlstd{}\hlkwb{AL}\hlstd{}\hlopt{,\ }\hlstd{}\hlnum{0}\hlstd{}\hlkwb{BH\ }\\
\hllin{43\ }\hlstd{}\hlstd{\ \ \ \ }\hlstd{}\hlkwa{INT\ }\hlstd{}\hlnum{21H\ }\\
\hllin{44\ }\hlstd{}\hlstd{\ \ \ \ }\hlstd{}\hlkwa{POP\ }\hlstd{}\hlkwb{DS\ }\\
\hllin{45\ }\hlstd{\\
\hllin{46\ }}\hlstd{\ \ \ \ }\hlstd{}\hlkwa{MOV\ }\hlstd{}\hlkwb{AL}\hlstd{}\hlopt{,\ }\hlstd{}\hlnum{0}\hlstd{F7H}\hlstd{\ \ \ \ \ \ \ }\hlstd{\\
\hllin{47\ }}\hlstd{\ \ \ \ }\hlstd{}\hlkwa{OUT\ }\hlstd{}\hlnum{21H}\hlstd{}\hlopt{,\ }\hlstd{}\hlkwb{AL}\\
\hllin{48\ }\hlstd{}\\
\hllin{49\ }\hlkwc{WAIT\textunderscore FOR:}\hlstd{\ \ \ \ }\hlkwc{}\\
\hllin{50\ }\hlstd{}\hlstd{\ \ \ \ }\hlstd{}\hlkwa{MOV\ }\hlstd{}\hlkwb{BL}\hlstd{}\hlopt{,\ }\hlstd{FLAG\\
\hllin{51\ }}\hlstd{\ \ \ \ }\hlstd{}\hlkwa{CMP\ }\hlstd{}\hlkwb{BL}\hlstd{}\hlopt{,\ }\hlstd{}\hlnum{1}\\
\hllin{52\ }\hlstd{}\hlstd{\ \ \ \ }\hlstd{}\hlkwa{JZ\ }\hlstd{ISINT}\hlstd{\ \ \ \ \ \ \ \ \ \ \ }\hlstd{\\
\hllin{53\ }}\hlstd{\ \ \ \ }\hlstd{}\hlkwa{JMP\ }\hlstd{WAIT\textunderscore FOR\\
\hllin{54\ }}\hlstd{\ \ \ }\hlstd{}\\
\hllin{55\ }\hlkwc{ISINT:}\hlstd{\ \ \ \ \ \ \ \ \ \ \ \ \ \ \ \ \ }\hlkwc{}\\
\hllin{56\ }\hlstd{}\hlstd{\ \ \ \ }\hlstd{}\hlkwa{MOV\ }\hlstd{FLAG}\hlopt{,\ }\hlstd{}\hlnum{0}\hlstd{\ \ \ \ \ \ \ \ }\hlnum{}\\
\hllin{57\ }\hlstd{}\hlstd{\ \ \ \ }\hlstd{}\hlkwa{JMP\ }\hlstd{WAIT\textunderscore FOR}\hlstd{\ \ \ \ \ \ \ }\hlstd{\\
\hllin{58\ }}\hlstd{\ \ \ \ \ \ \ \ }\hlstd{}\\
\hllin{59\ }\hlkwc{EXIT:}\\
\hllin{60\ }\hlstd{}\hlstd{\ \ \ \ }\hlstd{\\
\hllin{61\ }}\hlstd{\ \ \ \ }\hlstd{}\hlkwa{MOV\ }\hlstd{}\hlkwb{AL}\hlstd{}\hlopt{,\ }\hlstd{}\hlnum{0}\hlstd{FFH}\hlstd{\ \ \ \ \ \ \ }\hlstd{\\
\hllin{62\ }}\hlstd{\ \ \ \ }\hlstd{}\hlkwa{OUT\ }\hlstd{}\hlnum{21H}\hlstd{}\hlopt{,\ }\hlstd{}\hlkwb{AL}\hlstd{\ \ \ \ \ }\hlkwb{}\\
\hllin{63\ }\hlstd{\\
\hllin{64\ }}\hlstd{\ \ \ \ }\hlstd{\\
\hllin{65\ }}\hlstd{\ \ \ \ }\hlstd{}\hlkwa{PUSH\ }\hlstd{}\hlkwb{DS}\hlstd{\ \ }\hlkwb{}\\
\hllin{66\ }\hlstd{}\hlstd{\ \ \ \ }\hlstd{}\hlkwa{MOV\ }\hlstd{}\hlkwb{DX}\hlstd{}\hlopt{,\ }\hlstd{KEEP\textunderscore IP}\hlstd{\ \ \ \ }\hlstd{\\
\hllin{67\ }}\hlstd{\ \ \ \ }\hlstd{}\hlkwa{MOV\ }\hlstd{}\hlkwb{AX}\hlstd{}\hlopt{,\ }\hlstd{KEEP\textunderscore CS\ \\
\hllin{68\ }}\hlstd{\ \ \ \ }\hlstd{}\hlkwa{MOV\ }\hlstd{}\hlkwb{DS}\hlstd{}\hlopt{,\ }\hlstd{}\hlkwb{AX\ }\\
\hllin{69\ }\hlstd{}\hlstd{\ \ \ \ }\hlstd{}\hlkwa{MOV\ }\hlstd{}\hlkwb{AH}\hlstd{}\hlopt{,\ }\hlstd{}\hlnum{25H\ }\\
\hllin{70\ }\hlstd{}\hlstd{\ \ \ \ }\hlstd{}\hlkwa{MOV\ }\hlstd{}\hlkwb{AL}\hlstd{}\hlopt{,\ }\hlstd{}\hlnum{0}\hlstd{}\hlkwb{BH\ }\\
\hllin{71\ }\hlstd{}\hlstd{\ \ \ \ }\hlstd{}\hlkwa{INT\ }\hlstd{}\hlnum{21H\ }\\
\hllin{72\ }\hlstd{}\hlstd{\ \ \ \ }\hlstd{}\hlkwa{POP\ }\hlstd{}\hlkwb{DS\ }\\
\hllin{73\ }\hlstd{}\hlstd{\ \ \ \ }\hlstd{}\hlkwa{MOV\ }\hlstd{}\hlkwb{AH}\hlstd{}\hlopt{,}\hlstd{}\hlnum{4}\hlstd{}\hlkwb{CH}\\
\hllin{74\ }\hlstd{}\hlstd{\ \ \ \ }\hlstd{}\hlkwa{INT\ }\hlstd{}\hlnum{21H}\\
\hllin{75\ }\hlstd{}\hlstd{\ \ \ \ }\hlstd{\\
\hllin{76\ }INTR\ }\hlkwa{PROC}\\
\hllin{77\ }\hlstd{}\hlstd{\ \ \ \ }\hlstd{}\hlkwa{MOV\ }\hlstd{FLAG}\hlopt{,\ }\hlstd{}\hlnum{1}\hlstd{\ \ \ \ \ \ \ \ }\hlnum{}\\
\hllin{78\ }\hlstd{}\hlstd{\ \ \ \ }\hlstd{}\hlkwa{XOR\ }\hlstd{SIG}\hlopt{,}\hlstd{}\hlnum{0}\hlstd{FFH\\
\hllin{79\ }}\hlstd{\ \ \ \ }\hlstd{}\hlkwa{MOV\ }\hlstd{}\hlkwb{AL}\hlstd{}\hlopt{,\ }\hlstd{SIG\\
\hllin{80\ }}\hlstd{\ \ \ \ }\hlstd{}\hlkwa{MOV\ }\hlstd{}\hlkwb{DX}\hlstd{}\hlopt{,\ }\hlstd{}\hlnum{280H}\\
\hllin{81\ }\hlstd{}\hlstd{\ \ \ \ }\hlstd{}\hlkwa{OUT\ }\hlstd{}\hlkwb{DX}\hlstd{}\hlopt{,}\hlstd{}\hlkwb{AL}\\
\hllin{82\ }\hlstd{}\hlstd{\ \ \ \ }\hlstd{}\hlkwa{MOV\ }\hlstd{}\hlkwb{AL}\hlstd{}\hlopt{,\ }\hlstd{}\hlnum{20H}\hlstd{\ \ \ \ \ \ \ \ }\hlnum{}\\
\hllin{83\ }\hlstd{}\hlstd{\ \ \ \ }\hlstd{}\hlkwa{OUT\ }\hlstd{}\hlnum{20H}\hlstd{}\hlopt{,\ }\hlstd{}\hlkwb{AL}\hlstd{\ \ \ \ \ \ }\hlkwb{}\\
\hllin{84\ }\hlstd{}\hlstd{\ \ \ \ }\hlstd{}\hlkwa{IRET}\hlstd{\ \ \ \ \ \ \ \ }\hlkwa{}\\
\hllin{85\ }\hlstd{INTR\ }\hlkwa{ENDP}\\
\hllin{86\ }\hlstd{}\hlstd{\ \ \ \ }\hlstd{\\
\hllin{87\ }MAIN\ }\hlkwa{ENDP}\\
\hllin{88\ }\hlstd{CODE\ }\hlkwa{ENDS}\\
\hllin{89\ }\hlstd{}\hlstd{\ \ \ \ }\hlstd{}\hlkwa{END\ }\hlstd{START}\\
\mbox{}
\normalfont
\normalsize

\subsection{任务三}
8255电路A口以方式0输出,C口也初始化成方式0输出且仅
用其最低两位:PC1接数码管位码输入端S1,PC0接位码输
入端S0。程序实现当A口输出字形“0”的段码时,C口输出
01H,第一个数码管显示“0”,当A口输出字形“1”的段码
时,C口输出02H,于是第二个数码管显示“1”。每一位显示
之后调用一段延时程序,选择恰当的延时程序,使“01”几乎
同时显示在两位数码管上。
\subsubsection{必做任务}
代码如下所示:
\noindent
\ttfamily
\hlstd{\hllin{01\ }DATAS\ }\hlkwa{SEGMENT}\\
\hllin{02\ }\hlstd{DATAS\ }\hlkwa{ENDS}\\
\hllin{03\ }\hlstd{\\
\hllin{04\ }STACKS\ }\hlkwa{SEGMENT}\\
\hllin{05\ }\hlstd{STACKS\ }\hlkwa{ENDS}\\
\hllin{06\ }\hlstd{\\
\hllin{07\ }CODES\ }\hlkwa{SEGMENT}\\
\hllin{08\ }\hlstd{}\hlstd{\ \ \ \ }\hlstd{}\hlkwa{ASSUME\ }\hlstd{}\hlkwb{CS}\hlstd{}\hlopt{:}\hlstd{CODES}\hlopt{,}\hlstd{}\hlkwb{DS}\hlstd{}\hlopt{:}\hlstd{DATAS}\hlopt{,}\hlstd{}\hlkwb{SS}\hlstd{}\hlopt{:}\hlstd{STACKS\\
\hllin{09\ }\\
\hllin{10\ }MAIN\ }\hlkwa{PROC\ FAR}\hlstd{\ \ \ }\hlkwa{}\\
\hllin{11\ }\hlstd{}\hlkwc{START:}\\
\hllin{12\ }\hlstd{}\hlstd{\ \ \ \ }\hlstd{}\hlkwa{MOV\ }\hlstd{}\hlkwb{AX}\hlstd{}\hlopt{,}\hlstd{DATAS\\
\hllin{13\ }}\hlstd{\ \ \ \ }\hlstd{}\hlkwa{MOV\ }\hlstd{}\hlkwb{DS}\hlstd{}\hlopt{,}\hlstd{}\hlkwb{AX}\\
\hllin{14\ }\hlstd{}\hlstd{\ \ \ \ }\hlstd{\\
\hllin{15\ }}\hlstd{\ \ \ \ }\hlstd{}\hlkwa{MOV\ }\hlstd{}\hlkwb{DX}\hlstd{}\hlopt{,\ }\hlstd{}\hlnum{283H}\\
\hllin{16\ }\hlstd{}\hlstd{\ \ \ \ }\hlstd{}\hlkwa{MOV\ }\hlstd{}\hlkwb{AL}\hlstd{}\hlopt{,\ }\hlstd{}\hlnum{10000000}\hlstd{B\ \\
\hllin{17\ }}\hlstd{\ \ \ \ }\hlstd{}\hlkwa{OUT\ }\hlstd{}\hlkwb{DX}\hlstd{}\hlopt{,\ }\hlstd{}\hlkwb{AL}\\
\hllin{18\ }\hlstd{}\hlstd{\ \ \ \ }\hlstd{}\\
\hllin{19\ }\hlkwc{NEXT:}\\
\hllin{20\ }\hlstd{}\hlstd{\ \ \ \ }\hlstd{}\hlkwa{MOV\ }\hlstd{}\hlkwb{DX}\hlstd{}\hlopt{,\ }\hlstd{}\hlnum{282H}\\
\hllin{21\ }\hlstd{}\hlstd{\ \ \ \ }\hlstd{}\hlkwa{MOV\ }\hlstd{}\hlkwb{AL}\hlstd{}\hlopt{,\ }\hlstd{}\hlnum{01H}\\
\hllin{22\ }\hlstd{}\hlstd{\ \ \ \ }\hlstd{}\hlkwa{OUT\ }\hlstd{}\hlkwb{DX}\hlstd{}\hlopt{,\ }\hlstd{}\hlkwb{AL}\hlstd{\ \ \ \ \ \ \ \ }\hlkwb{}\\
\hllin{23\ }\hlstd{}\hlstd{\ \ \ \ }\hlstd{}\hlkwa{MOV\ }\hlstd{}\hlkwb{DX}\hlstd{}\hlopt{,\ }\hlstd{}\hlnum{280H}\\
\hllin{24\ }\hlstd{}\hlstd{\ \ \ \ }\hlstd{}\hlkwa{MOV\ }\hlstd{}\hlkwb{AL}\hlstd{}\hlopt{,\ }\hlstd{}\hlnum{3}\hlstd{FH\\
\hllin{25\ }}\hlstd{\ \ \ \ }\hlstd{}\hlkwa{OUT\ }\hlstd{}\hlkwb{DX}\hlstd{}\hlopt{,\ }\hlstd{}\hlkwb{AL}\hlstd{\ \ \ \ \ \ \ \ }\hlkwb{}\\
\hllin{26\ }\hlstd{}\hlstd{\ \ \ \ }\hlstd{\\
\hllin{27\ }}\hlstd{\ \ \ \ }\hlstd{}\hlkwa{CALL\ }\hlstd{DELAY\\
\hllin{28\ }}\hlstd{\ \ \ \ }\hlstd{\\
\hllin{29\ }}\hlstd{\ \ \ \ }\hlstd{}\hlkwa{MOV\ }\hlstd{}\hlkwb{DX}\hlstd{}\hlopt{,\ }\hlstd{}\hlnum{280H}\\
\hllin{30\ }\hlstd{}\hlstd{\ \ \ \ }\hlstd{}\hlkwa{MOV\ }\hlstd{}\hlkwb{AL}\hlstd{}\hlopt{,\ }\hlstd{}\hlnum{06H}\\
\hllin{31\ }\hlstd{}\hlstd{\ \ \ \ }\hlstd{}\hlkwa{OUT\ }\hlstd{}\hlkwb{DX}\hlstd{}\hlopt{,\ }\hlstd{}\hlkwb{AL}\hlstd{\ \ \ \ \ \ \ \ }\hlkwb{}\\
\hllin{32\ }\hlstd{}\hlstd{\ \ \ \ }\hlstd{}\hlkwa{MOV\ }\hlstd{}\hlkwb{DX}\hlstd{}\hlopt{,\ }\hlstd{}\hlnum{282H}\\
\hllin{33\ }\hlstd{}\hlstd{\ \ \ \ }\hlstd{}\hlkwa{MOV\ }\hlstd{}\hlkwb{AL}\hlstd{}\hlopt{,\ }\hlstd{}\hlnum{02H}\\
\hllin{34\ }\hlstd{}\hlstd{\ \ \ \ }\hlstd{}\hlkwa{OUT\ }\hlstd{}\hlkwb{DX}\hlstd{}\hlopt{,\ }\hlstd{}\hlkwb{AL}\hlstd{\ \ \ \ \ \ \ \ }\hlkwb{}\\
\hllin{35\ }\hlstd{\\
\hllin{36\ }}\hlstd{\ \ \ \ }\hlstd{}\hlkwa{CALL\ }\hlstd{DELAY\\
\hllin{37\ }}\hlstd{\ \ \ \ }\hlstd{\\
\hllin{38\ }}\hlstd{\ \ \ \ }\hlstd{}\hlkwa{MOV\ }\hlstd{}\hlkwb{AH}\hlstd{}\hlopt{,\ }\hlstd{}\hlnum{1}\hlstd{\ \ \ \ \ \ \ \ }\hlnum{}\\
\hllin{39\ }\hlstd{}\hlstd{\ \ \ \ }\hlstd{}\hlkwa{INT\ }\hlstd{}\hlnum{16H}\\
\hllin{40\ }\hlstd{}\hlstd{\ \ \ \ }\hlstd{}\hlkwa{JZ\ }\hlstd{NEXT\\
\hllin{41\ }}\hlstd{\ \ \ \ }\hlstd{}\hlkwa{CMP\ }\hlstd{}\hlkwb{AL}\hlstd{}\hlopt{,\ }\hlstd{}\hlnum{20H}\hlstd{\ \ \ \ \ \ \ }\hlnum{}\\
\hllin{42\ }\hlstd{}\hlstd{\ \ \ \ }\hlstd{}\hlkwa{JZ\ }\hlstd{EXIT\\
\hllin{43\ }}\hlstd{\ \ \ \ }\hlstd{}\hlkwa{JMP\ }\hlstd{NEXT\\
\hllin{44\ }}\hlstd{\ \ \ \ \ \ }\hlstd{}\\
\hllin{45\ }\hlkwc{EXIT:}\\
\hllin{46\ }\hlstd{}\hlstd{\ \ \ \ }\hlstd{}\hlkwa{MOV\ }\hlstd{}\hlkwb{AH}\hlstd{}\hlopt{,}\hlstd{}\hlnum{4}\hlstd{}\hlkwb{CH}\\
\hllin{47\ }\hlstd{}\hlstd{\ \ \ \ }\hlstd{}\hlkwa{INT\ }\hlstd{}\hlnum{21H}\\
\hllin{48\ }\hlstd{}\hlstd{\ \ \ \ }\hlstd{\\
\hllin{49\ }DELAY\ }\hlkwa{PROC}\\
\hllin{50\ }\hlstd{}\hlstd{\ \ \ \ }\hlstd{}\hlkwa{MOV\ }\hlstd{}\hlkwb{CX}\hlstd{}\hlopt{,\ }\hlstd{}\hlnum{0}\hlstd{FFH}\\
\hllin{51\ }\hlkwc{D\textunderscore WAIT:}\hlstd{\ \ \ \ }\hlkwc{}\hlstd{}\hlkwa{LOOP\ }\hlstd{D\textunderscore WAIT\\
\hllin{52\ }}\hlstd{\ \ \ \ }\hlstd{}\hlkwa{RET}\\
\hllin{53\ }\hlstd{DELAY\ }\hlkwa{ENDP}\\
\hllin{54\ }\hlstd{\\
\hllin{55\ }MAIN\ }\hlkwa{ENDP}\\
\hllin{56\ }\hlstd{CODES\ }\hlkwa{ENDS}\\
\hllin{57\ }\hlstd{}\hlstd{\ \ \ \ }\hlstd{}\hlkwa{END\ }\hlstd{START}\\
\mbox{}
\normalfont
\normalsize

\subsubsection{选做任务一}
当程序运行后,从计算机键盘上输入两位十进制数,分别在
两个数码管上显示。若继续输入数字则更新显示。若发现输
入了非数字键则退回DOS,代码如下所示。
\noindent
\ttfamily
\hlstd{\hllin{01\ }DATAS\ }\hlkwa{SEGMENT}\\
\hllin{02\ }\hlstd{NUM\ }\hlkwa{DB\ }\hlstd{}\hlnum{3}\hlstd{FH}\hlopt{,}\hlstd{}\hlnum{06H}\hlstd{}\hlopt{,}\hlstd{}\hlnum{5}\hlstd{}\hlkwb{BH}\hlstd{}\hlopt{,}\hlstd{}\hlnum{4}\hlstd{FH}\hlopt{,}\hlstd{}\hlnum{66H}\hlstd{}\hlopt{,}\hlstd{}\hlnum{6}\hlstd{}\hlkwb{DH}\hlstd{}\hlopt{,}\hlstd{}\hlnum{7}\hlstd{}\hlkwb{DH}\hlstd{}\hlopt{,}\hlstd{}\hlnum{07H}\hlstd{}\hlopt{,}\hlstd{}\hlnum{7}\hlstd{FH}\hlopt{,}\hlstd{}\hlnum{6}\hlstd{FH\\
\hllin{03\ }DATAS\ }\hlkwa{ENDS}\\
\hllin{04\ }\hlstd{\\
\hllin{05\ }STACKS\ }\hlkwa{SEGMENT}\\
\hllin{06\ }\hlstd{}\hlstd{\ \ \ }\hlstd{}\hlkwa{DB\ }\hlstd{}\hlnum{100\ }\hlstd{DUP}\hlopt{(}\hlstd{?}\hlopt{)}\\
\hllin{07\ }\hlstd{STACKS\ }\hlkwa{ENDS}\\
\hllin{08\ }\hlstd{\\
\hllin{09\ }CODES\ }\hlkwa{SEGMENT}\\
\hllin{10\ }\hlstd{}\hlstd{\ \ \ \ }\hlstd{}\hlkwa{ASSUME\ }\hlstd{}\hlkwb{CS}\hlstd{}\hlopt{:}\hlstd{CODES}\hlopt{,}\hlstd{}\hlkwb{DS}\hlstd{}\hlopt{:}\hlstd{DATAS}\hlopt{,}\hlstd{}\hlkwb{SS}\hlstd{}\hlopt{:}\hlstd{STACKS}\\
\hllin{11\ }\hlkwc{START:}\\
\hllin{12\ }\hlstd{}\hlstd{\ \ \ \ }\hlstd{}\hlkwa{MOV\ }\hlstd{}\hlkwb{AX}\hlstd{}\hlopt{,}\hlstd{DATAS\\
\hllin{13\ }}\hlstd{\ \ \ \ }\hlstd{}\hlkwa{MOV\ }\hlstd{}\hlkwb{DS}\hlstd{}\hlopt{,}\hlstd{}\hlkwb{AX}\\
\hllin{14\ }\hlstd{}\hlstd{\ \ \ \ }\hlstd{\\
\hllin{15\ }}\hlstd{\ \ \ \ }\hlstd{}\hlkwa{MOV\ }\hlstd{}\hlkwb{DX}\hlstd{}\hlopt{,}\hlstd{}\hlnum{283H}\hlstd{\ \ \ \ \ \ }\hlnum{}\\
\hllin{16\ }\hlstd{}\hlstd{\ \ \ \ }\hlstd{}\hlkwa{MOV\ }\hlstd{}\hlkwb{AL}\hlstd{}\hlopt{,}\hlstd{}\hlnum{10000000}\hlstd{B\\
\hllin{17\ }}\hlstd{\ \ \ \ }\hlstd{}\hlkwa{OUT\ }\hlstd{}\hlkwb{DX}\hlstd{}\hlopt{,}\hlstd{}\hlkwb{AL}\\
\hllin{18\ }\hlstd{\\
\hllin{19\ }}\hlstd{\ \ \ \ }\hlstd{}\hlkwa{MOV\ }\hlstd{}\hlkwb{BX}\hlstd{}\hlopt{,}\hlstd{}\hlnum{0}\\
\hllin{20\ }\hlstd{}\hlstd{\ \ \ \ }\hlstd{}\hlkwa{MOV\ }\hlstd{}\hlkwb{CX}\hlstd{}\hlopt{,}\hlstd{}\hlnum{0}\\
\hllin{21\ }\hlstd{}\hlkwc{MAIN:}\\
\hllin{22\ }\hlstd{}\hlstd{\ \ \ \ }\hlstd{}\hlkwa{MOV\ }\hlstd{}\hlkwb{AH}\hlstd{}\hlopt{,}\hlstd{}\hlnum{0}\hlstd{}\hlkwb{BH}\hlstd{\ \ \ \ \ \ \ }\hlkwb{}\\
\hllin{23\ }\hlstd{}\hlstd{\ \ \ \ }\hlstd{}\hlkwa{INT\ }\hlstd{}\hlnum{21H}\\
\hllin{24\ }\hlstd{}\hlstd{\ \ \ \ }\hlstd{}\hlkwa{INC\ }\hlstd{}\hlkwb{AL}\\
\hllin{25\ }\hlstd{}\hlstd{\ \ \ \ }\hlstd{}\hlkwa{JNE\ }\hlstd{NEXT\\
\hllin{26\ }}\hlstd{\ \ \ \ }\hlstd{\\
\hllin{27\ }}\hlstd{\ \ \ \ }\hlstd{}\hlkwa{MOV\ }\hlstd{}\hlkwb{AH}\hlstd{}\hlopt{,}\hlstd{}\hlnum{1}\hlstd{\ \ \ \ \ \ \ \ }\hlnum{}\\
\hllin{28\ }\hlstd{}\hlstd{\ \ \ \ }\hlstd{}\hlkwa{INT\ }\hlstd{}\hlnum{21H}\\
\hllin{29\ }\hlstd{}\hlstd{\ \ \ \ }\hlstd{}\hlkwa{CMP\ }\hlstd{}\hlkwb{AL}\hlstd{}\hlopt{,}\hlstd{}\hlstr{'0'}\hlstd{\\
\hllin{30\ }}\hlstd{\ \ \ \ }\hlstd{}\hlkwa{JS\ }\hlstd{EXIT\\
\hllin{31\ }}\hlstd{\ \ \ \ }\hlstd{}\hlkwa{CMP\ }\hlstd{}\hlkwb{AL}\hlstd{}\hlopt{,}\hlstd{}\hlstr{'9'}\hlstd{}\hlopt{+}\hlstd{}\hlnum{1}\\
\hllin{32\ }\hlstd{}\hlstd{\ \ \ \ }\hlstd{}\hlkwa{JNS\ }\hlstd{EXIT\\
\hllin{33\ }}\hlstd{\ \ \ \ }\hlstd{}\hlkwa{SUB\ }\hlstd{}\hlkwb{AL}\hlstd{}\hlopt{,}\hlstd{}\hlnum{30H}\\
\hllin{34\ }\hlstd{}\hlstd{\ \ \ \ }\hlstd{}\hlkwa{MOV\ }\hlstd{}\hlkwb{BL}\hlstd{}\hlopt{,}\hlstd{}\hlkwb{AL}\\
\hllin{35\ }\hlstd{}\hlstd{\ \ \ }\hlstd{\\
\hllin{36\ }}\hlstd{\ \ \ \ }\hlstd{}\hlkwa{MOV\ }\hlstd{}\hlkwb{AH}\hlstd{}\hlopt{,}\hlstd{}\hlnum{1}\\
\hllin{37\ }\hlstd{}\hlstd{\ \ \ \ }\hlstd{}\hlkwa{INT\ }\hlstd{}\hlnum{21H}\\
\hllin{38\ }\hlstd{}\hlstd{\ \ \ \ }\hlstd{}\hlkwa{CMP\ }\hlstd{}\hlkwb{AL}\hlstd{}\hlopt{,}\hlstd{}\hlstr{'0'}\hlstd{\\
\hllin{39\ }}\hlstd{\ \ \ \ }\hlstd{}\hlkwa{JS\ }\hlstd{EXIT\\
\hllin{40\ }}\hlstd{\ \ \ \ }\hlstd{}\hlkwa{CMP\ }\hlstd{}\hlkwb{AL}\hlstd{}\hlopt{,}\hlstd{}\hlstr{'9'}\hlstd{}\hlopt{+}\hlstd{}\hlnum{1}\\
\hllin{41\ }\hlstd{}\hlstd{\ \ \ \ }\hlstd{}\hlkwa{JNS\ }\hlstd{EXIT\ \\
\hllin{42\ }}\hlstd{\ \ \ \ }\hlstd{}\hlkwa{SUB\ }\hlstd{}\hlkwb{AL}\hlstd{}\hlopt{,}\hlstd{}\hlnum{30H}\\
\hllin{43\ }\hlstd{}\hlstd{\ \ \ \ }\hlstd{}\hlkwa{MOV\ }\hlstd{}\hlkwb{CL}\hlstd{}\hlopt{,}\hlstd{}\hlkwb{AL}\\
\hllin{44\ }\hlstd{}\\
\hllin{45\ }\hlkwa{CALL\ }\hlstd{DELAY}\hlstd{\ \ }\hlstd{}\\
\hllin{46\ }\hlkwa{CALL\ }\hlstd{DELAY}\hlstd{\ \ }\hlstd{}\\
\hllin{47\ }\hlkwa{CALL\ }\hlstd{DELAY}\hlstd{\ \ }\hlstd{}\\
\hllin{48\ }\\
\hllin{49\ }\hlkwc{NEXT:}\\
\hllin{50\ }\hlstd{\\
\hllin{51\ }\\
\hllin{52\ }}\hlstd{\ \ \ \ }\hlstd{}\hlkwa{MOV\ }\hlstd{}\hlkwb{DX}\hlstd{}\hlopt{,}\hlstd{}\hlnum{280H}\\
\hllin{53\ }\hlstd{}\hlstd{\ \ \ \ }\hlstd{}\hlkwa{PUSH\ }\hlstd{}\hlkwb{BX}\\
\hllin{54\ }\hlstd{}\hlstd{\ \ \ \ }\hlstd{}\hlkwa{MOV\ }\hlstd{}\hlkwb{BX}\hlstd{}\hlopt{,}\hlstd{}\hlkwb{CX}\\
\hllin{55\ }\hlstd{}\hlstd{\ \ \ \ }\hlstd{}\hlkwa{MOV\ }\hlstd{}\hlkwb{AL}\hlstd{}\hlopt{,{[}}\hlstd{NUM}\hlopt{+}\hlstd{}\hlkwb{BX}\hlstd{}\hlopt{{]}}\\
\hllin{56\ }\hlstd{}\hlstd{\ \ \ \ }\hlstd{}\hlkwa{OUT\ }\hlstd{}\hlkwb{DX}\hlstd{}\hlopt{,}\hlstd{}\hlkwb{AL}\hlstd{\ \ \ \ \ \ }\hlkwb{}\\
\hllin{57\ }\hlstd{}\hlstd{\ \ \ \ }\hlstd{}\hlkwa{MOV\ }\hlstd{}\hlkwb{DX}\hlstd{}\hlopt{,}\hlstd{}\hlnum{282H}\\
\hllin{58\ }\hlstd{}\hlstd{\ \ \ \ }\hlstd{}\hlkwa{MOV\ }\hlstd{}\hlkwb{AL}\hlstd{}\hlopt{,}\hlstd{}\hlnum{01H}\\
\hllin{59\ }\hlstd{}\hlstd{\ \ \ \ }\hlstd{}\hlkwa{OUT\ }\hlstd{}\hlkwb{DX}\hlstd{}\hlopt{,}\hlstd{}\hlkwb{AL}\\
\hllin{60\ }\hlstd{}\hlstd{\ \ \ \ }\hlstd{}\hlkwa{POP\ }\hlstd{}\hlkwb{BX}\\
\hllin{61\ }\hlstd{}\hlstd{\ \ \ \ }\hlstd{\\
\hllin{62\ }}\hlstd{\ \ \ \ }\hlstd{}\hlkwa{CALL\ }\hlstd{DELAY}\hlstd{\ \ \ \ \ }\hlstd{\\
\hllin{63\ }\\
\hllin{64\ }\\
\hllin{65\ }}\hlstd{\ \ \ \ }\hlstd{}\hlkwa{MOV\ }\hlstd{}\hlkwb{DX}\hlstd{}\hlopt{,}\hlstd{}\hlnum{282H}\\
\hllin{66\ }\hlstd{}\hlstd{\ \ \ \ }\hlstd{}\hlkwa{MOV\ }\hlstd{}\hlkwb{AL}\hlstd{}\hlopt{,}\hlstd{}\hlnum{00H}\\
\hllin{67\ }\hlstd{}\hlstd{\ \ \ \ }\hlstd{}\hlkwa{OUT\ }\hlstd{}\hlkwb{DX}\hlstd{}\hlopt{,}\hlstd{}\hlkwb{AL}\hlstd{\ \ \ \ }\hlkwb{}\\
\hllin{68\ }\hlstd{\\
\hllin{69\ }}\hlstd{\ \ \ \ }\hlstd{}\hlkwa{MOV\ }\hlstd{}\hlkwb{DX}\hlstd{}\hlopt{,}\hlstd{}\hlnum{280H}\\
\hllin{70\ }\hlstd{}\hlstd{\ \ \ \ }\hlstd{}\hlkwa{MOV\ }\hlstd{}\hlkwb{AL}\hlstd{}\hlopt{,{[}}\hlstd{NUM}\hlopt{+}\hlstd{}\hlkwb{BX}\hlstd{}\hlopt{{]}}\\
\hllin{71\ }\hlstd{}\hlstd{\ \ \ \ }\hlstd{}\hlkwa{OUT\ }\hlstd{}\hlkwb{DX}\hlstd{}\hlopt{,}\hlstd{}\hlkwb{AL}\\
\hllin{72\ }\hlstd{}\hlstd{\ \ \ \ }\hlstd{}\hlkwa{MOV\ }\hlstd{}\hlkwb{DX}\hlstd{}\hlopt{,}\hlstd{}\hlnum{282H}\\
\hllin{73\ }\hlstd{}\hlstd{\ \ \ \ }\hlstd{}\hlkwa{MOV\ }\hlstd{}\hlkwb{AL}\hlstd{}\hlopt{,}\hlstd{}\hlnum{02H}\\
\hllin{74\ }\hlstd{}\hlstd{\ \ \ \ }\hlstd{}\hlkwa{OUT\ }\hlstd{}\hlkwb{DX}\hlstd{}\hlopt{,}\hlstd{}\hlkwb{AL}\\
\hllin{75\ }\hlstd{}\hlstd{\ \ \ \ }\hlstd{\\
\hllin{76\ }}\hlstd{\ \ \ \ }\hlstd{}\hlkwa{CALL\ }\hlstd{DELAY}\hlstd{\ \ \ \ }\hlstd{\\
\hllin{77\ }\\
\hllin{78\ }\\
\hllin{79\ }}\hlstd{\ \ \ \ }\hlstd{}\hlkwa{MOV\ }\hlstd{}\hlkwb{DX}\hlstd{}\hlopt{,}\hlstd{}\hlnum{282H}\\
\hllin{80\ }\hlstd{}\hlstd{\ \ \ \ }\hlstd{}\hlkwa{MOV\ }\hlstd{}\hlkwb{AL}\hlstd{}\hlopt{,}\hlstd{}\hlnum{00H}\\
\hllin{81\ }\hlstd{}\hlstd{\ \ \ \ }\hlstd{}\hlkwa{OUT\ }\hlstd{}\hlkwb{DX}\hlstd{}\hlopt{,}\hlstd{}\hlkwb{AL}\hlstd{\ \ \ \ \ }\hlkwb{}\\
\hllin{82\ }\hlstd{}\hlstd{\ \ \ \ }\hlstd{}\hlkwa{JMP\ }\hlstd{MAIN}\hlstd{\ \ }\hlstd{\\
\hllin{83\ }}\hlstd{\ \ }\hlstd{}\\
\hllin{84\ }\hlkwc{EXIT:}\hlstd{\ \ }\hlkwc{}\\
\hllin{85\ }\hlstd{}\hlstd{\ \ \ \ }\hlstd{}\hlkwa{MOV\ }\hlstd{}\hlkwb{AH}\hlstd{}\hlopt{,}\hlstd{}\hlnum{4}\hlstd{}\hlkwb{CH}\\
\hllin{86\ }\hlstd{}\hlstd{\ \ \ \ }\hlstd{}\hlkwa{INT\ }\hlstd{}\hlnum{21H}\\
\hllin{87\ }\hlstd{\\
\hllin{88\ }\\
\hllin{89\ }DELAY}\hlstd{\ \ }\hlstd{}\hlkwa{PROC}\hlstd{\ \ }\hlkwa{}\\
\hllin{90\ }\hlstd{}\hlstd{\ \ \ \ \ \ \ }\hlstd{}\hlkwa{PUSH}\hlstd{\ \ \ \ }\hlkwa{}\hlstd{}\hlkwb{CX}\\
\hllin{91\ }\hlstd{}\hlstd{\ \ \ \ \ \ \ }\hlstd{}\hlkwa{PUSH}\hlstd{\ \ \ \ }\hlkwa{}\hlstd{}\hlkwb{AX}\\
\hllin{92\ }\hlstd{}\hlstd{\ \ \ \ \ \ \ }\hlstd{}\hlkwa{MOV}\hlstd{\ \ \ \ }\hlkwa{}\hlstd{}\hlkwb{AX}\hlstd{}\hlopt{,}\hlstd{}\hlnum{000}\hlstd{FH}\\
\hllin{93\ }\hlkwc{X1:}\hlstd{\ \ \ \ }\hlkwc{}\hlstd{}\hlkwa{MOV}\hlstd{\ \ \ \ }\hlkwa{}\hlstd{}\hlkwb{CX}\hlstd{}\hlopt{,}\hlstd{}\hlnum{0}\hlstd{FFFH}\\
\hllin{94\ }\hlkwc{X2:}\hlstd{\ \ \ \ }\hlkwc{}\hlstd{}\hlkwa{DEC}\hlstd{\ \ \ \ }\hlkwa{}\hlstd{}\hlkwb{CX}\\
\hllin{95\ }\hlstd{}\hlstd{\ \ \ \ \ \ \ }\hlstd{}\hlkwa{JNE}\hlstd{\ \ \ \ }\hlkwa{}\hlstd{X2\\
\hllin{96\ }}\hlstd{\ \ \ \ \ \ \ }\hlstd{}\hlkwa{DEC}\hlstd{\ \ \ \ }\hlkwa{}\hlstd{}\hlkwb{AX}\\
\hllin{97\ }\hlstd{}\hlstd{\ \ \ \ \ \ \ }\hlstd{}\hlkwa{JNE}\hlstd{\ \ \ \ }\hlkwa{}\hlstd{X1\\
\hllin{98\ }}\hlstd{\ \ \ \ \ \ \ }\hlstd{}\hlkwa{POP}\hlstd{\ \ \ \ }\hlkwa{}\hlstd{}\hlkwb{AX}\\
\hllin{99\ }\hlstd{}\hlstd{\ \ \ \ \ \ \ }\hlstd{}\hlkwa{POP}\hlstd{\ \ \ \ }\hlkwa{}\hlstd{}\hlkwb{CX}\\
\hllin{100\ }\hlstd{}\hlstd{\ \ \ \ \ \ \ }\hlstd{}\hlkwa{RET}\\
\hllin{101\ }\hlstd{DELAY}\hlstd{\ \ }\hlstd{}\hlkwa{ENDP}\hlstd{\ \ }\hlkwa{}\\
\hllin{102\ }\hlstd{\\
\hllin{103\ }}\hlstd{\ \ \ \ }\hlstd{\\
\hllin{104\ }CODES\ }\hlkwa{ENDS}\\
\hllin{105\ }\hlstd{}\hlstd{\ \ \ \ }\hlstd{}\hlkwa{END\ }\hlstd{START}\\
\hllin{106\ }\\
\hllin{107\ }\\
\hllin{108\ }\\
\hllin{109\ }\\
\mbox{}
\normalfont
\normalsize

\subsubsection{选做任务二}
使用TPC实验台上的8253定时计数电路来代替前面的软件延
时。8253定时器自动重复工作,每工作一个周期发出一次中
断请求信号,在中断服务程序里同步更换段码和位码,实现
扫描显示,代码如下所示。
\noindent
\ttfamily
\hlstd{\hllin{01\ }DATA\ }\hlkwa{SEGMENT}\\
\hllin{02\ }\hlstd{KEEPIP\ }\hlkwa{DW\ }\hlstd{}\hlnum{0}\\
\hllin{03\ }\hlstd{KEEPCS\ }\hlkwa{DW\ }\hlstd{}\hlnum{0}\\
\hllin{04\ }\hlstd{DATA\ }\hlkwa{ENDS}\\
\hllin{05\ }\hlstd{\\
\hllin{06\ }STACK\ }\hlkwa{SEGMENT}\\
\hllin{07\ }\hlstd{}\hlstd{\ \ \ \ }\hlstd{}\hlkwa{DB\ }\hlstd{}\hlnum{100\ }\hlstd{DUP}\hlopt{(}\hlstd{?}\hlopt{)}\\
\hllin{08\ }\hlstd{STACK\ }\hlkwa{ENDS}\\
\hllin{09\ }\hlstd{\\
\hllin{10\ }CODE\ }\hlkwa{SEGMENT}\\
\hllin{11\ }\hlstd{}\hlstd{\ \ \ \ }\hlstd{}\hlkwa{ASSUME\ }\hlstd{}\hlkwb{CS}\hlstd{}\hlopt{:}\hlstd{CODE}\hlopt{,}\hlstd{}\hlkwb{DS}\hlstd{}\hlopt{:}\hlstd{DATA}\hlopt{,}\hlstd{}\hlkwb{ES}\hlstd{}\hlopt{:}\hlstd{DATA}\hlopt{,}\hlstd{}\hlkwb{SS}\hlstd{}\hlopt{:}\hlstd{STACK}\\
\hllin{12\ }\hlkwc{START:\ }\\
\hllin{13\ }\hlstd{\\
\hllin{14\ }}\hlstd{\ \ \ \ }\hlstd{}\hlkwa{MOV\ }\hlstd{}\hlkwb{AX}\hlstd{}\hlopt{,}\hlstd{DATA}\hlstd{\ \ \ \ }\hlstd{\\
\hllin{15\ }}\hlstd{\ \ \ \ }\hlstd{}\hlkwa{MOV\ }\hlstd{}\hlkwb{DS}\hlstd{}\hlopt{,}\hlstd{}\hlkwb{AX}\\
\hllin{16\ }\hlstd{}\hlstd{\ \ \ \ }\hlstd{}\hlkwa{MOV\ }\hlstd{}\hlkwb{ES}\hlstd{}\hlopt{,}\hlstd{}\hlkwb{AX}\\
\hllin{17\ }\hlstd{\\
\hllin{18\ }}\hlstd{\ \ \ \ }\hlstd{}\hlkwa{MOV\ }\hlstd{}\hlkwb{DX}\hlstd{}\hlopt{,}\hlstd{}\hlnum{293H}\\
\hllin{19\ }\hlstd{}\hlstd{\ \ \ \ }\hlstd{}\hlkwa{MOV\ }\hlstd{}\hlkwb{AL}\hlstd{}\hlopt{,}\hlstd{}\hlnum{00110111}\hlstd{B\\
\hllin{20\ }}\hlstd{\ \ \ \ }\hlstd{}\hlkwa{OUT\ }\hlstd{}\hlkwb{DX}\hlstd{}\hlopt{,}\hlstd{}\hlkwb{AL}\\
\hllin{21\ }\hlstd{}\hlstd{\ \ \ \ }\hlstd{\\
\hllin{22\ }}\hlstd{\ \ \ \ }\hlstd{}\hlkwa{MOV\ }\hlstd{}\hlkwb{DX}\hlstd{}\hlopt{,}\hlstd{}\hlnum{290H}\\
\hllin{23\ }\hlstd{}\hlstd{\ \ \ \ }\hlstd{}\hlkwa{XOR\ }\hlstd{}\hlkwb{AL}\hlstd{}\hlopt{,}\hlstd{}\hlkwb{AL}\\
\hllin{24\ }\hlstd{}\hlstd{\ \ \ \ }\hlstd{}\hlkwa{OUT\ }\hlstd{}\hlkwb{DX}\hlstd{}\hlopt{,}\hlstd{}\hlkwb{AL}\\
\hllin{25\ }\hlstd{}\hlstd{\ \ \ \ }\hlstd{}\hlkwa{MOV\ }\hlstd{}\hlkwb{AL}\hlstd{}\hlopt{,}\hlstd{}\hlnum{50}\\
\hllin{26\ }\hlstd{}\hlstd{\ \ \ \ }\hlstd{}\hlkwa{OUT\ }\hlstd{}\hlkwb{DX}\hlstd{}\hlopt{,}\hlstd{}\hlkwb{AL}\\
\hllin{27\ }\hlstd{\\
\hllin{28\ }}\hlstd{\ \ \ \ }\hlstd{}\hlkwa{MOV\ }\hlstd{}\hlkwb{DX}\hlstd{}\hlopt{,}\hlstd{}\hlnum{283H}\hlstd{\ \ \ \ \ }\hlnum{}\\
\hllin{29\ }\hlstd{}\hlstd{\ \ \ \ }\hlstd{}\hlkwa{MOV\ }\hlstd{}\hlkwb{AL}\hlstd{}\hlopt{,}\hlstd{}\hlnum{10000000}\hlstd{B\\
\hllin{30\ }}\hlstd{\ \ \ \ }\hlstd{}\hlkwa{OUT\ }\hlstd{}\hlkwb{DX}\hlstd{}\hlopt{,}\hlstd{}\hlkwb{AL}\hlstd{\ \ \ \ \ \ \ }\hlkwb{}\\
\hllin{31\ }\hlstd{\\
\hllin{32\ }\\
\hllin{33\ }}\hlstd{\ \ \ \ }\hlstd{}\hlkwa{MOV\ }\hlstd{}\hlkwb{AH}\hlstd{}\hlopt{,}\hlstd{}\hlnum{35H}\hlstd{\ \ \ \ }\hlnum{}\\
\hllin{34\ }\hlstd{}\hlstd{\ \ \ \ }\hlstd{}\hlkwa{MOV\ }\hlstd{}\hlkwb{AL}\hlstd{}\hlopt{,}\hlstd{}\hlnum{0}\hlstd{}\hlkwb{BH}\\
\hllin{35\ }\hlstd{}\hlstd{\ \ \ \ }\hlstd{}\hlkwa{INT\ }\hlstd{}\hlnum{21H}\\
\hllin{36\ }\hlstd{}\hlstd{\ \ \ \ }\hlstd{}\hlkwa{MOV\ }\hlstd{KEEPIP}\hlopt{,}\hlstd{}\hlkwb{BX\ }\\
\hllin{37\ }\hlstd{}\hlstd{\ \ \ \ }\hlstd{}\hlkwa{MOV\ }\hlstd{KEEPCS}\hlopt{,}\hlstd{}\hlkwb{ES}\\
\hllin{38\ }\hlstd{}\hlstd{\ \ \ \ \ \ \ \ }\hlstd{\\
\hllin{39\ }}\hlstd{\ \ \ \ }\hlstd{}\hlkwa{PUSH\ }\hlstd{}\hlkwb{DS}\hlstd{\ \ \ \ \ \ \ }\hlkwb{}\\
\hllin{40\ }\hlstd{}\hlstd{\ \ \ \ }\hlstd{}\hlkwa{MOV\ }\hlstd{}\hlkwb{DX}\hlstd{}\hlopt{,}\hlstd{}\hlkwb{OFFSET\ }\hlstd{INTR\\
\hllin{41\ }}\hlstd{\ \ \ \ }\hlstd{}\hlkwa{MOV\ }\hlstd{}\hlkwb{AX}\hlstd{}\hlopt{,}\hlstd{}\hlkwa{SEG\ }\hlstd{INTR\\
\hllin{42\ }}\hlstd{\ \ \ \ }\hlstd{}\hlkwa{MOV\ }\hlstd{}\hlkwb{DS}\hlstd{}\hlopt{,}\hlstd{}\hlkwb{AX}\\
\hllin{43\ }\hlstd{}\hlstd{\ \ \ \ }\hlstd{}\hlkwa{MOV\ }\hlstd{}\hlkwb{AH}\hlstd{}\hlopt{,}\hlstd{}\hlnum{25H}\\
\hllin{44\ }\hlstd{}\hlstd{\ \ \ \ }\hlstd{}\hlkwa{MOV\ }\hlstd{}\hlkwb{AL}\hlstd{}\hlopt{,}\hlstd{}\hlnum{0}\hlstd{}\hlkwb{BH}\\
\hllin{45\ }\hlstd{}\hlstd{\ \ \ \ }\hlstd{}\hlkwa{INT\ }\hlstd{}\hlnum{21H}\\
\hllin{46\ }\hlstd{}\hlstd{\ \ \ \ }\hlstd{}\hlkwa{POP\ }\hlstd{}\hlkwb{DS}\\
\hllin{47\ }\hlstd{}\hlstd{\ \ \ \ \ \ \ \ }\hlstd{\\
\hllin{48\ }\\
\hllin{49\ }}\hlstd{\ \ \ \ }\hlstd{}\hlkwa{IN\ }\hlstd{}\hlkwb{AL}\hlstd{}\hlopt{,}\hlstd{}\hlnum{21H}\\
\hllin{50\ }\hlstd{}\hlstd{\ \ \ \ }\hlstd{}\hlkwa{AND\ }\hlstd{}\hlkwb{AL}\hlstd{}\hlopt{,}\hlstd{}\hlnum{011110111}\hlstd{B\\
\hllin{51\ }}\hlstd{\ \ \ \ }\hlstd{}\hlkwa{OUT\ }\hlstd{}\hlnum{21H}\hlstd{}\hlopt{,}\hlstd{}\hlkwb{AL}\hlstd{\ \ \ \ \ \ \ \ \ \ \ }\hlkwb{}\\
\hllin{52\ }\hlstd{}\hlstd{\ \ \ \ }\hlstd{\\
\hllin{53\ }}\hlstd{\ \ \ \ }\hlstd{}\hlkwa{MOV\ }\hlstd{}\hlkwb{BL}\hlstd{}\hlopt{,}\hlstd{}\hlnum{0}\\
\hllin{54\ }\hlstd{}\hlkwc{MAIN:}\hlstd{\ \ \ \ \ \ \ \ \ \ \ \ \ \ \ \ \ }\hlkwc{}\\
\hllin{55\ }\hlstd{}\hlstd{\ \ \ \ }\hlstd{}\hlkwa{HLT}\\
\hllin{56\ }\hlstd{}\hlstd{\ \ \ \ }\hlstd{}\hlkwa{MOV\ }\hlstd{}\hlkwb{AH}\hlstd{}\hlopt{,}\hlstd{}\hlnum{1}\\
\hllin{57\ }\hlstd{}\hlstd{\ \ \ \ }\hlstd{}\hlkwa{INT\ }\hlstd{}\hlnum{16H}\hlstd{\ \ \ \ \ \ \ \ \ \ \ }\hlnum{}\\
\hllin{58\ }\hlstd{}\hlstd{\ \ \ \ }\hlstd{}\hlkwa{JNZ\ }\hlstd{EXIT\\
\hllin{59\ }}\hlstd{\ \ \ \ }\hlstd{}\hlkwa{JMP\ }\hlstd{MAIN}\\
\hllin{60\ }\\
\hllin{61\ }\hlkwc{EXIT:}\\
\hllin{62\ }\hlstd{\\
\hllin{63\ }}\hlstd{\ \ \ \ }\hlstd{}\hlkwa{IN}\hlstd{\ \ }\hlkwa{}\hlstd{}\hlkwb{AL}\hlstd{}\hlopt{,}\hlstd{}\hlnum{21H}\\
\hllin{64\ }\hlstd{}\hlstd{\ \ \ \ }\hlstd{}\hlkwa{OR}\hlstd{\ \ }\hlkwa{}\hlstd{}\hlkwb{AL}\hlstd{}\hlopt{,}\hlstd{}\hlnum{00001000}\hlstd{B\\
\hllin{65\ }}\hlstd{\ \ \ \ }\hlstd{}\hlkwa{OUT\ }\hlstd{}\hlnum{21H}\hlstd{}\hlopt{,}\hlstd{}\hlkwb{AL}\\
\hllin{66\ }\hlstd{}\hlstd{\ \ \ \ \ \ \ \ }\hlstd{\\
\hllin{67\ }\\
\hllin{68\ }}\hlstd{\ \ \ \ }\hlstd{}\hlkwa{PUSH\ }\hlstd{}\hlkwb{DS}\\
\hllin{69\ }\hlstd{}\hlstd{\ \ \ \ }\hlstd{}\hlkwa{MOV\ }\hlstd{}\hlkwb{DX}\hlstd{}\hlopt{,}\hlstd{KEEPIP\\
\hllin{70\ }}\hlstd{\ \ \ \ }\hlstd{}\hlkwa{MOV\ }\hlstd{}\hlkwb{AX}\hlstd{}\hlopt{,}\hlstd{KEEPCS\\
\hllin{71\ }}\hlstd{\ \ \ \ }\hlstd{}\hlkwa{MOV\ }\hlstd{}\hlkwb{DS}\hlstd{}\hlopt{,}\hlstd{}\hlkwb{AX}\\
\hllin{72\ }\hlstd{}\hlstd{\ \ \ \ }\hlstd{}\hlkwa{MOV\ }\hlstd{}\hlkwb{AH}\hlstd{}\hlopt{,}\hlstd{}\hlnum{25H}\\
\hllin{73\ }\hlstd{}\hlstd{\ \ \ \ }\hlstd{}\hlkwa{MOV\ }\hlstd{}\hlkwb{AL}\hlstd{}\hlopt{,}\hlstd{}\hlnum{0}\hlstd{}\hlkwb{BH}\\
\hllin{74\ }\hlstd{}\hlstd{\ \ \ \ }\hlstd{}\hlkwa{INT\ }\hlstd{}\hlnum{21H}\\
\hllin{75\ }\hlstd{}\hlstd{\ \ \ \ }\hlstd{}\hlkwa{POP\ }\hlstd{}\hlkwb{DS}\\
\hllin{76\ }\hlstd{}\hlstd{\ \ \ \ }\hlstd{\\
\hllin{77\ }}\hlstd{\ \ \ \ }\hlstd{}\hlkwa{MOV\ }\hlstd{}\hlkwb{AH}\hlstd{}\hlopt{,}\hlstd{}\hlnum{4}\hlstd{}\hlkwb{CH}\hlstd{\ \ \ \ \ \ \ \ }\hlkwb{}\\
\hllin{78\ }\hlstd{}\hlstd{\ \ \ \ }\hlstd{}\hlkwa{INT\ }\hlstd{}\hlnum{21H}\\
\hllin{79\ }\hlstd{\\
\hllin{80\ }\\
\hllin{81\ }\\
\hllin{82\ }INTR\ }\hlkwa{PROC}\\
\hllin{83\ }\hlstd{\\
\hllin{84\ }\\
\hllin{85\ }}\hlstd{\ \ \ \ }\hlstd{}\hlkwa{MOV\ }\hlstd{}\hlkwb{DX}\hlstd{}\hlopt{,}\hlstd{}\hlnum{282H}\\
\hllin{86\ }\hlstd{}\hlstd{\ \ \ \ }\hlstd{}\hlkwa{MOV\ }\hlstd{}\hlkwb{AL}\hlstd{}\hlopt{,}\hlstd{}\hlnum{00H}\\
\hllin{87\ }\hlstd{}\hlstd{\ \ \ \ }\hlstd{}\hlkwa{OUT\ }\hlstd{}\hlkwb{DX}\hlstd{}\hlopt{,}\hlstd{}\hlkwb{AL}\\
\hllin{88\ }\hlstd{\\
\hllin{89\ }}\hlstd{\ \ \ \ }\hlstd{}\hlkwa{CMP\ }\hlstd{}\hlkwb{BL}\hlstd{}\hlopt{,}\hlstd{}\hlnum{0}\\
\hllin{90\ }\hlstd{}\hlstd{\ \ \ \ }\hlstd{}\hlkwa{JNZ\ }\hlstd{OUT1}\\
\hllin{91\ }\hlkwc{OUT0:}\\
\hllin{92\ }\hlstd{}\hlstd{\ \ \ \ }\hlstd{}\hlkwa{MOV\ }\hlstd{}\hlkwb{DX}\hlstd{}\hlopt{,}\hlstd{}\hlnum{280H}\\
\hllin{93\ }\hlstd{}\hlstd{\ \ \ \ }\hlstd{}\hlkwa{MOV\ }\hlstd{}\hlkwb{AL}\hlstd{}\hlopt{,}\hlstd{}\hlnum{3}\hlstd{FH\\
\hllin{94\ }}\hlstd{\ \ \ \ }\hlstd{}\hlkwa{OUT\ }\hlstd{}\hlkwb{DX}\hlstd{}\hlopt{,}\hlstd{}\hlkwb{AL\ }\\
\hllin{95\ }\hlstd{}\hlstd{\ \ \ \ }\hlstd{}\hlkwa{MOV\ }\hlstd{}\hlkwb{DX}\hlstd{}\hlopt{,}\hlstd{}\hlnum{282H}\\
\hllin{96\ }\hlstd{}\hlstd{\ \ \ \ }\hlstd{}\hlkwa{MOV\ }\hlstd{}\hlkwb{AL}\hlstd{}\hlopt{,}\hlstd{}\hlnum{01H}\\
\hllin{97\ }\hlstd{}\hlstd{\ \ \ \ }\hlstd{}\hlkwa{OUT\ }\hlstd{}\hlkwb{DX}\hlstd{}\hlopt{,}\hlstd{}\hlkwb{AL}\\
\hllin{98\ }\hlstd{}\hlstd{\ \ \ \ }\hlstd{}\hlkwa{MOV\ }\hlstd{}\hlkwb{BL}\hlstd{}\hlopt{,}\hlstd{}\hlnum{1}\\
\hllin{99\ }\hlstd{}\hlstd{\ \ \ \ }\hlstd{}\hlkwa{JMP\ }\hlstd{END\textunderscore INTR}\\
\hllin{100\ }\hlkwc{OUT1:}\hlstd{\ \ \ }\hlkwc{}\\
\hllin{101\ }\hlstd{\\
\hllin{102\ }}\hlstd{\ \ \ \ }\hlstd{}\hlkwa{MOV\ }\hlstd{}\hlkwb{DX}\hlstd{}\hlopt{,}\hlstd{}\hlnum{280H}\\
\hllin{103\ }\hlstd{}\hlstd{\ \ \ \ }\hlstd{}\hlkwa{MOV\ }\hlstd{}\hlkwb{AL}\hlstd{}\hlopt{,}\hlstd{}\hlnum{06H}\\
\hllin{104\ }\hlstd{}\hlstd{\ \ \ \ }\hlstd{}\hlkwa{OUT\ }\hlstd{}\hlkwb{DX}\hlstd{}\hlopt{,}\hlstd{}\hlkwb{AL}\\
\hllin{105\ }\hlstd{}\hlstd{\ \ \ \ }\hlstd{}\hlkwa{MOV\ }\hlstd{}\hlkwb{DX}\hlstd{}\hlopt{,}\hlstd{}\hlnum{282H}\\
\hllin{106\ }\hlstd{}\hlstd{\ \ \ \ }\hlstd{}\hlkwa{MOV\ }\hlstd{}\hlkwb{AL}\hlstd{}\hlopt{,}\hlstd{}\hlnum{02H}\\
\hllin{107\ }\hlstd{}\hlstd{\ \ \ \ }\hlstd{}\hlkwa{OUT\ }\hlstd{}\hlkwb{DX}\hlstd{}\hlopt{,}\hlstd{}\hlkwb{AL}\\
\hllin{108\ }\hlstd{}\hlstd{\ \ \ \ }\hlstd{}\hlkwa{MOV\ }\hlstd{}\hlkwb{BL}\hlstd{}\hlopt{,}\hlstd{}\hlnum{0}\\
\hllin{109\ }\hlstd{}\hlstd{\ \ \ \ }\hlstd{}\hlkwa{JMP\ }\hlstd{END\textunderscore INTR\\
\hllin{110\ }\ }\\
\hllin{111\ }\hlkwc{END\textunderscore INTR:}\\
\hllin{112\ }\hlstd{}\hlstd{\ \ \ \ }\hlstd{}\hlkwa{MOV\ }\hlstd{}\hlkwb{AL}\hlstd{}\hlopt{,}\hlstd{}\hlnum{20H}\hlstd{\ \ \ \ \ \ \ \ \ }\hlnum{}\\
\hllin{113\ }\hlstd{}\hlstd{\ \ \ \ }\hlstd{}\hlkwa{OUT\ }\hlstd{}\hlnum{0}\hlstd{A0H}\hlopt{,}\hlstd{}\hlkwb{AL}\\
\hllin{114\ }\hlstd{}\hlstd{\ \ \ \ }\hlstd{}\hlkwa{OUT\ }\hlstd{}\hlnum{20H}\hlstd{}\hlopt{,}\hlstd{}\hlkwb{AL}\\
\hllin{115\ }\hlstd{}\hlkwa{IRET}\\
\hllin{116\ }\hlstd{INTR\ }\hlkwa{ENDP}\\
\hllin{117\ }\hlstd{\\
\hllin{118\ }\\
\hllin{119\ }CODE\ }\hlkwa{ENDS}\\
\hllin{120\ }\hlstd{}\hlkwa{END\ }\hlstd{START}\hlstd{\ \ \ \ }\hlstd{}\\
\hllin{121\ }\\
\hllin{122\ }\\
\hllin{123\ }\\
\hllin{124\ }\\
\hllin{125\ }\\
\hllin{126\ }\\
\hllin{127\ }\\
\hllin{128\ }\\
\mbox{}
\normalfont
\normalsize

\subsection{完成情况及心得体会}
本次实验使用了并口等元件完成了CPU对外设的控制,提升
了汇编语言学习能力。
\end{document}