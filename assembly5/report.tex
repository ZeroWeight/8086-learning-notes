\documentclass[UTF8,a4paper]{paper}
\usepackage{ctex}
\usepackage[utf8]{inputenc}
\usepackage{amsmath}
\usepackage{pdfpages}
\usepackage{graphicx}
\usepackage{wrapfig}
\usepackage{listings}
\usepackage{color}
\usepackage{alltt}
\usepackage{marvosym}
\usepackage{xcolor}
\input {highlight.sty}
\title{计算机原理第五次实验报告}
\author{张蔚桐\ 2015011493\ 自55}
\begin {document}
\maketitle
\section{实验目的}
\begin{enumerate}
\item 理解中断控制器8259 及计数/定时器8253 的工作原理,掌握其使用方法。
\item 练习使用DOS 及BIOS 功能调用来编写I/O 程序.
\end{enumerate}
\section{单击测试}
直接对计算机自带的8250 芯片编程,将其设置成1200 波特率、8 个数据位、1 个停止位、奇校验规则、
自检方式。用DOS 功能调用(INT21H) 接收键入字符(需要回显),通过8250 以查询方式发送又自己接收
并在CRT 上显示。直至键入空格(ASC￿ 码为20H) 时退回DOS。\\ 

\noindent
\ttfamily
\hlstd{\hllin{01\ }DATA}\hlstd{\ \ }\hlstd{}\hlkwa{SEGMENT}\\
\hllin{02\ }\hlstd{}\hlstd{\ \ \ \ \ \ \ }\hlstd{MESS1}\hlstd{\ \ }\hlstd{}\hlkwa{DB}\hlstd{\ \ \ \ \ \ \ \ }\hlkwa{}\hlstd{}\hlstr{'RECEIVING:'}\hlstd{}\hlopt{,}\hlstd{}\hlstr{'\$'}\hlstd{\\
\hllin{03\ }}\hlstd{\ \ \ \ \ \ \ }\hlstd{MESS2}\hlstd{\ \ }\hlstd{}\hlkwa{DB}\hlstd{\ \ \ \ \ \ \ \ }\hlkwa{}\hlstd{}\hlstr{'WRONG\ INPUT!'}\hlstd{}\hlopt{,}\hlstd{}\hlnum{0}\hlstd{}\hlkwb{DH}\hlstd{}\hlopt{,}\hlstd{}\hlnum{0}\hlstd{}\hlkwb{AH}\hlstd{}\hlopt{,}\hlstd{}\hlstr{'\$'}\hlstd{\\
\hllin{04\ }}\hlstd{\ \ \ \ \ \ \ }\hlstd{MESS3}\hlstd{\ \ }\hlstd{}\hlkwa{DB}\hlstd{\ \ \ \ \ \ \ \ }\hlkwa{}\hlstd{}\hlstr{'HAVE\ DONE'}\hlstd{}\hlopt{,}\hlstd{}\hlnum{0}\hlstd{}\hlkwb{DH}\hlstd{}\hlopt{,}\hlstd{}\hlnum{0}\hlstd{}\hlkwb{AH}\hlstd{}\hlopt{,}\hlstd{}\hlstr{'\$'}\hlstd{\\
\hllin{05\ }}\hlstd{\ \ \ \ \ \ \ }\hlstd{MESS4}\hlstd{\ \ }\hlstd{}\hlkwa{DB}\hlstd{\ \ \ \ \ \ \ \ }\hlkwa{}\hlstd{}\hlnum{0}\hlstd{}\hlkwb{DH}\hlstd{}\hlopt{,}\hlstd{}\hlnum{0}\hlstd{}\hlkwb{AH}\hlstd{}\hlopt{,}\hlstd{}\hlstr{'\$'}\hlstd{\\
\hllin{06\ }DATA}\hlstd{\ \ }\hlstd{}\hlkwa{ENDS}\\
\hllin{07\ }\hlstd{}\hlstd{\ \ }\hlstd{\\
\hllin{08\ }}\hlstd{\ \ \ \ \ \ \ }\hlstd{STACK}\hlstd{\ \ }\hlstd{}\hlkwa{SEGMENT}\\
\hllin{09\ }\hlstd{}\hlstd{\ \ \ \ \ \ \ \ \ \ \ \ \ \ }\hlstd{}\hlkwa{DB}\hlstd{\ \ \ \ \ \ \ \ }\hlkwa{}\hlstd{}\hlnum{100\ }\hlstd{DUP}\hlopt{(}\hlstd{?}\hlopt{)}\\
\hllin{10\ }\hlstd{}\hlstd{\ \ \ \ \ \ \ }\hlstd{STACK}\hlstd{\ \ }\hlstd{}\hlkwa{ENDS}\\
\hllin{11\ }\hlstd{}\hlstd{\ \ \ \ }\hlstd{\\
\hllin{12\ }}\hlstd{\ \ \ \ \ \ \ \ }\hlstd{CODE}\hlstd{\ \ }\hlstd{}\hlkwa{SEGMENT}\\
\hllin{13\ }\hlstd{}\hlstd{\ \ \ \ \ \ \ \ \ \ \ \ \ \ }\hlstd{}\hlkwa{ASSUME}\hlstd{\ \ \ \ }\hlkwa{}\hlstd{}\hlkwb{CS}\hlstd{}\hlopt{:}\hlstd{CODE}\hlopt{,}\hlstd{}\hlkwb{DS}\hlstd{}\hlopt{:}\hlstd{DATA}\hlopt{,}\hlstd{}\hlkwb{ES}\hlstd{}\hlopt{:}\hlstd{DATA}\hlopt{,}\hlstd{}\hlkwb{SS}\hlstd{}\hlopt{:}\hlstd{STACK\\
\hllin{14\ }}\hlstd{\ \ \ \ \ \ }\hlstd{START}\hlopt{:}\\
\hllin{15\ }\hlstd{}\hlstd{\ \ \ \ \ \ \ \ \ \ \ \ \ \ }\hlstd{}\hlkwa{MOV}\hlstd{\ \ \ \ \ \ \ }\hlkwa{}\hlstd{}\hlkwb{AX}\hlstd{}\hlopt{,}\hlstd{DATA\\
\hllin{16\ }}\hlstd{\ \ \ \ \ \ \ \ \ \ \ \ \ \ }\hlstd{}\hlkwa{MOV}\hlstd{\ \ \ \ \ \ \ }\hlkwa{}\hlstd{}\hlkwb{DS}\hlstd{}\hlopt{,}\hlstd{}\hlkwb{AX}\\
\hllin{17\ }\hlstd{}\hlstd{\ \ \ \ \ \ \ \ \ \ \ \ \ \ }\hlstd{}\hlkwa{MOV}\hlstd{\ \ \ \ \ \ \ }\hlkwa{}\hlstd{}\hlkwb{ES}\hlstd{}\hlopt{,}\hlstd{}\hlkwb{AX}\hlstd{\ \ \ }\hlkwb{}\\
\hllin{18\ }\hlstd{}\hlstd{\ \ \ \ \ \ \ \ \ \ \ \ \ \ }\hlstd{}\hlkwa{MOV}\hlstd{\ \ \ \ \ \ \ }\hlkwa{}\hlstd{}\hlkwb{DX}\hlstd{}\hlopt{,}\hlstd{}\hlnum{3}\hlstd{FBH\\
\hllin{19\ }}\hlstd{\ \ \ \ \ }\hlstd{}\hlkwa{MOV}\hlstd{\ \ \ \ \ \ \ }\hlkwa{}\hlstd{}\hlkwb{AL}\hlstd{}\hlopt{,}\hlstd{}\hlnum{80H}\hlstd{\ \ \ \ \ \ }\hlnum{}\\
\hllin{20\ }\hlstd{}\hlstd{\ \ \ \ \ \ \ \ \ \ \ \ \ \ }\hlstd{}\hlkwa{OUT}\hlstd{\ \ \ \ \ \ \ }\hlkwa{}\hlstd{}\hlkwb{DX}\hlstd{}\hlopt{,}\hlstd{}\hlkwb{AL}\hlstd{\ \ \ \ }\hlkwb{}\\
\hllin{21\ }\hlstd{}\hlstd{\ \ \ \ \ \ \ \ \ \ \ \ \ \ }\hlstd{}\hlkwa{MOV}\hlstd{\ \ \ \ \ \ \ }\hlkwa{}\hlstd{}\hlkwb{DX}\hlstd{}\hlopt{,}\hlstd{}\hlnum{3}\hlstd{F9H\\
\hllin{22\ }}\hlstd{\ \ \ \ \ }\hlstd{}\hlkwa{MOV}\hlstd{\ \ }\hlkwa{}\hlstd{}\hlkwb{AL}\hlstd{}\hlopt{,}\hlstd{}\hlnum{0}\\
\hllin{23\ }\hlstd{}\hlstd{\ \ \ \ \ \ \ \ \ \ \ \ \ \ }\hlstd{}\hlkwa{OUT}\hlstd{\ \ \ \ \ \ \ }\hlkwa{}\hlstd{}\hlkwb{DX}\hlstd{}\hlopt{,}\hlstd{}\hlkwb{AL}\\
\hllin{24\ }\hlstd{}\hlstd{\ \ \ \ \ \ \ \ \ \ \ \ \ \ }\hlstd{}\hlkwa{MOV}\hlstd{\ \ \ \ \ \ \ }\hlkwa{}\hlstd{}\hlkwb{DX}\hlstd{}\hlopt{,}\hlstd{}\hlnum{3}\hlstd{F8H\\
\hllin{25\ }}\hlstd{\ \ \ \ \ }\hlstd{}\hlkwa{MOV}\hlstd{\ \ \ \ \ \ \ }\hlkwa{}\hlstd{}\hlkwb{AX}\hlstd{}\hlopt{,}\hlstd{}\hlnum{60H}\hlstd{\ \ \ \ \ \ \ }\hlnum{}\\
\hllin{26\ }\hlstd{}\hlstd{\ \ \ \ \ \ \ \ \ \ \ \ \ \ }\hlstd{}\hlkwa{OUT}\hlstd{\ \ \ \ \ \ \ }\hlkwa{}\hlstd{}\hlkwb{DX}\hlstd{}\hlopt{,}\hlstd{}\hlkwb{AL}\hlstd{\ \ \ \ \ \ \ \ \ \ \ \ \ \ }\hlkwb{}\\
\hllin{27\ }\hlstd{}\hlstd{\ \ \ \ \ \ \ \ \ \ \ \ \ \ }\hlstd{}\hlkwa{MOV}\hlstd{\ \ \ \ \ \ \ }\hlkwa{}\hlstd{}\hlkwb{DX}\hlstd{}\hlopt{,}\hlstd{}\hlnum{3}\hlstd{FBH\\
\hllin{28\ }}\hlstd{\ \ \ \ \ \ \ \ \ \ \ \ \ \ }\hlstd{}\hlkwa{MOV}\hlstd{\ \ \ \ \ \ \ }\hlkwa{}\hlstd{}\hlkwb{AL}\hlstd{}\hlopt{,}\hlstd{}\hlnum{00001011}\hlstd{B}\hlstd{\ \ }\hlstd{\\
\hllin{29\ }}\hlstd{\ \ \ \ \ }\hlstd{}\hlkwa{OUT}\hlstd{\ \ \ \ \ \ \ }\hlkwa{}\hlstd{}\hlkwb{DX}\hlstd{}\hlopt{,}\hlstd{}\hlkwb{AL}\hlstd{\ \ \ \ \ \ \ \ \ \ }\hlkwb{}\\
\hllin{30\ }\hlstd{}\hlstd{\ \ \ \ \ \ \ \ \ \ \ \ \ \ }\hlstd{}\hlkwa{MOV}\hlstd{\ \ \ \ \ \ \ }\hlkwa{}\hlstd{}\hlkwb{DX}\hlstd{}\hlopt{,}\hlstd{}\hlnum{3}\hlstd{FCH\\
\hllin{31\ }}\hlstd{\ \ \ \ \ }\hlstd{}\hlkwa{MOV}\hlstd{\ \ \ \ \ \ \ }\hlkwa{}\hlstd{}\hlkwb{AL}\hlstd{}\hlopt{,}\hlstd{}\hlnum{13H}\hlstd{\ \ \ \ \ \ \ \ \ \ \ }\hlnum{}\\
\hllin{32\ }\hlstd{}\hlstd{\ \ \ \ \ \ \ \ \ \ \ \ \ \ }\hlstd{}\hlkwa{OUT}\hlstd{\ \ \ \ \ \ \ }\hlkwa{}\hlstd{}\hlkwb{DX}\hlstd{}\hlopt{,}\hlstd{}\hlkwb{AL}\\
\hllin{33\ }\hlstd{}\hlstd{\ \ \ \ \ }\hlstd{}\hlkwa{MOV}\hlstd{\ \ \ \ \ \ \ }\hlkwa{}\hlstd{}\hlkwb{DX}\hlstd{}\hlopt{,}\hlstd{}\hlnum{3}\hlstd{F9H\\
\hllin{34\ }}\hlstd{\ \ \ \ \ \ \ \ \ \ \ \ \ \ }\hlstd{}\hlkwa{MOV}\hlstd{\ \ \ \ \ \ \ }\hlkwa{}\hlstd{}\hlkwb{AL}\hlstd{}\hlopt{,}\hlstd{}\hlnum{0}\hlstd{\ \ \ \ \ \ \ \ \ \ \ \ \ \ \ \ }\hlnum{}\\
\hllin{35\ }\hlstd{}\hlstd{\ \ \ \ \ \ \ \ \ \ \ \ \ \ }\hlstd{}\hlkwa{OUT}\hlstd{\ \ \ \ \ \ \ }\hlkwa{}\hlstd{}\hlkwb{DX}\hlstd{}\hlopt{,}\hlstd{}\hlkwb{AL}\\
\hllin{36\ }\hlstd{}\hlstd{\ \ \ }\hlstd{WAIT\textunderscore FOR}\hlopt{:}\\
\hllin{37\ }\hlstd{}\hlstd{\ \ \ \ \ \ \ \ \ \ \ \ \ \ }\hlstd{}\hlkwa{MOV}\hlstd{\ \ \ \ \ \ \ }\hlkwa{}\hlstd{}\hlkwb{DX}\hlstd{}\hlopt{,}\hlstd{}\hlnum{3}\hlstd{FDH\\
\hllin{38\ }}\hlstd{\ \ \ \ \ \ \ \ \ \ \ \ \ \ }\hlstd{}\hlkwa{IN}\hlstd{\ \ \ \ \ \ \ \ }\hlkwa{}\hlstd{}\hlkwb{AL}\hlstd{}\hlopt{,}\hlstd{}\hlkwb{DX}\\
\hllin{39\ }\hlstd{}\hlstd{\ \ \ \ \ \ \ \ \ \ \ \ \ \ }\hlstd{}\hlkwa{TEST}\hlstd{\ \ \ \ \ \ }\hlkwa{}\hlstd{}\hlkwb{AL}\hlstd{}\hlopt{,}\hlstd{}\hlnum{00011110}\hlstd{B\\
\hllin{40\ }}\hlstd{\ \ \ \ \ \ \ \ \ \ \ \ \ \ }\hlstd{}\hlkwa{JNZ}\hlstd{\ \ \ \ \ \ \ }\hlkwa{}\hlstd{ERROR\\
\hllin{41\ }}\hlstd{\ \ \ \ \ \ \ \ \ \ \ \ \ \ }\hlstd{}\hlkwa{TEST}\hlstd{\ \ \ \ \ \ }\hlkwa{}\hlstd{}\hlkwb{AL}\hlstd{}\hlopt{,}\hlstd{}\hlnum{1}\hlstd{\ \ \ \ \ \ \ \ \ \ \ \ \ }\hlnum{}\\
\hllin{42\ }\hlstd{}\hlstd{\ \ \ \ \ \ \ \ \ \ \ \ \ \ }\hlstd{}\hlkwa{JNZ}\hlstd{\ \ \ \ \ \ \ }\hlkwa{}\hlstd{RECEIVE}\hlstd{\ \ }\hlstd{\\
\hllin{43\ }}\hlstd{\ \ \ \ \ \ \ \ \ \ \ \ \ \ }\hlstd{}\hlkwa{TEST}\hlstd{\ \ \ \ \ \ }\hlkwa{}\hlstd{}\hlkwb{AL}\hlstd{}\hlopt{,}\hlstd{}\hlnum{00100000}\hlstd{B\\
\hllin{44\ }}\hlstd{\ \ \ \ \ \ \ \ \ \ \ \ \ \ }\hlstd{}\hlkwa{JZ}\hlstd{\ \ \ \ \ \ \ \ }\hlkwa{}\hlstd{WAIT\textunderscore FOR\ \\
\hllin{45\ }}\hlstd{\ \ \ \ \ \ \ \ \ \ \ \ \ \ }\hlstd{}\hlkwa{MOV}\hlstd{\ \ \ \ \ \ \ }\hlkwa{}\hlstd{}\hlkwb{AH}\hlstd{}\hlopt{,}\hlstd{}\hlnum{1}\\
\hllin{46\ }\hlstd{}\hlstd{\ \ \ \ \ \ \ \ \ \ \ \ \ \ }\hlstd{}\hlkwa{INT}\hlstd{\ \ \ \ \ \ \ }\hlkwa{}\hlstd{}\hlnum{21H}\\
\hllin{47\ }\hlstd{}\hlstd{\ \ \ \ \ \ \ \ \ \ \ \ \ \ }\hlstd{}\hlkwa{CMP}\hlstd{\ \ \ \ \ \ \ }\hlkwa{}\hlstd{}\hlkwb{AL}\hlstd{}\hlopt{,}\hlstd{}\hlnum{20H}\\
\hllin{48\ }\hlstd{}\hlstd{\ \ \ \ \ \ \ \ \ \ \ \ \ \ }\hlstd{}\hlkwa{JE}\hlstd{\ \ \ \ \ \ \ \ }\hlkwa{}\hlstd{STOPWORK\\
\hllin{49\ }}\hlstd{\ \ \ \ \ \ \ \ \ \ \ \ \ \ }\hlstd{}\hlkwa{MOV}\hlstd{\ \ \ \ \ \ \ }\hlkwa{}\hlstd{}\hlkwb{CL}\hlstd{}\hlopt{,}\hlstd{}\hlkwb{AL}\hlstd{\ \ \ \ \ \ \ }\hlkwb{}\\
\hllin{50\ }\hlstd{}\hlstd{\ \ \ \ \ \ \ \ \ \ \ \ \ \ }\hlstd{}\hlkwa{MOV}\hlstd{\ \ \ \ \ \ \ }\hlkwa{}\hlstd{}\hlkwb{DX}\hlstd{}\hlopt{,}\hlstd{}\hlnum{3}\hlstd{F8H\\
\hllin{51\ }}\hlstd{\ \ \ \ \ \ \ \ \ \ \ \ \ \ }\hlstd{}\hlkwa{OUT}\hlstd{\ \ \ \ \ \ \ }\hlkwa{}\hlstd{}\hlkwb{DX}\hlstd{}\hlopt{,}\hlstd{}\hlkwb{AL}\hlstd{\ \ \ \ \ \ \ }\hlkwb{}\\
\hllin{52\ }\hlstd{}\hlstd{\ \ \ \ \ \ \ \ \ \ \ \ \ \ }\hlstd{}\hlkwa{JMP}\hlstd{\ \ \ \ \ \ \ }\hlkwa{}\hlstd{WAIT\textunderscore FOR\\
\hllin{53\ }}\hlstd{\ \ \ \ }\hlstd{RECEIVE}\hlopt{:}\\
\hllin{54\ }\hlstd{}\hlstd{\ \ \ \ \ \ \ \ \ \ \ \ \ \ }\hlstd{}\hlkwa{LEA}\hlstd{\ \ \ \ \ \ \ }\hlkwa{}\hlstd{}\hlkwb{DX}\hlstd{}\hlopt{,}\hlstd{MESS1\\
\hllin{55\ }}\hlstd{\ \ \ \ \ \ \ \ \ \ \ \ \ \ }\hlstd{}\hlkwa{MOV}\hlstd{\ \ \ \ \ \ \ }\hlkwa{}\hlstd{}\hlkwb{AH}\hlstd{}\hlopt{,}\hlstd{}\hlnum{9}\\
\hllin{56\ }\hlstd{}\hlstd{\ \ \ \ \ \ \ \ \ \ \ \ \ \ }\hlstd{}\hlkwa{INT}\hlstd{\ \ \ \ \ \ \ }\hlkwa{}\hlstd{}\hlnum{21H}\hlstd{\ \ \ \ \ \ \ \ \ }\hlnum{}\\
\hllin{57\ }\hlstd{}\hlstd{\ \ \ \ \ \ \ \ \ \ \ \ \ \ }\hlstd{}\hlkwa{MOV}\hlstd{\ \ \ \ \ \ \ }\hlkwa{}\hlstd{}\hlkwb{DX}\hlstd{}\hlopt{,}\hlstd{}\hlnum{3}\hlstd{F8H\\
\hllin{58\ }}\hlstd{\ \ \ \ \ \ \ \ \ \ \ \ \ \ }\hlstd{}\hlkwa{IN}\hlstd{\ \ \ \ \ \ \ \ }\hlkwa{}\hlstd{}\hlkwb{AL}\hlstd{}\hlopt{,}\hlstd{}\hlkwb{DX}\\
\hllin{59\ }\hlstd{}\hlstd{\ \ \ \ \ \ \ \ \ \ \ \ \ \ }\hlstd{}\hlkwa{MOV}\hlstd{\ \ \ \ \ \ \ }\hlkwa{}\hlstd{}\hlkwb{DL}\hlstd{}\hlopt{,}\hlstd{}\hlkwb{AL}\\
\hllin{60\ }\hlstd{}\hlstd{\ \ \ \ \ \ \ \ \ \ \ \ \ \ }\hlstd{}\hlkwa{MOV}\hlstd{\ \ \ \ \ \ \ }\hlkwa{}\hlstd{}\hlkwb{AH}\hlstd{}\hlopt{,}\hlstd{}\hlnum{02H}\hlstd{\ \ \ \ \ \ }\hlnum{}\\
\hllin{61\ }\hlstd{}\hlstd{\ \ \ \ \ \ \ \ \ \ \ \ \ \ }\hlstd{}\hlkwa{INT}\hlstd{\ \ \ \ \ \ \ }\hlkwa{}\hlstd{}\hlnum{21H}\\
\hllin{62\ }\hlstd{}\hlstd{\ \ \ \ \ \ \ \ \ \ \ \ \ \ }\hlstd{}\hlkwa{LEA}\hlstd{\ \ \ \ \ \ \ }\hlkwa{}\hlstd{}\hlkwb{DX}\hlstd{}\hlopt{,}\hlstd{MESS4\\
\hllin{63\ }}\hlstd{\ \ \ \ \ \ \ \ \ \ \ \ \ \ }\hlstd{}\hlkwa{MOV}\hlstd{\ \ \ \ \ \ \ }\hlkwa{}\hlstd{}\hlkwb{AH}\hlstd{}\hlopt{,}\hlstd{}\hlnum{09H}\\
\hllin{64\ }\hlstd{}\hlstd{\ \ \ \ \ \ \ \ \ \ \ \ \ \ }\hlstd{}\hlkwa{INT}\hlstd{\ \ \ \ \ \ \ }\hlkwa{}\hlstd{}\hlnum{21H}\\
\hllin{65\ }\hlstd{}\hlstd{\ \ \ \ \ \ \ \ \ \ \ \ \ \ }\hlstd{}\hlkwa{JMP}\hlstd{\ \ \ \ \ \ \ }\hlkwa{}\hlstd{WAIT\textunderscore FOR\\
\hllin{66\ }}\hlstd{\ \ \ \ \ \ }\hlstd{ERROR}\hlopt{:}\hlstd{\ \ \ \ \ \ \ \ \ \ \ \ \ \ \ \ \ \ \ \ \ \ \ \ }\hlopt{}\\
\hllin{67\ }\hlstd{}\hlstd{\ \ \ \ \ \ \ \ \ \ \ \ \ \ }\hlstd{}\hlkwa{LEA}\hlstd{\ \ \ \ \ \ \ }\hlkwa{}\hlstd{}\hlkwb{DX}\hlstd{}\hlopt{,}\hlstd{MESS2\\
\hllin{68\ }}\hlstd{\ \ \ \ \ \ \ \ \ \ \ \ \ \ }\hlstd{}\hlkwa{MOV}\hlstd{\ \ \ \ \ \ \ }\hlkwa{}\hlstd{}\hlkwb{AH}\hlstd{}\hlopt{,}\hlstd{}\hlnum{9}\\
\hllin{69\ }\hlstd{}\hlstd{\ \ \ \ \ \ \ \ \ \ \ \ \ \ }\hlstd{}\hlkwa{INT}\hlstd{\ \ \ \ \ \ \ }\hlkwa{}\hlstd{}\hlnum{21H}\hlstd{\ \ \ \ \ \ \ \ \ }\hlnum{}\\
\hllin{70\ }\hlstd{}\hlstd{\ \ \ \ \ \ \ \ \ \ \ \ \ \ }\hlstd{}\hlkwa{JMP}\hlstd{\ \ \ \ \ \ \ }\hlkwa{}\hlstd{WAIT\textunderscore FOR\\
\hllin{71\ }}\hlstd{\ \ \ \ \ }\hlstd{STOPWORK}\hlopt{:}\\
\hllin{72\ }\hlstd{}\hlstd{\ \ \ \ \ \ \ \ \ \ \ \ \ \ }\hlstd{}\hlkwa{LEA}\hlstd{\ \ \ \ \ \ \ }\hlkwa{}\hlstd{}\hlkwb{DX}\hlstd{}\hlopt{,}\hlstd{MESS3\\
\hllin{73\ }}\hlstd{\ \ \ \ \ \ \ \ \ \ \ \ \ \ }\hlstd{}\hlkwa{MOV}\hlstd{\ \ \ \ \ \ \ }\hlkwa{}\hlstd{}\hlkwb{AH}\hlstd{}\hlopt{,}\hlstd{}\hlnum{9}\\
\hllin{74\ }\hlstd{}\hlstd{\ \ \ \ \ \ \ \ \ \ \ \ \ \ }\hlstd{}\hlkwa{INT}\hlstd{\ \ \ \ \ \ \ }\hlkwa{}\hlstd{}\hlnum{21H}\\
\hllin{75\ }\hlstd{}\hlstd{\ \ \ \ \ \ \ \ \ \ \ \ \ \ }\hlstd{}\hlkwa{MOV}\hlstd{\ \ \ \ \ \ \ }\hlkwa{}\hlstd{}\hlkwb{AH}\hlstd{}\hlopt{,}\hlstd{}\hlnum{4}\hlstd{}\hlkwb{CH}\hlstd{\ \ \ \ \ \ }\hlkwb{}\\
\hllin{76\ }\hlstd{}\hlstd{\ \ \ \ \ \ \ \ \ \ \ \ \ \ }\hlstd{}\hlkwa{INT}\hlstd{\ \ \ \ \ \ \ }\hlkwa{}\hlstd{}\hlnum{21H}\\
\hllin{77\ }\hlstd{}\hlstd{\ \ \ \ \ \ \ \ }\hlstd{CODE}\hlstd{\ \ }\hlstd{}\hlkwa{ENDS}\\
\hllin{78\ }\hlstd{}\hlstd{\ \ \ \ \ \ \ \ \ \ \ \ \ \ }\hlstd{}\hlkwa{END}\hlstd{\ \ \ \ \ \ \ }\hlkwa{}\hlstd{START}\\
\mbox{}
\normalfont
\normalsize

\section{计算机间通信}
将上程序修改成两台计算机之间以查询方式通信,即一方键入的字符在另一个CRT 上显示,反之亦然,
任何一方键入空格,双方都退出。\\ 

\noindent
\ttfamily
\hlstd{\hllin{01\ }DATAS}\hlstd{\ \ }\hlstd{}\hlkwa{SEGMENT}\hlstd{\ \ \ \ \ \ \ \ \ \ \ \ \ \ }\hlkwa{}\\
\hllin{02\ }\hlstd{DIVID}\hlstd{\ \ }\hlstd{}\hlkwa{DW}\hlstd{\ \ \ \ }\hlkwa{}\hlstd{}\hlnum{60H}\\
\hllin{03\ }\hlstd{DATAS}\hlstd{\ \ }\hlstd{}\hlkwa{ENDS}\\
\hllin{04\ }\hlstd{\\
\hllin{05\ }STACKS}\hlstd{\ \ }\hlstd{}\hlkwa{SEGMENT}\hlstd{\ \ \ }\hlkwa{}\hlstd{STACK}\hlstd{\ \ \ \ \ \ }\hlstd{\\
\hllin{06\ }}\hlstd{\ \ \ \ }\hlstd{}\hlkwa{DW}\hlstd{\ \ \ \ }\hlkwa{}\hlstd{}\hlnum{128\ }\hlstd{DUP}\hlopt{(}\hlstd{?}\hlopt{)}\hlstd{\ \ }\hlopt{}\\
\hllin{07\ }\hlstd{STACKS}\hlstd{\ \ }\hlstd{}\hlkwa{ENDS}\\
\hllin{08\ }\hlstd{\\
\hllin{09\ }CODES}\hlstd{\ \ }\hlstd{}\hlkwa{SEGMENT}\hlstd{\ \ \ \ \ \ \ \ \ \ \ \ \ \ \ }\hlkwa{}\\
\hllin{10\ }\hlstd{}\hlkwa{ASSUME}\hlstd{\ \ \ \ }\hlkwa{}\hlstd{}\hlkwb{CS}\hlstd{}\hlopt{:}\hlstd{CODES}\hlopt{,}\hlstd{}\hlkwb{DS}\hlstd{}\hlopt{:}\hlstd{DATAS\\
\hllin{11\ }SUB1}\hlstd{\ \ }\hlstd{}\hlkwa{PROC}\hlstd{\ \ \ \ \ \ }\hlkwa{FAR}\\
\hllin{12\ }\hlstd{}\hlstd{\ \ }\hlstd{}\\
\hllin{13\ }\hlkwc{START:}\hlstd{\ \ }\hlkwc{}\hlstd{}\hlkwa{MOV}\hlstd{\ \ \ \ \ \ \ }\hlkwa{}\hlstd{}\hlkwb{AX}\hlstd{}\hlopt{,}\hlstd{DATAS\\
\hllin{14\ }}\hlstd{\ \ \ \ \ \ \ \ }\hlstd{}\hlkwa{MOV}\hlstd{\ \ \ \ \ \ \ }\hlkwa{}\hlstd{}\hlkwb{DS}\hlstd{}\hlopt{,}\hlstd{}\hlkwb{AX}\\
\hllin{15\ }\hlstd{}\hlslc{;设置波特率为1200}\\
\hllin{16\ }\hlstd{}\hlstd{\ \ \ \ \ \ \ \ }\hlstd{}\hlkwa{MOV}\hlstd{\ \ \ \ \ \ \ }\hlkwa{}\hlstd{}\hlkwb{AL}\hlstd{}\hlopt{,}\hlstd{}\hlnum{80H}\\
\hllin{17\ }\hlstd{}\hlstd{\ \ \ \ \ \ \ \ }\hlstd{}\hlkwa{MOV}\hlstd{\ \ \ \ \ \ \ }\hlkwa{}\hlstd{}\hlkwb{DX}\hlstd{}\hlopt{,}\hlstd{}\hlnum{3}\hlstd{FBH\\
\hllin{18\ }}\hlstd{\ \ \ \ \ \ \ \ }\hlstd{}\hlkwa{OUT}\hlstd{\ \ \ \ \ \ \ }\hlkwa{}\hlstd{}\hlkwb{DX}\hlstd{}\hlopt{,}\hlstd{}\hlkwb{AL}\\
\hllin{19\ }\hlstd{}\hlstd{\ \ \ \ \ \ \ \ }\hlstd{}\hlkwa{MOV}\hlstd{\ \ \ \ \ \ \ }\hlkwa{}\hlstd{}\hlkwb{AX}\hlstd{}\hlopt{,}\hlstd{DIVID\\
\hllin{20\ }}\hlstd{\ \ \ \ \ \ \ \ }\hlstd{}\hlkwa{MOV}\hlstd{\ \ \ \ \ \ \ }\hlkwa{}\hlstd{}\hlkwb{DX}\hlstd{}\hlopt{,}\hlstd{}\hlnum{3}\hlstd{F8H\\
\hllin{21\ }}\hlstd{\ \ \ \ \ \ \ \ }\hlstd{}\hlkwa{OUT}\hlstd{\ \ \ \ \ \ \ }\hlkwa{}\hlstd{}\hlkwb{DX}\hlstd{}\hlopt{,}\hlstd{}\hlkwb{AL}\hlstd{\ \ \ \ \ \ \ }\hlkwb{}\\
\hllin{22\ }\hlstd{}\hlstd{\ \ \ \ \ \ \ \ }\hlstd{}\hlkwa{MOV}\hlstd{\ \ \ \ \ \ \ }\hlkwa{}\hlstd{}\hlkwb{AL}\hlstd{}\hlopt{,}\hlstd{}\hlkwb{AH}\\
\hllin{23\ }\hlstd{}\hlstd{\ \ \ \ \ \ \ \ }\hlstd{}\hlkwa{MOV}\hlstd{\ \ \ \ \ \ \ }\hlkwa{}\hlstd{}\hlkwb{DX}\hlstd{}\hlopt{,}\hlstd{}\hlnum{3}\hlstd{F9H\\
\hllin{24\ }}\hlstd{\ \ \ \ \ \ \ \ }\hlstd{}\hlkwa{OUT}\hlstd{\ \ \ \ \ \ \ }\hlkwa{}\hlstd{}\hlkwb{DX}\hlstd{}\hlopt{,}\hlstd{}\hlkwb{AL}\hlstd{\ \ \ \ \ \ \ \ }\hlkwb{}\\
\hllin{25\ }\hlstd{}\hlstd{\ \ \ \ \ \ \ \ }\hlstd{}\hlkwa{MOV}\hlstd{\ \ \ \ \ \ \ }\hlkwa{}\hlstd{}\hlkwb{AL}\hlstd{}\hlopt{,}\hlstd{}\hlnum{00001011}\hlstd{B\\
\hllin{26\ }}\hlstd{\ \ \ \ \ \ \ \ }\hlstd{}\hlkwa{MOV}\hlstd{\ \ \ \ \ \ \ }\hlkwa{}\hlstd{}\hlkwb{DX}\hlstd{}\hlopt{,}\hlstd{}\hlnum{3}\hlstd{FBH}\hlstd{\ \ \ \ \ }\hlstd{\\
\hllin{27\ }}\hlstd{\ \ \ \ \ \ \ \ }\hlstd{}\hlkwa{OUT}\hlstd{\ \ \ \ \ \ \ }\hlkwa{}\hlstd{}\hlkwb{DX}\hlstd{}\hlopt{,}\hlstd{}\hlkwb{AL}\\
\hllin{28\ }\hlstd{}\hlstd{\ \ \ \ \ \ \ \ }\hlstd{}\hlkwa{MOV}\hlstd{\ \ \ \ \ \ \ }\hlkwa{}\hlstd{}\hlkwb{AL}\hlstd{}\hlopt{,}\hlstd{}\hlnum{00000011}\hlstd{B\\
\hllin{29\ }}\hlstd{\ \ \ \ \ \ \ \ }\hlstd{}\hlkwa{MOV}\hlstd{\ \ \ \ \ \ \ }\hlkwa{}\hlstd{}\hlkwb{DX}\hlstd{}\hlopt{,}\hlstd{}\hlnum{3}\hlstd{FCH\\
\hllin{30\ }}\hlstd{\ \ \ \ \ \ \ \ }\hlstd{}\hlkwa{OUT}\hlstd{\ \ \ \ \ \ \ }\hlkwa{}\hlstd{}\hlkwb{DX}\hlstd{}\hlopt{,}\hlstd{}\hlkwb{AL}\\
\hllin{31\ }\hlstd{}\hlstd{\ \ \ \ \ \ \ \ }\hlstd{}\hlkwa{MOV}\hlstd{\ \ \ \ \ \ \ }\hlkwa{}\hlstd{}\hlkwb{AL}\hlstd{}\hlopt{,}\hlstd{}\hlnum{0}\\
\hllin{32\ }\hlstd{}\hlstd{\ \ \ \ \ \ \ \ }\hlstd{}\hlkwa{MOV}\hlstd{\ \ \ \ \ \ \ }\hlkwa{}\hlstd{}\hlkwb{DX}\hlstd{}\hlopt{,}\hlstd{}\hlnum{3}\hlstd{F9H\\
\hllin{33\ }}\hlstd{\ \ \ \ \ \ \ \ }\hlstd{}\hlkwa{OUT}\hlstd{\ \ \ \ \ \ \ }\hlkwa{}\hlstd{}\hlkwb{DX}\hlstd{}\hlopt{,}\hlstd{}\hlkwb{AL}\hlstd{\ \ \ }\hlkwb{}\\
\hllin{34\ }\hlstd{}\hlkwc{WAIT\textunderscore FOR:}\\
\hllin{35\ }\hlstd{}\hlstd{\ \ \ \ \ \ \ \ }\hlstd{}\hlkwa{MOV}\hlstd{\ \ \ \ \ \ \ }\hlkwa{}\hlstd{}\hlkwb{DX}\hlstd{}\hlopt{,}\hlstd{}\hlnum{3}\hlstd{FDH\\
\hllin{36\ }}\hlstd{\ \ \ \ \ \ \ \ }\hlstd{}\hlkwa{IN}\hlstd{\ \ \ \ \ \ \ \ }\hlkwa{}\hlstd{}\hlkwb{AL}\hlstd{}\hlopt{,}\hlstd{}\hlkwb{DX}\\
\hllin{37\ }\hlstd{}\hlstd{\ \ \ \ \ \ \ \ }\hlstd{}\hlkwa{TEST}\hlstd{\ \ \ \ \ \ }\hlkwa{}\hlstd{}\hlkwb{AL}\hlstd{}\hlopt{,}\hlstd{}\hlnum{1}\hlstd{EH\ \\
\hllin{38\ }}\hlstd{\ \ \ \ \ \ \ \ }\hlstd{}\hlkwa{JNZ}\hlstd{\ \ \ \ \ \ \ }\hlkwa{}\hlstd{ERROR\\
\hllin{39\ }}\hlstd{\ \ \ \ \ \ \ \ }\hlstd{}\hlkwa{TEST}\hlstd{\ \ \ \ \ \ }\hlkwa{}\hlstd{}\hlkwb{AL}\hlstd{}\hlopt{,}\hlstd{}\hlnum{1\ }\\
\hllin{40\ }\hlstd{}\hlstd{\ \ \ \ \ \ \ \ }\hlstd{}\hlkwa{JNZ}\hlstd{\ \ \ \ \ \ \ }\hlkwa{}\hlstd{RECEIVE\\
\hllin{41\ }}\hlstd{\ \ \ \ \ \ \ \ }\hlstd{}\hlkwa{TEST}\hlstd{\ \ \ \ \ \ }\hlkwa{}\hlstd{}\hlkwb{AL}\hlstd{}\hlopt{,}\hlstd{}\hlnum{20H\ }\\
\hllin{42\ }\hlstd{}\hlstd{\ \ \ \ \ \ \ \ }\hlstd{}\hlkwa{JZ}\hlstd{\ \ \ \ \ \ \ \ }\hlkwa{}\hlstd{WAIT\textunderscore FOR\ \\
\hllin{43\ }}\hlstd{\ \ \ \ \ \ \ \ }\hlstd{}\hlkwa{MOV}\hlstd{\ \ \ \ \ \ \ }\hlkwa{}\hlstd{}\hlkwb{AH}\hlstd{}\hlopt{,}\hlstd{}\hlnum{1\ }\\
\hllin{44\ }\hlstd{}\hlstd{\ \ \ \ \ \ \ \ }\hlstd{}\hlkwa{INT}\hlstd{\ \ \ \ \ \ \ }\hlkwa{}\hlstd{}\hlnum{16H}\hlstd{\ \ \ \ \ \ \ \ \ }\hlnum{}\\
\hllin{45\ }\hlstd{}\hlstd{\ \ \ \ \ \ \ \ }\hlstd{}\hlkwa{JZ}\hlstd{\ \ \ \ \ \ \ \ }\hlkwa{}\hlstd{WAIT\textunderscore FOR\\
\hllin{46\ }}\hlstd{\ \ \ \ \ \ \ \ }\hlstd{}\hlkwa{MOV}\hlstd{\ \ \ \ \ \ \ }\hlkwa{}\hlstd{}\hlkwb{AH}\hlstd{}\hlopt{,}\hlstd{}\hlnum{1\ }\\
\hllin{47\ }\hlstd{}\hlstd{\ \ \ \ \ \ \ \ }\hlstd{}\hlkwa{INT}\hlstd{\ \ \ \ \ \ \ }\hlkwa{}\hlstd{}\hlnum{21H}\\
\hllin{48\ }\hlstd{}\hlstd{\ \ \ \ \ \ \ \ }\hlstd{}\hlkwa{CMP}\hlstd{\ \ \ \ \ \ \ }\hlkwa{}\hlstd{}\hlkwb{AL}\hlstd{}\hlopt{,}\hlstd{}\hlnum{0}\hlstd{}\hlkwb{DH}\\
\hllin{49\ }\hlstd{}\hlstd{\ \ \ \ \ \ \ \ }\hlstd{}\hlkwa{JNZ}\hlstd{\ \ \ \ \ \ \ }\hlkwa{}\hlstd{SENDCHAR}\hlstd{\ \ \ \ }\hlstd{\\
\hllin{50\ }}\hlstd{\ \ \ \ \ \ \ \ }\hlstd{}\hlkwa{MOV}\hlstd{\ \ \ \ \ \ \ }\hlkwa{}\hlstd{}\hlkwb{AL}\hlstd{}\hlopt{,}\hlstd{}\hlnum{0}\hlstd{}\hlkwb{AH\ }\\
\hllin{51\ }\hlstd{}\hlstd{\ \ \ \ \ \ \ \ }\hlstd{}\hlkwa{MOV}\hlstd{\ \ \ \ \ \ \ }\hlkwa{}\hlstd{}\hlkwb{AH}\hlstd{}\hlopt{,}\hlstd{}\hlnum{0}\hlstd{EH\\
\hllin{52\ }}\hlstd{\ \ \ \ \ \ \ \ }\hlstd{}\hlkwa{INT}\hlstd{\ \ \ \ \ \ \ }\hlkwa{}\hlstd{}\hlnum{10H}\\
\hllin{53\ }\hlstd{}\hlstd{\ \ \ \ \ \ \ \ }\hlstd{}\\
\hllin{54\ }\hlkwc{SENDCHAR:}\hlstd{\ \ }\hlkwc{}\\
\hllin{55\ }\hlstd{}\hlstd{\ \ \ \ \ \ \ \ }\hlstd{}\hlkwa{MOV}\hlstd{\ \ \ \ \ \ \ }\hlkwa{}\hlstd{}\hlkwb{DX}\hlstd{}\hlopt{,}\hlstd{}\hlnum{3}\hlstd{F8H\\
\hllin{56\ }}\hlstd{\ \ \ \ \ \ \ \ }\hlstd{}\hlkwa{OUT}\hlstd{\ \ \ \ \ \ \ }\hlkwa{}\hlstd{}\hlkwb{DX}\hlstd{}\hlopt{,}\hlstd{}\hlkwb{AL}\\
\hllin{57\ }\hlstd{}\hlstd{\ \ \ \ \ \ \ \ }\hlstd{}\hlkwa{CMP}\hlstd{\ \ \ \ \ \ \ }\hlkwa{}\hlstd{}\hlkwb{AL}\hlstd{}\hlopt{,}\hlstd{}\hlnum{20H}\\
\hllin{58\ }\hlstd{}\hlstd{\ \ \ \ \ \ \ \ }\hlstd{}\hlkwa{JNZ}\hlstd{\ \ \ \ \ \ \ }\hlkwa{}\hlstd{NO\textunderscore STOP\\
\hllin{59\ }}\hlstd{\ \ \ \ \ \ \ \ }\hlstd{}\hlkwa{MOV}\hlstd{\ \ \ \ \ \ \ }\hlkwa{}\hlstd{}\hlkwb{AH}\hlstd{}\hlopt{,}\hlstd{}\hlnum{4}\hlstd{}\hlkwb{CH}\\
\hllin{60\ }\hlstd{}\hlstd{\ \ \ \ \ \ \ \ }\hlstd{}\hlkwa{INT}\hlstd{\ \ \ \ \ \ \ }\hlkwa{}\hlstd{}\hlnum{21H}\\
\hllin{61\ }\hlstd{}\hlkwc{NO\textunderscore STOP:}\hlstd{\ \ }\hlkwc{}\\
\hllin{62\ }\hlstd{}\hlstd{\ \ \ \ \ \ \ \ }\hlstd{}\hlkwa{JMP}\hlstd{\ \ \ \ \ \ \ }\hlkwa{}\hlstd{WAIT\textunderscore FOR}\hlstd{\ \ \ \ }\hlstd{}\\
\hllin{63\ }\hlkwc{RECEIVE:}\\
\hllin{64\ }\hlstd{}\hlstd{\ \ \ \ \ \ \ \ }\hlstd{}\hlkwa{MOV}\hlstd{\ \ \ \ \ \ \ }\hlkwa{}\hlstd{}\hlkwb{DX}\hlstd{}\hlopt{,}\hlstd{}\hlnum{3}\hlstd{F8H\\
\hllin{65\ }}\hlstd{\ \ \ \ \ \ \ \ }\hlstd{}\hlkwa{IN}\hlstd{\ \ \ \ \ \ \ \ }\hlkwa{}\hlstd{}\hlkwb{AL}\hlstd{}\hlopt{,}\hlstd{}\hlkwb{DX}\\
\hllin{66\ }\hlstd{}\hlstd{\ \ \ \ \ \ \ \ }\hlstd{}\hlkwa{CMP}\hlstd{\ \ \ \ \ \ \ }\hlkwa{}\hlstd{}\hlkwb{AL}\hlstd{}\hlopt{,}\hlstd{}\hlnum{20H}\\
\hllin{67\ }\hlstd{}\hlstd{\ \ \ \ \ \ \ \ }\hlstd{}\hlkwa{JNZ}\hlstd{\ \ \ \ \ \ \ }\hlkwa{}\hlstd{CHAR\\
\hllin{68\ }}\hlstd{\ \ \ \ \ \ \ \ }\hlstd{}\hlkwa{MOV}\hlstd{\ \ \ \ \ \ \ }\hlkwa{}\hlstd{}\hlkwb{AH}\hlstd{}\hlopt{,}\hlstd{}\hlnum{4}\hlstd{}\hlkwb{CH}\\
\hllin{69\ }\hlstd{}\hlstd{\ \ \ \ \ \ \ \ }\hlstd{}\hlkwa{INT}\hlstd{\ \ \ \ \ \ \ }\hlkwa{}\hlstd{}\hlnum{21H}\\
\hllin{70\ }\hlstd{}\hlkwc{CHAR:}\hlstd{\ \ \ }\hlkwc{}\hlstd{}\hlkwa{PUSH}\hlstd{\ \ \ \ \ \ }\hlkwa{}\hlstd{}\hlkwb{AX}\hlstd{\ \ \ \ \ \ \ \ \ }\hlkwb{}\\
\hllin{71\ }\hlstd{}\hlstd{\ \ \ \ \ \ \ \ }\hlstd{}\hlkwa{MOV}\hlstd{\ \ \ \ \ \ \ }\hlkwa{}\hlstd{}\hlkwb{AH}\hlstd{}\hlopt{,}\hlstd{}\hlnum{0}\hlstd{EH\\
\hllin{72\ }}\hlstd{\ \ \ \ \ \ \ \ }\hlstd{}\hlkwa{INT}\hlstd{\ \ \ \ \ \ \ }\hlkwa{}\hlstd{}\hlnum{10H}\\
\hllin{73\ }\hlstd{}\hlstd{\ \ \ \ \ \ \ \ }\hlstd{}\hlkwa{POP}\hlstd{\ \ \ \ \ \ \ }\hlkwa{}\hlstd{}\hlkwb{AX}\\
\hllin{74\ }\hlstd{}\hlstd{\ \ \ \ \ \ \ \ }\hlstd{}\hlkwa{CMP}\hlstd{\ \ \ \ \ \ \ }\hlkwa{}\hlstd{}\hlkwb{AL}\hlstd{}\hlopt{,}\hlstd{}\hlnum{0}\hlstd{}\hlkwb{DH}\\
\hllin{75\ }\hlstd{}\hlstd{\ \ \ \ \ \ \ \ }\hlstd{}\hlkwa{JNZ}\hlstd{\ \ \ \ \ \ \ }\hlkwa{}\hlstd{WAIT\textunderscore FOR\\
\hllin{76\ }}\hlstd{\ \ \ \ \ \ \ \ }\hlstd{}\hlkwa{MOV}\hlstd{\ \ \ \ \ \ \ }\hlkwa{}\hlstd{}\hlkwb{AL}\hlstd{}\hlopt{,}\hlstd{}\hlnum{0}\hlstd{}\hlkwb{AH}\\
\hllin{77\ }\hlstd{}\hlstd{\ \ \ \ \ \ \ \ }\hlstd{}\hlkwa{MOV}\hlstd{\ \ \ \ \ \ \ }\hlkwa{}\hlstd{}\hlkwb{AH}\hlstd{}\hlopt{,}\hlstd{}\hlnum{0}\hlstd{EH}\hlstd{\ \ \ \ }\hlstd{\\
\hllin{78\ }}\hlstd{\ \ \ \ \ \ \ \ }\hlstd{}\hlkwa{INT}\hlstd{\ \ \ \ \ \ \ }\hlkwa{}\hlstd{}\hlnum{10H}\\
\hllin{79\ }\hlstd{}\hlstd{\ \ \ \ \ \ \ \ }\hlstd{}\hlkwa{JMP}\hlstd{\ \ \ \ \ \ \ }\hlkwa{}\hlstd{WAIT\textunderscore FOR}\\
\hllin{80\ }\hlkwc{ERROR:}\hlstd{\ \ }\hlkwc{}\hlstd{}\hlkwa{MOV}\hlstd{\ \ \ \ \ \ \ }\hlkwa{}\hlstd{}\hlkwb{DX}\hlstd{}\hlopt{,}\hlstd{}\hlnum{3}\hlstd{F8H\\
\hllin{81\ }}\hlstd{\ \ \ \ \ \ \ \ }\hlstd{}\hlkwa{IN}\hlstd{\ \ \ \ \ \ \ \ }\hlkwa{}\hlstd{}\hlkwb{AL}\hlstd{}\hlopt{,}\hlstd{}\hlkwb{DX}\\
\hllin{82\ }\hlstd{}\hlstd{\ \ \ \ \ \ \ \ }\hlstd{}\hlkwa{MOV}\hlstd{\ \ \ \ \ \ \ }\hlkwa{}\hlstd{}\hlkwb{AL}\hlstd{}\hlopt{,}\hlstd{}\hlstr{'?'}\hlstd{}\hlstd{\ \ \ \ \ \ }\hlstd{\\
\hllin{83\ }}\hlstd{\ \ \ \ \ \ \ \ }\hlstd{}\hlkwa{MOV}\hlstd{\ \ \ \ \ \ \ }\hlkwa{}\hlstd{}\hlkwb{AH}\hlstd{}\hlopt{,}\hlstd{}\hlnum{14}\\
\hllin{84\ }\hlstd{}\hlstd{\ \ \ \ \ \ \ \ }\hlstd{}\hlkwa{INT}\hlstd{\ \ \ \ \ \ \ }\hlkwa{}\hlstd{}\hlnum{10H}\\
\hllin{85\ }\hlstd{}\hlstd{\ \ \ \ \ \ \ \ }\hlstd{}\hlkwa{JMP}\hlstd{\ \ \ \ \ \ \ }\hlkwa{}\hlstd{WAIT\textunderscore FOR\\
\hllin{86\ }SUB1}\hlstd{\ \ }\hlstd{}\hlkwa{ENDP}\\
\hllin{87\ }\hlstd{\\
\hllin{88\ }CODES}\hlstd{\ \ }\hlstd{}\hlkwa{ENDS}\\
\hllin{89\ }\hlstd{}\hlkwa{END}\hlstd{\ \ \ \ }\hlkwa{}\hlstd{START}\\
\mbox{}
\normalfont
\normalsize

\section{选做部分}
监视键盘,若键入字母键“S”则将事先存在数据区中的一个字符串串行传送给对方显示;若键入字母“R”
则将对方机器数据区中的一个字符串传送过来在CRT 上显示。两个字符串都以\$ 为结束符。\\ 

\noindent
\ttfamily
\hlstd{\hllin{01\ }DATA\ }\hlkwa{SEGMENT}\\
\hllin{02\ }\hlstd{}\hlstd{\ \ \ \ \ }\hlstd{DIVID\ }\hlkwa{DW\ }\hlstd{}\hlnum{60}\\
\hllin{03\ }\hlstd{}\hlstd{\ \ \ \ \ }\hlstd{MY\textunderscore STR\ }\hlkwa{DB\ }\hlstd{}\hlstr{'gq'}\hlstd{}\hlopt{,}\hlstd{}\hlnum{0}\hlstd{}\hlkwb{DH}\hlstd{}\hlopt{,}\hlstd{}\hlstr{'\$'}\hlstd{\\
\hllin{04\ }}\hlstd{\ \ \ \ \ }\hlstd{\\
\hllin{05\ }}\hlstd{\ \ \ \ \ }\hlstd{MESS2\ }\hlkwa{DB\ }\hlstd{}\hlstr{'PROGRAM\ DONE!'}\hlstd{}\hlopt{,}\hlstd{}\hlnum{0}\hlstd{}\hlkwb{DH}\hlstd{}\hlopt{,}\hlstd{}\hlstr{'\$'}\hlstd{\\
\hllin{06\ }}\hlstd{\ \ \ \ \ }\hlstd{CRLF}\hlstd{\ \ }\hlstd{}\hlkwa{DB\ }\hlstd{}\hlnum{0}\hlstd{}\hlkwb{DH}\hlstd{}\hlopt{,}\hlstd{}\hlnum{0}\hlstd{}\hlkwb{AH}\hlstd{}\hlopt{,}\hlstd{}\hlstr{'\$'}\hlstd{\\
\hllin{07\ }DATA\ }\hlkwa{ENDS}\\
\hllin{08\ }\hlstd{STACK1\ }\hlkwa{SEGMENT\ }\hlstd{PARA\ STACK\\
\hllin{09\ }}\hlstd{\ \ \ \ \ \ }\hlstd{}\hlkwa{DB\ }\hlstd{}\hlnum{100\ }\hlstd{DUP}\hlopt{(}\hlstd{?}\hlopt{)}\\
\hllin{10\ }\hlstd{STACK1\ }\hlkwa{ENDS}\\
\hllin{11\ }\hlstd{CODE\ }\hlkwa{SEGMENT}\\
\hllin{12\ }\hlstd{}\hlstd{\ \ \ \ \ }\hlstd{}\hlkwa{ASSUME\ }\hlstd{}\hlkwb{CS}\hlstd{}\hlopt{:}\hlstd{CODE}\hlopt{,}\hlstd{}\hlkwb{DS}\hlstd{}\hlopt{:}\hlstd{DATA}\hlopt{,}\hlstd{}\hlkwb{ES}\hlstd{}\hlopt{:}\hlstd{DATA}\hlopt{,}\hlstd{}\hlkwb{SS}\hlstd{}\hlopt{:}\hlstd{STACK1\\
\hllin{13\ }DOUBLE\ }\hlkwa{PROC}\\
\hllin{14\ }\hlstd{}\hlkwc{START:}\hlstd{}\hlkwa{MOV\ }\hlstd{}\hlkwb{AX}\hlstd{}\hlopt{,}\hlstd{DATA\\
\hllin{15\ }}\hlstd{\ \ \ \ \ \ }\hlstd{}\hlkwa{MOV\ }\hlstd{}\hlkwb{DS}\hlstd{}\hlopt{,}\hlstd{}\hlkwb{AX}\\
\hllin{16\ }\hlstd{}\hlstd{\ \ \ \ \ \ }\hlstd{}\hlkwa{MOV\ }\hlstd{}\hlkwb{ES}\hlstd{}\hlopt{,}\hlstd{}\hlkwb{AX}\\
\hllin{17\ }\hlstd{}\hlstd{\ \ \ \ \ \ }\hlstd{}\hlkwa{MOV\ }\hlstd{}\hlkwb{DX}\hlstd{}\hlopt{,}\hlstd{}\hlnum{3}\hlstd{FBH\\
\hllin{18\ }}\hlstd{\ \ \ \ \ \ }\hlstd{}\hlkwa{MOV\ }\hlstd{}\hlkwb{AL}\hlstd{}\hlopt{,}\hlstd{}\hlnum{80H}\\
\hllin{19\ }\hlstd{}\hlstd{\ \ \ \ \ \ }\hlstd{}\hlkwa{OUT\ }\hlstd{}\hlkwb{DX}\hlstd{}\hlopt{,}\hlstd{}\hlkwb{AL}\\
\hllin{20\ }\hlstd{}\hlstd{\ \ \ \ \ \ }\hlstd{}\hlkwa{MOV\ }\hlstd{}\hlkwb{AX}\hlstd{}\hlopt{,}\hlstd{DIVID\\
\hllin{21\ }}\hlstd{\ \ \ \ \ \ }\hlstd{}\hlkwa{MOV\ }\hlstd{}\hlkwb{DX}\hlstd{}\hlopt{,}\hlstd{}\hlnum{3}\hlstd{F8H\\
\hllin{22\ }}\hlstd{\ \ \ \ \ \ }\hlstd{}\hlkwa{OUT\ }\hlstd{}\hlkwb{DX}\hlstd{}\hlopt{,}\hlstd{}\hlkwb{AL}\\
\hllin{23\ }\hlstd{}\hlstd{\ \ \ \ \ \ }\hlstd{}\hlkwa{MOV\ }\hlstd{}\hlkwb{AL}\hlstd{}\hlopt{,}\hlstd{}\hlkwb{AH}\\
\hllin{24\ }\hlstd{}\hlstd{\ \ \ \ \ \ }\hlstd{}\hlkwa{MOV\ }\hlstd{}\hlkwb{DX}\hlstd{}\hlopt{,}\hlstd{}\hlnum{3}\hlstd{F9H\\
\hllin{25\ }}\hlstd{\ \ \ \ \ \ }\hlstd{}\hlkwa{OUT\ }\hlstd{}\hlkwb{DX}\hlstd{}\hlopt{,}\hlstd{}\hlkwb{AL}\\
\hllin{26\ }\hlstd{}\hlstd{\ \ \ \ \ \ }\hlstd{}\hlkwa{MOV\ }\hlstd{}\hlkwb{AL}\hlstd{}\hlopt{,}\hlstd{}\hlnum{0}\hlstd{}\hlkwb{BH}\\
\hllin{27\ }\hlstd{}\hlstd{\ \ \ \ \ \ }\hlstd{}\hlkwa{MOV\ }\hlstd{}\hlkwb{DX}\hlstd{}\hlopt{,}\hlstd{}\hlnum{3}\hlstd{FBH\\
\hllin{28\ }}\hlstd{\ \ \ \ \ \ }\hlstd{}\hlkwa{OUT\ }\hlstd{}\hlkwb{DX}\hlstd{}\hlopt{,}\hlstd{}\hlkwb{AL}\\
\hllin{29\ }\hlstd{}\hlstd{\ \ \ \ \ \ }\hlstd{}\hlkwa{MOV\ }\hlstd{}\hlkwb{DX}\hlstd{}\hlopt{,}\hlstd{}\hlnum{3}\hlstd{FCH\\
\hllin{30\ }}\hlstd{\ \ \ \ \ \ }\hlstd{}\hlkwa{MOV\ }\hlstd{}\hlkwb{AL}\hlstd{}\hlopt{,}\hlstd{}\hlnum{00000011}\hlstd{B\\
\hllin{31\ }}\hlstd{\ \ \ \ \ \ }\hlstd{}\hlkwa{OUT\ }\hlstd{}\hlkwb{DX}\hlstd{}\hlopt{,}\hlstd{}\hlkwb{AL}\\
\hllin{32\ }\hlstd{}\hlstd{\ \ \ \ \ \ }\hlstd{}\hlkwa{MOV\ }\hlstd{}\hlkwb{DX}\hlstd{}\hlopt{,}\hlstd{}\hlnum{3}\hlstd{F9H\\
\hllin{33\ }}\hlstd{\ \ \ \ \ \ }\hlstd{}\hlkwa{MOV\ }\hlstd{}\hlkwb{AL}\hlstd{}\hlopt{,}\hlstd{}\hlnum{0}\\
\hllin{34\ }\hlstd{}\hlstd{\ \ \ \ \ \ }\hlstd{}\hlkwa{OUT\ }\hlstd{}\hlkwb{DX}\hlstd{}\hlopt{,}\hlstd{}\hlkwb{AL}\\
\hllin{35\ }\hlstd{}\\
\hllin{36\ }\hlkwc{KEEP\textunderscore TRY:}\hlstd{}\hlkwa{MOV\ }\hlstd{}\hlkwb{DX}\hlstd{}\hlopt{,}\hlstd{}\hlnum{3}\hlstd{FDH\\
\hllin{37\ }}\hlstd{\ \ \ \ \ \ \ \ \ }\hlstd{}\hlkwa{IN\ }\hlstd{}\hlkwb{AL}\hlstd{}\hlopt{,}\hlstd{}\hlkwb{DX}\\
\hllin{38\ }\hlstd{}\hlstd{\ \ \ \ \ \ \ \ \ }\hlstd{}\hlkwa{TEST\ }\hlstd{}\hlkwb{AL}\hlstd{}\hlopt{,}\hlstd{}\hlnum{1}\hlstd{EH\\
\hllin{39\ }}\hlstd{\ \ \ \ \ \ \ \ \ }\hlstd{}\hlkwa{JNE}\hlstd{\ \ }\hlkwa{}\hlstd{ERROR\\
\hllin{40\ }}\hlstd{\ \ \ \ \ \ \ \ \ }\hlstd{}\hlkwa{TEST\ }\hlstd{}\hlkwb{AL}\hlstd{}\hlopt{,}\hlstd{}\hlnum{1}\\
\hllin{41\ }\hlstd{}\hlstd{\ \ \ \ \ \ \ \ \ }\hlstd{}\hlkwa{JNZ}\hlstd{\ \ }\hlkwa{}\hlstd{RECEIVE\\
\hllin{42\ }}\hlstd{\ \ \ \ \ \ \ \ \ }\hlstd{}\hlkwa{TEST\ }\hlstd{}\hlkwb{AL}\hlstd{}\hlopt{,}\hlstd{}\hlnum{20H}\\
\hllin{43\ }\hlstd{}\hlstd{\ \ \ \ \ \ \ \ \ }\hlstd{}\hlkwa{JZ\ }\hlstd{KEEP\textunderscore TRY\\
\hllin{44\ }}\hlstd{\ \ \ \ \ \ \ \ \ }\hlstd{}\hlkwa{MOV\ }\hlstd{}\hlkwb{AH}\hlstd{}\hlopt{,}\hlstd{}\hlnum{1}\\
\hllin{45\ }\hlstd{}\hlstd{\ \ \ \ \ \ \ \ \ }\hlstd{}\hlkwa{INT\ }\hlstd{}\hlnum{16H}\\
\hllin{46\ }\hlstd{}\hlstd{\ \ \ \ \ \ \ \ \ }\hlstd{}\hlkwa{JZ\ }\hlstd{KEEP\textunderscore TRY\\
\hllin{47\ }}\hlstd{\ \ \ \ \ \ \ \ \ }\hlstd{}\hlkwa{MOV\ }\hlstd{}\hlkwb{AH}\hlstd{}\hlopt{,}\hlstd{}\hlnum{1}\\
\hllin{48\ }\hlstd{}\hlstd{\ \ \ \ \ \ \ \ \ }\hlstd{}\hlkwa{INT\ }\hlstd{}\hlnum{21H}\\
\hllin{49\ }\hlstd{}\hlstd{\ \ \ \ \ \ \ \ \ }\hlstd{}\hlkwa{MOV\ }\hlstd{}\hlkwb{DX}\hlstd{}\hlopt{,}\hlstd{}\hlnum{3}\hlstd{F8H\\
\hllin{50\ }}\hlstd{\ \ \ \ \ \ \ \ \ }\hlstd{}\hlkwa{OUT\ }\hlstd{}\hlkwb{DX}\hlstd{}\hlopt{,}\hlstd{}\hlkwb{AL}\\
\hllin{51\ }\hlstd{}\hlstd{\ \ \ \ \ \ \ \ \ }\hlstd{}\hlkwa{CMP\ }\hlstd{}\hlkwb{AL}\hlstd{}\hlopt{,}\hlstd{}\hlnum{20H}\\
\hllin{52\ }\hlstd{}\hlstd{\ \ \ \ \ \ \ \ \ }\hlstd{}\hlkwa{JE\ }\hlstd{EXIT\\
\hllin{53\ }}\hlstd{\ \ \ \ \ \ \ \ \ }\hlstd{}\hlkwa{CMP\ }\hlstd{}\hlkwb{AL}\hlstd{}\hlopt{,}\hlstd{}\hlstr{'S'}\hlstd{\\
\hllin{54\ }}\hlstd{\ \ \ \ \ \ \ \ \ }\hlstd{}\hlkwa{JE\ }\hlstd{SEND\textunderscore STR\\
\hllin{55\ }}\hlstd{\ \ \ }\hlstd{}\hlkwa{CMP\ }\hlstd{}\hlkwb{AL}\hlstd{}\hlopt{,}\hlstd{}\hlstr{'s'}\hlstd{\\
\hllin{56\ }}\hlstd{\ \ \ \ \ \ \ \ \ }\hlstd{}\hlkwa{JE\ }\hlstd{SEND\textunderscore STR\\
\hllin{57\ }}\hlstd{\ \ \ \ \ \ \ \ \ }\hlstd{}\hlkwa{CMP\ }\hlstd{}\hlkwb{AL}\hlstd{}\hlopt{,}\hlstd{}\hlstr{'R'}\hlstd{\\
\hllin{58\ }}\hlstd{\ \ \ \ \ \ \ \ \ }\hlstd{}\hlkwa{JE\ }\hlstd{RE\textunderscore STR\\
\hllin{59\ }}\hlstd{\ \ \ \ \ \ \ \ \ }\hlstd{}\hlkwa{CMP\ }\hlstd{}\hlkwb{AL}\hlstd{}\hlopt{,}\hlstd{}\hlstr{'r'}\hlstd{\\
\hllin{60\ }}\hlstd{\ \ \ \ \ \ \ \ \ }\hlstd{}\hlkwa{JE\ }\hlstd{RE\textunderscore STR\\
\hllin{61\ }}\hlstd{\ \ \ \ \ \ \ \ \ }\hlstd{}\hlkwa{JMP\ }\hlstd{KEEP\textunderscore TRY}\\
\hllin{62\ }\\
\hllin{63\ }\hlkwc{RECEIVE:\ }\hlstd{}\hlkwa{MOV\ }\hlstd{}\hlkwb{DX}\hlstd{}\hlopt{,}\hlstd{}\hlnum{3}\hlstd{F8H\\
\hllin{64\ }}\hlstd{\ \ \ \ \ \ \ \ \ }\hlstd{}\hlkwa{IN\ }\hlstd{}\hlkwb{AL}\hlstd{}\hlopt{,}\hlstd{}\hlkwb{DX}\\
\hllin{65\ }\hlstd{}\hlstd{\ \ \ \ \ \ \ \ \ }\hlstd{}\hlkwa{CMP\ }\hlstd{}\hlkwb{AL}\hlstd{}\hlopt{,}\hlstd{}\hlnum{20H}\\
\hllin{66\ }\hlstd{}\hlstd{\ \ \ \ \ \ \ \ \ }\hlstd{}\hlkwa{JE\ }\hlstd{EXIT\\
\hllin{67\ }}\hlstd{\ \ \ \ \ \ \ \ \ }\hlstd{}\hlkwa{CMP\ }\hlstd{}\hlkwb{AL}\hlstd{}\hlopt{,}\hlstd{}\hlstr{'S'}\hlstd{\\
\hllin{68\ }}\hlstd{\ \ \ \ \ \ \ \ \ }\hlstd{}\hlkwa{JE\ }\hlstd{RE\textunderscore STR}\hlstd{\ \ \ \ \ \ \ \ }\hlstd{\\
\hllin{69\ }}\hlstd{\ \ \ \ \ \ \ \ \ }\hlstd{}\hlkwa{CMP\ }\hlstd{}\hlkwb{AL}\hlstd{}\hlopt{,}\hlstd{}\hlstr{'R'}\hlstd{\\
\hllin{70\ }}\hlstd{\ \ \ \ \ \ \ \ \ }\hlstd{}\hlkwa{JE\ }\hlstd{SEND\textunderscore STR\\
\hllin{71\ }}\hlstd{\ \ \ \ \ \ \ \ \ }\hlstd{}\hlkwa{JMP\ }\hlstd{KEEP\textunderscore TRY}\\
\hllin{72\ }\\
\hllin{73\ }\hlkwc{ERROR:}\hlstd{\ \ \ }\hlkwc{}\hlstd{}\hlkwa{MOV\ }\hlstd{}\hlkwb{DX}\hlstd{}\hlopt{,}\hlstd{}\hlnum{3}\hlstd{F8H\\
\hllin{74\ }}\hlstd{\ \ \ \ \ \ \ \ \ }\hlstd{}\hlkwa{IN\ }\hlstd{}\hlkwb{AL}\hlstd{}\hlopt{,}\hlstd{}\hlkwb{DX}\\
\hllin{75\ }\hlstd{}\hlstd{\ \ \ \ \ \ \ \ \ }\hlstd{}\hlkwa{MOV\ }\hlstd{}\hlkwb{AL}\hlstd{}\hlopt{,}\hlstd{}\hlstr{'?'}\hlstd{\\
\hllin{76\ }}\hlstd{\ \ \ \ \ \ \ \ \ }\hlstd{}\hlkwa{MOV\ }\hlstd{}\hlkwb{AH}\hlstd{}\hlopt{,}\hlstd{}\hlnum{14}\\
\hllin{77\ }\hlstd{}\hlstd{\ \ \ \ \ \ \ \ \ }\hlstd{}\hlkwa{INT\ }\hlstd{}\hlnum{10H}\\
\hllin{78\ }\hlstd{}\hlstd{\ \ \ \ \ \ \ \ \ }\hlstd{}\hlkwa{JMP\ }\hlstd{KEEP\textunderscore TRY\\
\hllin{79\ }}\hlstd{\ \ \ \ \ \ \ \ \ }\hlstd{}\\
\hllin{80\ }\hlkwc{SEND\textunderscore STR:}\hlstd{}\hlkwa{CALL\ }\hlstd{SEND\\
\hllin{81\ }}\hlstd{\ \ \ \ \ \ \ \ \ }\hlstd{}\hlkwa{JMP\ }\hlstd{KEEP\textunderscore TRY}\\
\hllin{82\ }\hlkwc{RE\textunderscore STR:}\hlstd{\ \ }\hlkwc{}\hlstd{}\hlkwa{CALL\ }\hlstd{REC\\
\hllin{83\ }}\hlstd{\ \ \ \ \ \ \ \ \ }\hlstd{}\hlkwa{JMP\ }\hlstd{KEEP\textunderscore TRY\\
\hllin{84\ }}\hlstd{\ \ \ \ \ \ \ \ \ }\hlstd{}\\
\hllin{85\ }\hlkwc{EXIT:}\hlstd{\ \ \ \ }\hlkwc{}\hlstd{}\hlkwa{LEA\ }\hlstd{}\hlkwb{DX}\hlstd{}\hlopt{,}\hlstd{MESS2\\
\hllin{86\ }}\hlstd{\ \ \ \ \ \ \ \ \ }\hlstd{}\hlkwa{MOV\ }\hlstd{}\hlkwb{AH}\hlstd{}\hlopt{,}\hlstd{}\hlnum{9}\\
\hllin{87\ }\hlstd{}\hlstd{\ \ \ \ \ \ \ \ \ }\hlstd{}\hlkwa{INT\ }\hlstd{}\hlnum{21H}\\
\hllin{88\ }\hlstd{}\hlstd{\ \ \ \ \ \ \ \ \ }\hlstd{}\hlkwa{MOV\ }\hlstd{}\hlkwb{AH}\hlstd{}\hlopt{,}\hlstd{}\hlnum{4}\hlstd{}\hlkwb{CH}\\
\hllin{89\ }\hlstd{}\hlstd{\ \ \ \ \ \ \ \ \ }\hlstd{}\hlkwa{INT\ }\hlstd{}\hlnum{21H}\\
\hllin{90\ }\hlstd{DOUBLE}\hlstd{\ \ \ \ \ }\hlstd{}\hlkwa{ENDP}\\
\hllin{91\ }\hlstd{\\
\hllin{92\ }SEND\ }\hlkwa{PROC}\hlstd{\ \ \ \ \ \ \ \ \ \ \ \ \ \ \ \ \ \ \ }\hlkwa{}\\
\hllin{93\ }\hlstd{}\hlstd{\ \ \ \ \ }\hlstd{}\hlkwa{LEA\ }\hlstd{}\hlkwb{SI}\hlstd{}\hlopt{,}\hlstd{MY\textunderscore STR}\\
\hllin{94\ }\hlkwc{S\textunderscore WAIT:}\hlstd{}\hlkwa{MOV\ }\hlstd{}\hlkwb{DX}\hlstd{}\hlopt{,}\hlstd{}\hlnum{3}\hlstd{F8H\\
\hllin{95\ }}\hlstd{\ \ \ \ \ \ \ }\hlstd{}\hlkwa{MOV\ }\hlstd{}\hlkwb{AL}\hlstd{}\hlopt{,{[}}\hlstd{}\hlkwb{SI}\hlstd{}\hlopt{{]}}\\
\hllin{96\ }\hlstd{}\hlstd{\ \ \ \ \ \ \ }\hlstd{}\hlkwa{OUT\ }\hlstd{}\hlkwb{DX}\hlstd{}\hlopt{,}\hlstd{}\hlkwb{AL}\\
\hllin{97\ }\hlstd{}\hlstd{\ \ \ \ \ \ \ }\hlstd{}\hlkwa{CMP\ }\hlstd{}\hlkwb{AL}\hlstd{}\hlopt{,}\hlstd{}\hlstr{'\$'}\hlstd{\\
\hllin{98\ }}\hlstd{\ \ \ \ \ \ \ }\hlstd{}\hlkwa{JE\ }\hlstd{S\textunderscore DONE\\
\hllin{99\ }}\hlstd{\ \ \ \ \ \ \ }\hlstd{}\hlkwa{INC\ }\hlstd{}\hlkwb{SI}\\
\hllin{100\ }\hlstd{}\hlstd{\ \ \ \ \ \ \ }\hlstd{}\hlkwa{JMP\ }\hlstd{S\textunderscore WAIT}\\
\hllin{101\ }\hlkwc{S\textunderscore DONE:}\hlstd{}\hlkwa{RET}\\
\hllin{102\ }\hlstd{SEND\ }\hlkwa{ENDP}\\
\hllin{103\ }\hlstd{\\
\hllin{104\ }REC\ }\hlkwa{PROC}\hlstd{\ \ \ \ \ \ \ \ \ \ \ \ \ \ \ \ \ \ \ \ \ }\hlkwa{}\\
\hllin{105\ }\hlstd{}\hlkwc{REC\textunderscore WAIT:}\\
\hllin{106\ }\hlstd{}\hlstd{\ \ \ \ }\hlstd{}\hlkwa{MOV\ }\hlstd{}\hlkwb{DX}\hlstd{}\hlopt{,}\hlstd{}\hlnum{3}\hlstd{FDH\\
\hllin{107\ }}\hlstd{\ \ \ \ }\hlstd{}\hlkwa{IN\ }\hlstd{}\hlkwb{AL}\hlstd{}\hlopt{,}\hlstd{}\hlkwb{DX}\\
\hllin{108\ }\hlstd{}\hlstd{\ \ \ \ }\hlstd{}\hlkwa{TEST\ }\hlstd{}\hlkwb{AL}\hlstd{}\hlopt{,}\hlstd{}\hlnum{1}\hlstd{EH\\
\hllin{109\ }}\hlstd{\ \ \ \ }\hlstd{}\hlkwa{JNE}\hlstd{\ \ }\hlkwa{}\hlstd{ERROR\\
\hllin{110\ }}\hlstd{\ \ \ \ }\hlstd{}\hlkwa{TEST\ }\hlstd{}\hlkwb{AL}\hlstd{}\hlopt{,}\hlstd{}\hlnum{1}\\
\hllin{111\ }\hlstd{}\hlstd{\ \ \ \ }\hlstd{}\hlkwa{JZ\ }\hlstd{REC\textunderscore WAIT\\
\hllin{112\ }}\hlstd{\ \ \ \ }\hlstd{}\hlkwa{MOV\ }\hlstd{}\hlkwb{DX}\hlstd{}\hlopt{,}\hlstd{}\hlnum{3}\hlstd{F8H\\
\hllin{113\ }}\hlstd{\ \ \ \ }\hlstd{}\hlkwa{IN\ }\hlstd{}\hlkwb{AL}\hlstd{}\hlopt{,}\hlstd{}\hlkwb{DX}\\
\hllin{114\ }\hlstd{}\hlstd{\ \ \ \ }\hlstd{}\hlkwa{CMP\ }\hlstd{}\hlkwb{AL}\hlstd{}\hlopt{,}\hlstd{}\hlstr{'\$'}\hlstd{\\
\hllin{115\ }}\hlstd{\ \ \ \ }\hlstd{}\hlkwa{JE\ }\hlstd{REC\textunderscore DONE\\
\hllin{116\ }}\hlstd{\ \ \ \ }\hlstd{}\hlkwa{MOV\ }\hlstd{}\hlkwb{AH}\hlstd{}\hlopt{,}\hlstd{}\hlnum{0}\hlstd{EH\\
\hllin{117\ }}\hlstd{\ \ \ \ }\hlstd{}\hlkwa{INT\ }\hlstd{}\hlnum{10H}\\
\hllin{118\ }\hlstd{}\hlstd{\ \ \ \ }\hlstd{}\hlkwa{JMP\ }\hlstd{REC\textunderscore WAIT}\\
\hllin{119\ }\hlkwc{REC\textunderscore DONE:}\\
\hllin{120\ }\hlstd{}\hlstd{\ \ \ \ }\hlstd{}\hlkwa{LEA\ }\hlstd{}\hlkwb{DX}\hlstd{}\hlopt{,}\hlstd{CRLF\\
\hllin{121\ }}\hlstd{\ \ \ \ }\hlstd{}\hlkwa{MOV\ }\hlstd{}\hlkwb{AH}\hlstd{}\hlopt{,}\hlstd{}\hlnum{9}\\
\hllin{122\ }\hlstd{}\hlstd{\ \ \ \ }\hlstd{}\hlkwa{INT\ }\hlstd{}\hlnum{21H}\\
\hllin{123\ }\hlstd{}\hlstd{\ \ \ \ }\hlstd{}\hlkwa{RET}\\
\hllin{124\ }\hlstd{REC\ }\hlkwa{ENDP}\\
\hllin{125\ }\hlstd{\\
\hllin{126\ }CODE}\hlstd{\ \ \ \ \ }\hlstd{}\hlkwa{ENDS}\\
\hllin{127\ }\hlstd{}\hlkwa{END}\hlstd{\ \ \ \ \ \ }\hlkwa{}\hlstd{START}\\
\mbox{}
\normalfont
\normalsize

\section{完成情况及心得体会}
这次实验的主要目的是复习8250 的工作原理及其在串行通信中的应用。实验过程中需要用到连接好的电
脑,所以调试也只能在实验室内完成,使得本次实验的难度又有所增加。作为本学期最后一次实验,确实花
费了更多的时间,不过相较于以前更加有趣。本次实验复习了课堂知识,巩固了汇编语言的编写,收获不小。
\end{document}