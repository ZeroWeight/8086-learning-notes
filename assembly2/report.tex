\documentclass[UTF8,a4paper]{ctexart}
\usepackage[utf8]{inputenc}
\usepackage{amsmath}
\usepackage{pdfpages}
\usepackage{graphicx}
\usepackage{wrapfig}
\usepackage{listings}
\usepackage{color}
\usepackage{alltt}
\usepackage{marvosym}
\usepackage{xcolor}
\input {highlight.sty}
\title{计算机原理第二次实验报告}
\author{张蔚桐\ 2015011493\ 自55}
\begin {document}
\maketitle
\section{实验目的}
\begin{enumerate}
\item 巩固DEBUG和宏汇编的使用
\item 加深对运算指令的理解
\item 注意标志寄存器的变化
\end{enumerate}
\section{实验内容}
\subsection{16位二进制数的加减}
在数据区内定义两个16位二进制数,用8位运算指令进行加减运算,首先程序将两个数读入到AX和DX中,之后进行AL和DL的相加,不考虑进位,之后对AH和DH进行相加,考虑进位,得到运算结果

程序如下所示,运行结果如图\ref{p21}所示,我们可以验证,FFFE+FFFF=FFFD,FFFE-FFFF=FFFF结果是正确的。

\pagecolor{bgcolor}
\noindent
\ttfamily
\hlstd{\hllin{01\ }DATA\ }\hlkwa{SEGMENT}\\
\hllin{02\ }\hlstd{X\ }\hlkwa{DW\ }\hlstd{}\hlnum{0}\hlstd{EEFFH\ }\hlslc{;THE\ VALUE\ 0F\ X}\\
\hllin{03\ }\hlstd{Y\ }\hlkwa{DW\ }\hlstd{}\hlnum{0}\hlstd{FFEEH}\hlstd{\ \ }\hlstd{}\hlslc{;THE\ VALUE\ OF\ Y}\\
\hllin{04\ }\hlstd{P\ }\hlkwa{DW\ }\hlstd{}\hlnum{0}\hlstd{FFFFH\ }\hlslc{;X+Y}\\
\hllin{05\ }\hlstd{M\ }\hlkwa{DW\ }\hlstd{}\hlnum{0}\hlstd{FFFFH\ }\hlslc{;X{-}Y}\\
\hllin{06\ }\hlstd{}\hlslc{;Z=X+Y}\\
\hllin{07\ }\hlstd{DATA\ }\hlkwa{ENDS}\\
\hllin{08\ }\hlstd{\\
\hllin{09\ }STACK\ }\hlkwa{SEGMENT\ }\hlstd{PARA\ STACK}\\
\hllin{10\ }\hlkwa{DB\ }\hlstd{}\hlnum{10\ }\hlstd{DUP\ }\hlopt{(}\hlstd{?}\hlopt{)}\\
\hllin{11\ }\hlstd{STACK\ }\hlkwa{ENDS}\\
\hllin{12\ }\hlstd{\\
\hllin{13\ }CODE\ }\hlkwa{SEGMENT}\\
\hllin{14\ }\hlstd{}\hlkwa{ASSUME\ }\hlstd{}\hlkwb{DS}\hlstd{}\hlopt{:}\hlstd{DATA}\hlopt{,}\hlstd{}\hlkwb{ES}\hlstd{}\hlopt{:}\hlstd{DATA}\hlopt{,}\hlstd{}\hlkwb{SS}\hlstd{}\hlopt{:}\hlstd{STACK}\hlopt{,}\hlstd{}\hlkwb{CS}\hlstd{}\hlopt{:}\hlstd{CODE}\\
\hllin{15\ }\hlkwc{START:}\\
\hllin{16\ }\hlstd{}\hlkwa{MOV\ }\hlstd{}\hlkwb{AX}\hlstd{}\hlopt{,}\hlstd{DATA}\\
\hllin{17\ }\hlkwa{MOV\ }\hlstd{}\hlkwb{DS}\hlstd{}\hlopt{,}\hlstd{}\hlkwb{AX}\\
\hllin{18\ }\hlstd{}\hlkwa{MOV\ }\hlstd{}\hlkwb{ES}\hlstd{}\hlopt{,}\hlstd{}\hlkwb{AX}\\
\hllin{19\ }\hlstd{}\hlkwa{MOV\ }\hlstd{}\hlkwb{AX}\hlstd{}\hlopt{,}\hlstd{STACK}\\
\hllin{20\ }\hlkwa{MOV\ }\hlstd{}\hlkwb{SS}\hlstd{}\hlopt{,}\hlstd{}\hlkwb{AX}\\
\hllin{21\ }\hlstd{}\\
\hllin{22\ }\hlslc{;DATA\ READ\ IN}\\
\hllin{23\ }\hlstd{}\hlkwa{MOV\ }\hlstd{}\hlkwb{AX}\hlstd{}\hlopt{,}\hlstd{X}\\
\hllin{24\ }\hlkwa{MOV\ }\hlstd{}\hlkwb{DX}\hlstd{}\hlopt{,}\hlstd{Y}\\
\hllin{25\ }\hlslc{;ADD\ METHOD}\\
\hllin{26\ }\hlstd{}\hlkwa{ADD\ }\hlstd{}\hlkwb{AL}\hlstd{}\hlopt{,}\hlstd{}\hlkwb{DL}\\
\hllin{27\ }\hlstd{}\hlkwa{ADC\ }\hlstd{}\hlkwb{AH}\hlstd{}\hlopt{,}\hlstd{}\hlkwb{DH}\\
\hllin{28\ }\hlstd{}\hlkwa{MOV\ }\hlstd{P}\hlopt{,}\hlstd{}\hlkwb{AX}\\
\hllin{29\ }\hlstd{}\hlslc{;SUB\ METHOD}\\
\hllin{30\ }\hlstd{}\hlkwa{MOV\ }\hlstd{}\hlkwb{AX}\hlstd{}\hlopt{,}\hlstd{X}\\
\hllin{31\ }\hlkwa{SUB\ }\hlstd{}\hlkwb{AL}\hlstd{}\hlopt{,}\hlstd{}\hlkwb{DL}\\
\hllin{32\ }\hlstd{}\hlkwa{SBB\ }\hlstd{}\hlkwb{AH}\hlstd{}\hlopt{,}\hlstd{}\hlkwb{DH}\\
\hllin{33\ }\hlstd{}\hlkwa{MOV\ }\hlstd{M}\hlopt{,}\hlstd{}\hlkwb{AX}\\
\hllin{34\ }\hlstd{}\\
\hllin{35\ }\hlkwa{MOV\ }\hlstd{}\hlkwb{AH}\hlstd{}\hlopt{,}\hlstd{}\hlnum{04}\hlstd{}\hlkwb{CH}\\
\hllin{36\ }\hlstd{}\hlkwa{INT\ }\hlstd{}\hlnum{021H}\\
\hllin{37\ }\hlstd{CODE\ }\hlkwa{ENDS}\\
\hllin{38\ }\hlstd{}\hlkwa{END\ }\hlstd{START}\\
\mbox{}
\normalfont
\normalsize


\begin{figure}[b]
\centering
\includegraphics[width=100mm]{p21.png}
\caption{两个16位二进制数加减}
\label{p21}
\end{figure}
\clearpage
\subsection{压缩BCD码的加减}
在数据区定义两个压缩BCD码,程序首先将这两个数据读入到AX,DX中,之后对AL和DL进行ADD(不考虑进位)操作,之后使用DAA指令对AL和DL上的数据进行压缩BCD码调整,因为DAA指令只能对AL寄存器进行调整,因此将AL和AH寄存器数值进行交换XCHG,将AL和DH进行ADC操作(考虑进位),之后再次对AL上的值进行压缩BCD码调整,使用XCHG命令回到原始顺序下,输出结果

对于减法操作基本相同,只是用SUB命令代替ADD命令,用SBB命令代替ADC命令,而采用DAS命令代替DAA命令

程序如下所示,运行结果如图\ref{p211}所示,我们可以验证,98+99=99,98-99=99结果是正确的。 

\pagecolor{bgcolor}
\noindent
\ttfamily
\hlstd{\hllin{01\ }\\
\hllin{02\ }\\
\hllin{03\ }DATA\ }\hlkwa{SEGMENT}\\
\hllin{04\ }\hlstd{X\ }\hlkwa{DW\ }\hlstd{}\hlnum{00001H\ }\hlstd{}\hlslc{;THE\ VALUE\ 0F\ X}\\
\hllin{05\ }\hlstd{Y\ }\hlkwa{DW\ }\hlstd{}\hlnum{09999H}\hlstd{\ \ }\hlnum{}\hlstd{}\hlslc{;THE\ VALUE\ OF\ Y}\\
\hllin{06\ }\hlstd{P\ }\hlkwa{DW\ }\hlstd{}\hlnum{0}\hlstd{FFFFH\ }\hlslc{;X+Y}\\
\hllin{07\ }\hlstd{M\ }\hlkwa{DW\ }\hlstd{}\hlnum{0}\hlstd{FFFFH\ }\hlslc{;X{-}Y}\\
\hllin{08\ }\hlstd{}\hlslc{;Z=X+Y}\\
\hllin{09\ }\hlstd{DATA\ }\hlkwa{ENDS}\\
\hllin{10\ }\hlstd{\\
\hllin{11\ }STACK\ }\hlkwa{SEGMENT\ }\hlstd{PARA\ STACK}\\
\hllin{12\ }\hlkwa{DB\ }\hlstd{}\hlnum{10\ }\hlstd{DUP\ }\hlopt{(}\hlstd{?}\hlopt{)}\\
\hllin{13\ }\hlstd{STACK\ }\hlkwa{ENDS}\\
\hllin{14\ }\hlstd{\\
\hllin{15\ }CODE\ }\hlkwa{SEGMENT}\\
\hllin{16\ }\hlstd{}\hlkwa{ASSUME\ }\hlstd{}\hlkwb{DS}\hlstd{}\hlopt{:}\hlstd{DATA}\hlopt{,}\hlstd{}\hlkwb{ES}\hlstd{}\hlopt{:}\hlstd{DATA}\hlopt{,}\hlstd{}\hlkwb{SS}\hlstd{}\hlopt{:}\hlstd{STACK}\hlopt{,}\hlstd{}\hlkwb{CS}\hlstd{}\hlopt{:}\hlstd{CODE}\\
\hllin{17\ }\hlkwc{START:}\\
\hllin{18\ }\hlstd{}\hlkwa{MOV\ }\hlstd{}\hlkwb{AX}\hlstd{}\hlopt{,}\hlstd{DATA}\\
\hllin{19\ }\hlkwa{MOV\ }\hlstd{}\hlkwb{DS}\hlstd{}\hlopt{,}\hlstd{}\hlkwb{AX}\\
\hllin{20\ }\hlstd{}\hlkwa{MOV\ }\hlstd{}\hlkwb{ES}\hlstd{}\hlopt{,}\hlstd{}\hlkwb{AX}\\
\hllin{21\ }\hlstd{}\hlkwa{MOV\ }\hlstd{}\hlkwb{AX}\hlstd{}\hlopt{,}\hlstd{STACK}\\
\hllin{22\ }\hlkwa{MOV\ }\hlstd{}\hlkwb{SS}\hlstd{}\hlopt{,}\hlstd{}\hlkwb{AX}\\
\hllin{23\ }\hlstd{}\\
\hllin{24\ }\hlslc{;DATA\ READ\ IN}\\
\hllin{25\ }\hlstd{}\hlkwa{MOV\ }\hlstd{}\hlkwb{AX}\hlstd{}\hlopt{,}\hlstd{X}\\
\hllin{26\ }\hlkwa{MOV\ }\hlstd{}\hlkwb{DX}\hlstd{}\hlopt{,}\hlstd{Y}\\
\hllin{27\ }\hlslc{;ADD\ METHOD}\\
\hllin{28\ }\hlstd{}\hlkwa{ADD\ }\hlstd{}\hlkwb{AL}\hlstd{}\hlopt{,}\hlstd{}\hlkwb{DL}\\
\hllin{29\ }\hlstd{}\hlkwa{DAA}\\
\hllin{30\ }\hlstd{}\hlkwa{XCHG\ }\hlstd{}\hlkwb{AL}\hlstd{}\hlopt{,}\hlstd{}\hlkwb{AH}\\
\hllin{31\ }\hlstd{}\hlkwa{ADC\ }\hlstd{}\hlkwb{AL}\hlstd{}\hlopt{,}\hlstd{}\hlkwb{DH}\\
\hllin{32\ }\hlstd{}\hlkwa{DAA}\\
\hllin{33\ }\hlstd{}\hlkwa{XCHG\ }\hlstd{}\hlkwb{AL}\hlstd{}\hlopt{,}\hlstd{}\hlkwb{AH}\\
\hllin{34\ }\hlstd{}\hlkwa{MOV\ }\hlstd{P}\hlopt{,}\hlstd{}\hlkwb{AX}\\
\hllin{35\ }\hlstd{}\hlslc{;SUB\ METHOD}\\
\hllin{36\ }\hlstd{}\hlkwa{MOV\ }\hlstd{}\hlkwb{AX}\hlstd{}\hlopt{,}\hlstd{X}\\
\hllin{37\ }\hlkwa{SUB\ }\hlstd{}\hlkwb{AL}\hlstd{}\hlopt{,}\hlstd{}\hlkwb{DL}\\
\hllin{38\ }\hlstd{}\hlkwa{DAS}\\
\hllin{39\ }\hlstd{}\hlkwa{XCHG\ }\hlstd{}\hlkwb{AL}\hlstd{}\hlopt{,}\hlstd{}\hlkwb{AH}\\
\hllin{40\ }\hlstd{}\hlkwa{SBB\ }\hlstd{}\hlkwb{AL}\hlstd{}\hlopt{,}\hlstd{}\hlkwb{DH}\\
\hllin{41\ }\hlstd{}\hlkwa{DAS}\\
\hllin{42\ }\hlstd{}\hlkwa{XCHG\ }\hlstd{}\hlkwb{AL}\hlstd{}\hlopt{,}\hlstd{}\hlkwb{AH}\\
\hllin{43\ }\hlstd{}\hlkwa{MOV\ }\hlstd{M}\hlopt{,}\hlstd{}\hlkwb{AX}\\
\hllin{44\ }\hlstd{}\\
\hllin{45\ }\hlkwa{MOV\ }\hlstd{}\hlkwb{AH}\hlstd{}\hlopt{,}\hlstd{}\hlnum{04}\hlstd{}\hlkwb{CH}\\
\hllin{46\ }\hlstd{}\hlkwa{INT\ }\hlstd{}\hlnum{021H}\\
\hllin{47\ }\hlstd{CODE\ }\hlkwa{ENDS}\\
\hllin{48\ }\hlstd{}\hlkwa{END\ }\hlstd{START}\\
\mbox{}
\normalfont
\normalsize


\begin{figure}[b]
\centering
\includegraphics[width=\textwidth]{p211.png}
\caption{两个压缩BCD码的加减}
\label{p211}
\end{figure}
\clearpage
\subsection{压缩BCD码的乘法}
程序执行的策略是不断累加代替乘法,首先程序完成对数据的读入,存入DL和CL中,AX置零,其次程序进行循环,以下是循环的过程。

在一次循环中,程序将DL加入到AL中(ADD,不考虑进位),其次完成AL的BCD码调整DAA。之后,程序处理进位问题,将AH和0相加(ADC)处理进位,之后调换AL,AH的值,对AL(原AH)进行压缩BCD码调整,在将其换回原来位置。之后程序递减CL,并将其与AL交换进行压缩BCD码调整(DAS),再将其换回原位置

当CL最终被减至0的时候完成程序退出循环,同时完成数据的存储操作

程序如下所示,运行结果如图\ref{p22}所示,我们可以验证,77*7=539结果是正确的。
 
\pagecolor{bgcolor}
\noindent
\ttfamily
\hlstd{\hllin{01\ }\\
\hllin{02\ }\\
\hllin{03\ }DATA\ }\hlkwa{SEGMENT}\\
\hllin{04\ }\hlstd{X\ }\hlkwa{DB\ }\hlstd{}\hlnum{038H}\\
\hllin{05\ }\hlstd{Y\ }\hlkwa{DB\ }\hlstd{}\hlnum{023H}\\
\hllin{06\ }\hlstd{XY\ }\hlkwa{DW\ }\hlstd{}\hlnum{0}\hlstd{FFFFH}\hlslc{;X{*}Y}\\
\hllin{07\ }\hlstd{DATA\ }\hlkwa{ENDS}\\
\hllin{08\ }\hlstd{\\
\hllin{09\ }STACK\ }\hlkwa{SEGMENT\ }\hlstd{PARA\ STACK}\\
\hllin{10\ }\hlkwa{DB\ }\hlstd{}\hlnum{10\ }\hlstd{DUP\ }\hlopt{(}\hlstd{?}\hlopt{)}\\
\hllin{11\ }\hlstd{STACK\ }\hlkwa{ENDS}\\
\hllin{12\ }\hlstd{\\
\hllin{13\ }CODE\ }\hlkwa{SEGMENT}\\
\hllin{14\ }\hlstd{}\hlkwa{ASSUME\ }\hlstd{}\hlkwb{DS}\hlstd{}\hlopt{:}\hlstd{DATA}\hlopt{,}\hlstd{}\hlkwb{ES}\hlstd{}\hlopt{:}\hlstd{DATA}\hlopt{,}\hlstd{}\hlkwb{SS}\hlstd{}\hlopt{:}\hlstd{STACK}\hlopt{,}\hlstd{}\hlkwb{CS}\hlstd{}\hlopt{:}\hlstd{CODE}\\
\hllin{15\ }\hlkwc{START:}\\
\hllin{16\ }\hlstd{}\hlkwa{MOV\ }\hlstd{}\hlkwb{AX}\hlstd{}\hlopt{,}\hlstd{DATA}\\
\hllin{17\ }\hlkwa{MOV\ }\hlstd{}\hlkwb{DS}\hlstd{}\hlopt{,}\hlstd{}\hlkwb{AX}\\
\hllin{18\ }\hlstd{}\hlkwa{MOV\ }\hlstd{}\hlkwb{ES}\hlstd{}\hlopt{,}\hlstd{}\hlkwb{AX}\\
\hllin{19\ }\hlstd{}\hlkwa{MOV\ }\hlstd{}\hlkwb{AX}\hlstd{}\hlopt{,}\hlstd{STACK}\\
\hllin{20\ }\hlkwa{MOV\ }\hlstd{}\hlkwb{SS}\hlstd{}\hlopt{,}\hlstd{}\hlkwb{AX}\\
\hllin{21\ }\hlstd{}\\
\hllin{22\ }\hlslc{;DATA\ READ\ IN}\\
\hllin{23\ }\hlstd{}\hlkwa{MOV\ }\hlstd{}\hlkwb{DL}\hlstd{}\hlopt{,}\hlstd{X}\\
\hllin{24\ }\hlkwa{MOV\ }\hlstd{}\hlkwb{DH}\hlstd{}\hlopt{,}\hlstd{}\hlnum{000H}\\
\hllin{25\ }\hlstd{}\hlkwa{MOV\ }\hlstd{}\hlkwb{CL}\hlstd{}\hlopt{,}\hlstd{Y}\\
\hllin{26\ }\hlkwa{MOV\ }\hlstd{}\hlkwb{AX}\hlstd{}\hlopt{,}\hlstd{}\hlnum{00000H}\\
\hllin{27\ }\hlstd{}\hlkwc{L1:}\\
\hllin{28\ }\hlstd{}\hlkwa{ADD\ }\hlstd{}\hlkwb{AL}\hlstd{}\hlopt{,}\hlstd{}\hlkwb{DL}\\
\hllin{29\ }\hlstd{}\hlkwa{DAA\ }\\
\hllin{30\ }\hlstd{}\hlkwa{XCHG\ }\hlstd{}\hlkwb{AL}\hlstd{}\hlopt{,}\hlstd{}\hlkwb{AH}\\
\hllin{31\ }\hlstd{}\hlkwa{ADC\ }\hlstd{}\hlkwb{AL}\hlstd{}\hlopt{,}\hlstd{}\hlnum{0}\\
\hllin{32\ }\hlstd{}\hlkwa{DAA}\\
\hllin{33\ }\hlstd{}\hlkwa{XCHG\ }\hlstd{}\hlkwb{AH}\hlstd{}\hlopt{,}\hlstd{}\hlkwb{AL}\\
\hllin{34\ }\hlstd{}\hlkwa{DEC\ }\hlstd{}\hlkwb{CL}\\
\hllin{35\ }\hlstd{}\hlkwa{XCHG\ }\hlstd{}\hlkwb{AL}\hlstd{}\hlopt{,}\hlstd{}\hlkwb{CL}\\
\hllin{36\ }\hlstd{}\hlkwa{DAS}\\
\hllin{37\ }\hlstd{}\hlkwa{XCHG\ }\hlstd{}\hlkwb{AL}\hlstd{}\hlopt{,}\hlstd{}\hlkwb{CL}\\
\hllin{38\ }\hlstd{}\hlkwa{JNZ\ }\hlstd{L1}\\
\hllin{39\ }\hlkwa{MOV\ }\hlstd{XY}\hlopt{,}\hlstd{}\hlkwb{AX}\\
\hllin{40\ }\hlstd{}\\
\hllin{41\ }\hlkwa{MOV\ }\hlstd{}\hlkwb{AH}\hlstd{}\hlopt{,}\hlstd{}\hlnum{04}\hlstd{}\hlkwb{CH}\\
\hllin{42\ }\hlstd{}\hlkwa{INT\ }\hlstd{}\hlnum{021H}\\
\hllin{43\ }\hlstd{CODE\ }\hlkwa{ENDS}\\
\hllin{44\ }\hlstd{}\hlkwa{END\ }\hlstd{START}\\
\mbox{}
\normalfont
\normalsize


\begin{figure}[b]
\centering
\includegraphics[width=\textwidth]{p22.png}
\caption{两个压缩BCD码的乘法}
\label{p22}
\end{figure}
\clearpage
\subsection{ASCII码的乘法}
程序执行的策略是将一个数的每一位分别和另外一个数相乘处理。

程序首先将个位数Y读入CH中,并将CL赋值为X的位数4,之后将DI指向X的首地址(最低位),将SI指向XY的首地址(最低位)开始进行四次循环,DH置零,代表低位的进位。

每次循环过程中,首先程序将X的当前位读入到AL中,并和00FH做与处理从ASCII码得到数字,之后和CH进行乘法运算,对AX进行非压缩BCD码调整处理AAM,将低位AL和上一次的进位DH相加,进行非压缩BCD码调整AAA命令。在之后将DH重新幅值为AH,即为本位向上的进位信息,将AL加030H转换为ASCII码,存入SI所指向的位置,两个指针分别递增,CL递减完成一次循环

最后循环执行完成后即完成了乘法操作,如果此时DH仍存在进位则视为溢出处理。

程序如下所示,运行结果如图\ref{p221}所示,我们可以验证,2754*3=8262结果是正确的。

\pagecolor{bgcolor}
\noindent
\ttfamily
\hlstd{\hllin{01\ }\\
\hllin{02\ }\\
\hllin{03\ }DATA\ }\hlkwa{SEGMENT}\\
\hllin{04\ }\hlstd{X\ }\hlkwa{DB\ }\hlstd{}\hlstr{'4572'}\hlstd{\\
\hllin{05\ }Y\ }\hlkwa{DB\ }\hlstd{}\hlstr{'3'}\hlstd{\\
\hllin{06\ }XY\ }\hlkwa{DW\ }\hlstd{}\hlnum{0}\hlstd{FFFFH}\hlslc{;X{*}Y}\\
\hllin{07\ }\hlstd{DATA\ }\hlkwa{ENDS}\\
\hllin{08\ }\hlstd{\\
\hllin{09\ }STACK\ }\hlkwa{SEGMENT\ }\hlstd{PARA\ STACK}\\
\hllin{10\ }\hlkwa{DB\ }\hlstd{}\hlnum{10\ }\hlstd{DUP\ }\hlopt{(}\hlstd{?}\hlopt{)}\\
\hllin{11\ }\hlstd{STACK\ }\hlkwa{ENDS}\\
\hllin{12\ }\hlstd{\\
\hllin{13\ }CODE\ }\hlkwa{SEGMENT}\\
\hllin{14\ }\hlstd{}\hlkwa{ASSUME\ }\hlstd{}\hlkwb{DS}\hlstd{}\hlopt{:}\hlstd{DATA}\hlopt{,}\hlstd{}\hlkwb{ES}\hlstd{}\hlopt{:}\hlstd{DATA}\hlopt{,}\hlstd{}\hlkwb{SS}\hlstd{}\hlopt{:}\hlstd{STACK}\hlopt{,}\hlstd{}\hlkwb{CS}\hlstd{}\hlopt{:}\hlstd{CODE}\\
\hllin{15\ }\hlkwc{START:}\\
\hllin{16\ }\hlstd{}\hlkwa{MOV\ }\hlstd{}\hlkwb{AX}\hlstd{}\hlopt{,}\hlstd{DATA}\\
\hllin{17\ }\hlkwa{MOV\ }\hlstd{}\hlkwb{DS}\hlstd{}\hlopt{,}\hlstd{}\hlkwb{AX}\\
\hllin{18\ }\hlstd{}\hlkwa{MOV\ }\hlstd{}\hlkwb{ES}\hlstd{}\hlopt{,}\hlstd{}\hlkwb{AX}\\
\hllin{19\ }\hlstd{}\hlkwa{MOV\ }\hlstd{}\hlkwb{AX}\hlstd{}\hlopt{,}\hlstd{STACK}\\
\hllin{20\ }\hlkwa{MOV\ }\hlstd{}\hlkwb{SS}\hlstd{}\hlopt{,}\hlstd{}\hlkwb{AX}\\
\hllin{21\ }\hlstd{}\\
\hllin{22\ }\hlslc{;DATA\ READ\ IN}\\
\hllin{23\ }\hlstd{}\hlkwa{MOV\ }\hlstd{}\hlkwb{CH}\hlstd{}\hlopt{,}\hlstd{Y}\\
\hllin{24\ }\hlkwa{AND\ }\hlstd{}\hlkwb{CH}\hlstd{}\hlopt{,}\hlstd{}\hlnum{00}\hlstd{FH}\\
\hllin{25\ }\hlkwa{MOV\ }\hlstd{}\hlkwb{CL}\hlstd{}\hlopt{,}\hlstd{}\hlnum{4}\\
\hllin{26\ }\hlstd{}\hlkwa{MOV\ }\hlstd{}\hlkwb{DH}\hlstd{}\hlopt{,}\hlstd{}\hlnum{0}\\
\hllin{27\ }\hlstd{}\hlkwa{LEA\ }\hlstd{}\hlkwb{DI}\hlstd{}\hlopt{,}\hlstd{X}\\
\hllin{28\ }\hlkwa{LEA\ }\hlstd{}\hlkwb{SI}\hlstd{}\hlopt{,}\hlstd{XY}\\
\hllin{29\ }\hlkwc{L1:}\\
\hllin{30\ }\hlstd{}\hlkwa{MOV\ }\hlstd{}\hlkwb{AL}\hlstd{}\hlopt{,{[}}\hlstd{}\hlkwb{DI}\hlstd{}\hlopt{{]}}\\
\hllin{31\ }\hlstd{}\hlkwa{AND\ }\hlstd{}\hlkwb{AL}\hlstd{}\hlopt{,}\hlstd{}\hlnum{00}\hlstd{FH}\\
\hllin{32\ }\hlkwa{MUL\ }\hlstd{}\hlkwb{CH}\\
\hllin{33\ }\hlstd{}\hlkwa{AAM\ }\\
\hllin{34\ }\hlstd{}\hlkwa{ADD\ }\hlstd{}\hlkwb{AL}\hlstd{}\hlopt{,}\hlstd{}\hlkwb{DH}\\
\hllin{35\ }\hlstd{}\hlkwa{AAA}\\
\hllin{36\ }\hlstd{}\hlkwa{MOV\ }\hlstd{}\hlkwb{DH}\hlstd{}\hlopt{,}\hlstd{}\hlkwb{AH}\\
\hllin{37\ }\hlstd{}\hlkwa{ADD\ }\hlstd{}\hlkwb{AL}\hlstd{}\hlopt{,}\hlstd{}\hlnum{030H}\\
\hllin{38\ }\hlstd{}\hlkwa{MOV\ }\hlstd{}\hlopt{{[}}\hlstd{}\hlkwb{SI}\hlstd{}\hlopt{{]},}\hlstd{}\hlkwb{AL}\\
\hllin{39\ }\hlstd{}\hlkwa{INC\ }\hlstd{}\hlkwb{SI}\\
\hllin{40\ }\hlstd{}\hlkwa{INC\ }\hlstd{}\hlkwb{DI}\\
\hllin{41\ }\hlstd{}\hlkwa{DEC\ }\hlstd{}\hlkwb{CL}\\
\hllin{42\ }\hlstd{}\hlkwa{JNZ\ }\hlstd{L1}\\
\hllin{43\ }\\
\hllin{44\ }\hlkwa{MOV\ }\hlstd{}\hlkwb{AH}\hlstd{}\hlopt{,}\hlstd{}\hlnum{04}\hlstd{}\hlkwb{CH}\\
\hllin{45\ }\hlstd{}\hlkwa{INT\ }\hlstd{}\hlnum{021H}\\
\hllin{46\ }\hlstd{CODE\ }\hlkwa{ENDS}\\
\hllin{47\ }\hlstd{}\hlkwa{END\ }\hlstd{START}\\
\mbox{}
\normalfont
\normalsize

\begin{figure}[b]
\centering
\includegraphics[width=\textwidth]{p221.png}
\caption{两个压缩BCD码的乘法}
\label{p221}
\end{figure}
\clearpage
\section{完成情况及心得体会}
通过这次实验,我进一步熟悉了DOS 环境与DEBUG 的使用,对课堂能讲解的运算指令有了更深入地了解,也学会了使用E 指令来修改原始数据。因为实验之前的准备比较充分,已经完成了大部分的调试工作,因此整个实验也进行的比较顺利。
\end{document}